% \setcounter{secnumdepth}{5}  
% \usepackage{leipzig}
% \usepackage{gb4e}
% \noautomath
% \usepackage{tipa}
% \usepackage{here}
% \usepackage[dvipdfmx]{graphicx}
% \usepackage{multirow}
% \usepackage{longtable}
% \usepackage{diagbox}
% \usepackage{ifthen}
% \usepackage{xstring} % xstringパッケージを使用
% \usepackage[utf8]{inputenc}
% \usepackage[T1]{fontenc}
% \usepackage{lexikon}
% \usepackage{threeparttable}
% \usepackage{caption}
\setcounter{secnumdepth}{5}
\usepackage{leipzig}
\usepackage{gb4e}
\noautomath
\usepackage{tipa}
\usepackage{here}
\usepackage{graphicx}
\usepackage{multirow}
\usepackage{longtable}
\usepackage{diagbox}
\usepackage{ifthen}
\usepackage{xstring}
\usepackage{lexikon}
\usepackage{threeparttable}
\usepackage{caption}
% \usepackage{zxjfont}
\usepackage{booktabs}
\usepackage{hyperref}

% 文字列strの中のstr1をstr2に置き換えるマクロ
\newcommand{\changestr}[3]{%
  \StrSubstitute{#1}{#2}{#3}[\Result]%
  \Result
}

\setlength{\parindent}{0pt}
\makeglossaries
\newleipzig{red}{red}{reduplication}
\newleipzig{indef}{indef}{indefinite}
\newleipzig{mov}{mov}{movement case}
\newleipzig{inter}{inter}{interrogative}
\newleipzig{cl}{cl}{class}
\newleipzig{inch}{inch}{inchoative}
%\newleipzig{prog}{prog}{progressive}

\makeatletter
\newcommand{\subsubsubsection}{\@startsection{paragraph}{4}{\z@}%
  {1.0\Cvs \@plus.5\Cdp \@minus.2\Cdp}%
  {.1\Cvs \@plus.3\Cdp}%
  {\reset@font\sffamily\normalsize}
}
\makeatother
\setcounter{secnumdepth}{4}

\makeatletter
\newcommand{\subsubsubsubsection}{\@startsection{subparagraph}{5}{\z@}%
  {1.0\Cvs \@plus.5\Cdp \@minus.2\Cdp}%
  {.1\Cvs \@plus.3\Cdp}%
  {\reset@font\sffamily\normalsize}
}
\makeatother
\setcounter{secnumdepth}{5}

\def\langname{Kaapi語}
\def\P{\textipa{P}}
\def\E{\textipa{E}}
\def\O{\textipa{O}}
\def\I{\textipa{I}}
\def\U{\textipa{U}}

%\newcommand{\mylang}[1]{\'#1}

%------------------------
\makeatletter
%
\newcommand{\ZenbuOnpu}[1]{%
\@tfor\moji:=#1\do{\moji♪}
}
%
\newcommand{\BadExample}[1]{%
\@tfor\moji:={#1}\do{\moji♪}
}
%


\newcommand{\mylang}[1]{%
\@tfor\moji:=#1\do%
{%
  \myletter{\moji}%
}%
}%

\newcommand{\mydict}[2]{%
  \begin{supertabular}[l]{ll}%
    \multicolumn{2}{l}{\textbf{#1}}\\%
    \,\,\,\,\,\, & #2%
    \\%
  \end{supertabular}%
}%
\newcommand{\dictsection}[1]{%
{\Large\textbf{#1}}%
\addcontentsline{toc}{subsection}{#1}%
\,\\
\,\\
}%



\makeatother

%------------------------

%たとえば、「りおれうす」をこのコマンドに入れると、
%\ZenbuOnpu{りおれうす}
%となり、うっとおしい。

%なお、変なところに括弧を入れると、意図しない動作になる。
%やってみると、
%\BadExample{りおれうす}
%となる。

%たとえば、「りおれうす」をこのコマンドに入れると、り♪お♪れ♪う♪す♪となり、うっとおしい。
%なお、変なところに括弧を入れると、意図しない動作になる。やってみると、りおれうす♪となる。

%\def\emph#1is#2.{\textbf{#1}is\textit{#2}.}





\setcounter{tocdepth}{2}

\captionsetup[table]{labelsep=period, labelfont=bf, justification=raggedright, singlelinecheck=off}