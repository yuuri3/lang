\paragraph{音節構造}\quad\\
\langname の許される音節構造は CV,V の2種類のみである。

% TODO : IPA 現状エラー
\begin{tabular}{llll}
    \multicolumn{2}{l}{例 : } \\ 
    & u\'e & 良い \\
    & kuki & 目 \\
\end{tabular}

子音連続や末子音は存在せず、借用語においてはそれらを回避するために子音後に母音/i/が挿入される。
% 提案: 
% この現象は**「音節構造の調整ストラテジー」**として学術的に重要です。挿入される母音が常に /i/ であるなら、その音韻的根拠(例:中和された母音、または最も軽微な音響的変更)を簡単に説明すべきです。

\begin{tabular}{llll}
    \multicolumn{2}{l}{例 : } \\ 
    & k\'iritimati & 年齢(トクピシン krismas より) \\
\end{tabular}

語頭以外にV音節が発生したとき前音節との境界で母音連続が発生する。
連続する母音の種類に制限はなく、最大で3つの母音連続が発生する。

\begin{tabular}{llll}
    \multicolumn{2}{l}{例 : } \\ 
    & kuia & 豚 \\
    & kuki & 目 \\
\end{tabular}

音節内の母音にはそれぞれ2段階の声調のどちらかが付属する。
高声調(H)は C\'V/\'V 、低声調(L)は CV/V で表される。

\begin{tabular}{llll}
    \multicolumn{2}{l}{例 : } \\ 
    & m\'ema & 蛇 \\
    & meme & 犬 \\
\end{tabular}

\paragraph{語の音節数}\quad\\
\langname の語根は多くが2~3つの音節からなる。
1音節の語根は多くが接辞であり、4音節の語根はほとんどない。

\begin{tabular}{llll}
    \multicolumn{2}{l}{例 : } \\ 
    & -p\'a & 完了の接尾辞 \\
    & vet\'e & 水 \\
    & \'at\'ir\'i & 魚 \\
    & r\'iv\'or\'iv\'o & 蝶 \\
\end{tabular}

これらの語根に接辞が付加したり、語根どうしが複合したりすることで、語はより多くの音節を含むようになる。

\begin{tabular}{llll}
    \multicolumn{2}{l}{例 : } \\ 
    & vet\'e-vet\'e & 雨 \\
    & \'at\'ir\'i-r\'i & 魚(複数) \\
\end{tabular}
% 提案: 
% (1) 例として示されている vet'e-vet'e (雨) と 'at'ir'i-r'i (魚(複数)) は、それぞれ畳語(reduplication)と接尾辞の使用を示唆しています。
% (2) この接尾辞 -r'i が複数形を担う形態素であることを確認し、その形態音韻論的な振る舞いを別セクションで詳細に記述すべきです。