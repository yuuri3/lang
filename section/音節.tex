\paragraph{音節構造}\quad\\
\langname の許される音節構造は CV,V の2種類のみである。
% 提案: 
% (1) 音節構造の定義に加えて、可能な母音(V)と子音(C)のレパートリー(音素目録)の概要を記述することで、記述の完全性が増します。
% (2) V型音節の分布を明確にしてください。語頭のみか、語頭・語中・語末全てで許容されるか。

% TODO : IPA 現状エラー
\begin{tabular}{llll}
    \multicolumn{2}{l}{例 : } \\ 
    & u\'e & 良い \\
    & kuki & 目 \\
\end{tabular}
% 提案: 
% (1) u'e の音節境界を明示すべきです(例:V.CV または V.V)。(2) u'e の音訳が u'e である点について、声調記号(')と母音の音韻的性質(長母音ではないか、二重母音ではないか)を明確にするため、国際音声記号(IPA)による転写が必要です。これはTODOコメント(IPA 現状エラー)でも指摘されています。

子音連続や末子音は存在せず、借用語においてはそれらを回避するために子音後に母音/i/が挿入される。
% 提案: 
% この現象は**「音節構造の調整ストラテジー」**として学術的に重要です。挿入される母音が常に /i/ であるなら、その音韻的根拠(例:中和された母音、または最も軽微な音響的変更)を簡単に説明すべきです。

\begin{tabular}{llll}
    \multicolumn{2}{l}{例 : } \\ 
    & k\'iritimati & 年齢(トクピシン krismas より) \\
\end{tabular}

音節内の母音にはそれぞれ2段階の声調のどちらかが付属する。
% 提案: 
% (1) 2段階の声調が高低(H/L)のどちらに対応するのかを明記し、音韻的な声調表記(例:高声調 H /L、または á/a)を定義し、例に適用すべきです。
% (2) 声調の音韻的ステータス(音素的か否か)や音韻規則(例:声調の拡散、連声変化など)の存在について言及することで、この言語の記述が深まります。

\paragraph{語の音節数}\quad\\
\langname の語根は多くが2~3つの音節からなる。
% 提案: 
% 語根の音節数が**最小(例:1音節は不可)および最大(例:4音節を超えるか)**の制約を明確に記述すると、より厳密になります。

\begin{tabular}{llll}
    \multicolumn{2}{l}{例 : } \\ 
    & vet\'e & 水 \\
    & \'at\'ir\'i & 魚 \\
\end{tabular}
% 提案: 
% これらの例でも、声調記号の定義と、IPAによる厳密な音節境界の表記(例:/ve.té/)が必要です。

これらの語根に接辞が付加したり、語根どうしが複合したりすることで、語はより多くの音節を含むようになる。

\begin{tabular}{llll}
    \multicolumn{2}{l}{例 : } \\ 
    & vet\'e-vet\'e & 雨 \\
    & \'at\'ir\'i-r\'i & 魚(複数) \\
\end{tabular}
% 提案: 
% (1) 例として示されている vet'e-vet'e (雨) と 'at'ir'i-r'i (魚(複数)) は、それぞれ畳語(reduplication)と接尾辞の使用を示唆しています。
% (2) この接尾辞 -r'i が複数形を担う形態素であることを確認し、その形態音韻論的な振る舞いを別セクションで詳細に記述すべきです。