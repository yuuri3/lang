\paragraph{音節構造}\quad\\
\langname の許される音節構造は CV,V の2種類のみである。

% TODO : IPA 現状エラー
\begin{tabular}{llll}
    \multicolumn{2}{l}{例 : } \\ 
    & u\'e & 良い \\
    & kuki & 目 \\
\end{tabular}

子音連続や末子音は存在せず、借用語においてはそれらを回避するために子音後に母音/i/が挿入される。
% TODO /i/ を挿入する音韻論的根拠

\begin{tabular}{llll}
    \multicolumn{2}{l}{例 : } \\ 
    & k\'iritimati & 年齢(トクピシン krismas より) \\
\end{tabular}

V音節は語のあらゆる箇所に出現する。
語頭以外にV音節が発生したとき前音節との境界で母音連続が発生する。
連続する母音の種類に制限はなく、最大で3つの母音連続が発生する。

\begin{tabular}{llll}
    \multicolumn{2}{l}{例 : } \\ 
    & kuia & 豚 \\
    & kuki & 目 \\
\end{tabular}

音節内の母音にはそれぞれ2段階の声調のどちらかが付属する。
高声調(H)は C\'V/\'V 、低声調(L)は CV/V で表される。
% 提案: 
% 声調の音韻的ステータス(全ての単語で声調が弁別的か、特定の環境でのみ予測可能か)や、声調の音韻規則(例:隣接する音節への声調の拡散、畳語における声調変化など)の存在について言及することで、分析がさらに深まります。

\begin{tabular}{llll}
    \multicolumn{2}{l}{例 : } \\ 
    & m\'ema & 蛇 \\
    & meme & 犬 \\
\end{tabular}

\paragraph{語の音節数}\quad\\
\langname の語根は多くが2~3つの音節からなる。
1音節の語根は多くが接辞であり、4音節の語根はほとんどない。

\begin{tabular}{llll}
    \multicolumn{2}{l}{例 : } \\ 
    & -p\'a & 完了の接尾辞 \\
    & vet\'e & 水 \\
    & \'at\'ir\'i & 魚 \\
    & r\'iv\'or\'iv\'o & 蝶 \\
\end{tabular}
% 提案: r'iv'or'iv'o が「蝶」という単一の語根として記述されている場合、この単語が畳語ではないことを確認してください。もしこれが r'ivo の畳語形である場合、それを明記し、語根の制約(最大3音節)と整合性を保つべきです。

これらの語根に接辞が付加したり、語根どうしが複合したりすることで、語はより多くの音節を含むようになる。

\begin{tabular}{llll}
    \multicolumn{2}{l}{例 : } \\ 
    & vet\'e-vet\'e & 雨 \\
    & \'at\'ir\'i-r\'i & 魚(複数) \\
\end{tabular}