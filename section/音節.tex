\paragraph{音節構造}\quad\\
\langname の許される音節構造は CV,V の2種類のみである。

% TODO : IPA 現状エラー
\begin{tabular}{llll}
    & u\'e & 良い \\
    & kuki & 目 \\
\end{tabular}

子音連続や末子音は存在せず、借用語においてはそれらを回避するために子音後に母音/i/が挿入される。
% TODO /i/ を挿入する音韻論的根拠

\begin{tabular}{llll}
    & k\'iritimati & 年齢(トクピシン krismas より) \\
\end{tabular}

V音節は語のあらゆる箇所に出現する。
語頭以外にV音節が発生したとき前音節との境界で母音連続が発生する。
連続する母音の種類に制限はなく、最大で3つの母音連続が発生する。

\begin{tabular}{llll}
    & kuia & 豚 \\
\end{tabular}

音節内の各母音には必ず2段階の声調のどちらかが付属する。
高声調(H)は C\'V/\'V 、低声調(L)は CV/V で表される。

\begin{tabular}{llll}
    & m\'ema & 蛇 \\
    & meme & 犬 \\
\end{tabular}

\paragraph{語の音節数}\quad\\
\langname の語根は多くが2~3つの音節からなる。
1音節の語根は多くが接辞であり、4音節の語根はほとんどない。

\begin{tabular}{llll}
    & -p\'a & 完了の接尾辞 \\
    & vet\'e & 水 \\
    & \'at\'ir\'i & 魚 \\
    & r\'iv\'or\'iv\'o & 蝶(単一の語根) \\
\end{tabular}

これらの語根に接辞が付加したり、語根どうしが複合したりすることで、語はより多くの音節を含むようになる。

\begin{tabular}{llll}
    & vet\'e-vet\'e & 雨 \\
    & \'at\'ir\'i-r\'i & 魚(複数) \\
\end{tabular}