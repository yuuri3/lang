\paragraph{音節構造}\quad\\
\langname の許される音節構造は CV,V の2種類のみである。

% TODO : IPA 現状エラー
\begin{tabular}{llll}
    \multicolumn{2}{l}{例 : } \\ 
    & u\'e & 良い \\
    & kuki & 目 \\
\end{tabular}

子音連続や末子音は存在せず、借用語においてはそれらを回避するために子音後に母音/i/が挿入される。

\begin{tabular}{llll}
    \multicolumn{2}{l}{例 : } \\ 
    & k\'iritimati & 年齢(トクピシン krismas より) \\
\end{tabular}

音節内の母音にはそれぞれ2段階の声調のどちらかが付属する。

\paragraph{語の音節数}\quad\\
\langname の語根は多くが2~3つの音節からなる。

\begin{tabular}{llll}
    \multicolumn{2}{l}{例 : } \\ 
    & vet\'e & 水 \\
    & \'at\'ir\'i & 魚 \\
\end{tabular}

これらの語根に接辞が付加したり、語根どうしが複合したりすることで、語はより多くの音節を含むようになる。

\begin{tabular}{llll}
    \multicolumn{2}{l}{例 : } \\ 
    & vet\'e-vet\'e & 雨 \\
    & \'at\'ir\'i-r\'i & 魚(複数) \\
\end{tabular}