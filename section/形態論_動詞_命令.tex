動詞に対して\textbf{命令法 (Imperative)} の意味を付与する場合、接辞 \textbf{-wi} を用いる。
この命令接辞は、\textbf{相接辞(Aspect Marker)}の直後に後置される。

命令形を構成する際、動詞語幹に付加される相接辞は、例外なく \textbf{進行相 (Progressive Aspect)} 
を示す形態 \textbf{-ak'e} を取ることが義務付けられている。他の相形態との共起は許可されない。

この構造は以下のスキーマで表される。
%
\begin{enumerate}
    \item $[ \text{V}_{\text{root}} ] - \textbf{ak'e}_{\text{PROG}} - \textbf{wi}_{\text{IMP}} - [ \text{Agreement/Other Suffixes} ]$
\end{enumerate}
%
接辞 \textit{-wi} は、動詞の表す動作や状態を話者が聞き手に実行するよう促す\textbf{命令接辞}として
機能する。

命令法の表現において、接辞 \textbf{-wi} は\textbf{文体 (Style)} や
\textbf{レジスター (Register)} に応じた\textbf{異形態的変異}を示す。
特に、\textbf{乱暴な表現}や\textbf{親密で非公式な状況}において、
命令接辞 \textit{-wi} は省略されることがある。
この場合、命令の意味は進行相接辞 \textit{-ak'e} のみで担われ、文脈によって命令法として解釈される。

以下の例は、動詞 \textit{qipu} 「行く」が進行相接辞 \textit{-ak'e} と命令接辞 \textit{-wi} を
伴うことで、命令の意味「行け」を構成することを示す。

\begin{exe}
\ex \gll qipu-ak'e-wi \\
        go-PROG-IMP \\
        \glt `行け' \hfill (\textit{標準的/中立的命令})
\end{exe}

\begin{exe}
\ex \gll qipu-ak'e \\
        go-PROG \\
        \glt `行け' \hfill (\textit{乱暴な/非公式な命令})
\end{exe}
%
例文 (\ref{exe}: ex2) は、\textit{-wi} が省略された形式であり、
話し手が聞き手に対してより直接的、あるいは乱暴な口調を用いていることを示す。
この省略された形態は、\textit{-wi} が必須ではなく、社会言語的な機能を持つことを示唆する。