\label{sec:introduction}

本稿は、ニューギニア島近郊の無名の島嶼部の沿岸に位置する複数の村落で、現在、推定
約600人の話者によって話されている\langname の包括的な記述的研究を目的とする。
この地域の言語環境は生物学的・言語的多様性のホットスポットとして知られている。

\langname の系統関係は未だ立証されておらず、現在のところ
孤立した言語(language isolate)として扱われている。
歴史的比較研究を試みる上での資料不足と、言語の類型的な特異性から、近隣の主要な語族との明確な繋がりは
見出されていない。
% TODO : 削除 言語の類型的な特異性から
しかし、地理的に近接するオーストロネシア語族に属する諸言語とのわずかな
接触による語彙的・音韻的な影響が断片的に示唆されており、本研究の記述的分析を通じて、
これらの借用層の特定と言語年代学的な検証を行う。
% TODO : 削除 これらの借用層の特定と言語年代学的な検証を行う。

\langname 話者の大半はモノリンガルであり、当該島嶼部の村落内部には他言語話者が存在しない
という極めて特異な言語環境が維持されている。
この高度な言語的孤立は、外部からの言語接触の圧力が極めて少なく、\langname が
% TODO : 表現 : 言語的孤立
独自の音韻、形態、統語構造を発展させてきた可能性を示唆する。
% TODO : 示唆するなど、結論を導く文を序論には置かない。

本研究は、\langname の構造を記録し、その内在的な法則性を確立することを目的とする。
本研究は以下の3つの主要な部分から構成される。

\begin{enumerate}
    \item \textbf{音韻論}: \langname の分節音および超分節音素の目録を提示し、
    音節構造と音韻規則を詳述する。
    \item \textbf{形態論}: 語類体系を確立し、特に名詞の格体系および形容詞の非屈折的な性質に焦点を
    当てて記述する。
    % TODO : 名詞に着目するわけではない。
    \item \textbf{統語論}: 句構造規則、主要部(head)の配置、および基本的な文型を分析し、
    特にSOV、SVOなどの類型論的位置づけを議論する。
    % TODO : 形態論、統語論の定義について調べる。
\end{enumerate}