\label{sec:introduction}

本稿は、ニューギニア島近郊の無名の島嶼部の沿岸に位置する複数の村落で、現在、推定
約600人の話者によって話されている\langname の包括的な記述的研究を目的とする。

\langname 話者の大半はモノリンガルであり、当該島嶼部の村落内部には他言語話者が存在しない。
\langname の系統関係は未だ立証されておらず、現在のところ
孤立した言語(language isolate)として扱われている。
歴史的比較研究を試みる上での資料不足から、近隣の主要な語族との明確な繋がりは
見出されていない。
しかし、地理的に近接するオーストロネシア語族に属する諸言語とのわずかな
接触による語彙的・音韻的な影響が断片的に示唆されている。
本論では周辺言語との関係性については示唆するにとどめ、具体的な調査はしない。

本研究は、\langname の構造を記録し、その内在的な法則性を確立することを目的とする。
本研究は以下の3つの主要な部分から構成される。

\begin{enumerate}
    \item \textbf{音韻論}: \langname の分節音および超分節音素の目録を提示し、
    音節構造と音韻規則を詳述する。
    \item \textbf{統語論}: 句構造規則、主要部(head)の配置、および基本的な文型を分析する。
    \item \textbf{形態論}: 語類体系を確立し、品詞と接辞の体系をまとめる。
\end{enumerate}