\paragraph{母音の発音}
以下に、各母音の語内で実現される音声について説明する。
\langname は母音の高さに関して4段階の弁別的な対立を持つ。

\subparagraph{/a/}
/a/は後舌広母音\textipa{[A]}として発音される。

\begin{tabular}{llll}
    & \textipa{apa [ApA]~[AFA]} & 人 \\
\end{tabular}

\subparagraph{/\textipa{E}/}
/\textipa{E}/は前舌半広母音\textipa{[E]}として発音される。

\begin{tabular}{llll}
    & \textipa{\'Ena [EnA]} & 指さす \\
\end{tabular}

\subparagraph{/e/}
/e/は前舌半狭母音\textipa{[e]}として発音される。

\begin{tabular}{llll}
    & \textipa{keni [keni]} & 言葉 \\
\end{tabular}

\subparagraph{/i/}
/i/は前舌狭母音\textipa{[i]}として発音される。

\begin{tabular}{llll}
    & \textipa{\'\i p\'a [ipA]~[iFA]} & 鳥の一種 \\
\end{tabular}

\subparagraph{/\textipa{O}/}
/\textipa{O}/は後舌半広母音\textipa{[O]}として発音される。

\begin{tabular}{llll}
    & \textipa{Or\'E [ORE]} & 太った \\
\end{tabular}

\subparagraph{/o/}
/o/は後舌半狭母音\textipa{[o]}として発音される。

\begin{tabular}{llll}
    & \textipa{ova\'o [oBAo]~[o\textlowering{B}Ao]} & 人 \\
\end{tabular}

\subparagraph{/u/}
/u/は後舌狭母音\textipa{[u]}として発音される。

\begin{tabular}{llll}
    & \textipa{u\'E [uE]} & 良い \\
\end{tabular}

% TODO 異音の記述の充実

\paragraph{ミニマルペア}

以下に、母音のミニマルペアを例示する。

\subparagraph{a-\textipa{O}}
% TODO 例
%\begin{tabular}{lllll}
%    & keni & 言葉 & keni & 言葉 \\
% \end{tabular}

\subparagraph{\textipa{O}-o}
% TODO 例

\subparagraph{o-u}
% TODO 例

\subparagraph{a-\textipa{E}}
% TODO 例

\subparagraph{\textipa{E}-e}
% TODO 例

\subparagraph{e-i}
% TODO 例

\subparagraph{\textipa{O}-\textipa{E}}
% TODO 例

\subparagraph{o-e}
% TODO 例

\subparagraph{u-i}
% TODO 例