\paragraph{母音の発音}\quad\\
以下に、各母音の語内で実現される音声について説明する。
\langname は母音の高さに関して4段階の弁別的な対立を持つ。

\subparagraph{/a/}
/a/は後舌広母音\textipa{[A]}として発音される。
% TODO 例

\subparagraph{/\textipa{E}/}
/\textipa{E}/は前舌半広母音\textipa{[E]}として発音される。
% TODO 例

\subparagraph{/e/}
/e/は前舌半狭母音\textipa{[e]}として発音される。
% TODO 例

\subparagraph{/i/}
/i/は前舌狭母音\textipa{[i]}として発音される。
% TODO 例

\subparagraph{/\textipa{O}/}
/\textipa{O}/は後舌半広母音\textipa{[O]}として発音される。
% TODO 例

\subparagraph{/o/}
/o/は後舌半狭母音\textipa{[o]}として発音される。
% TODO 例

\subparagraph{/u/}
/u/は前舌狭母音\textipa{[u]}として発音される。
% TODO 例

% TODO 異音の記述の充実