\section{序論}\label{sec:introduction}

本稿は、ニューギニア島近郊の無名の島嶼部の沿岸に位置する複数の村落で、現在、推定
約600人の話者によって話されている\langname の包括的な記述的研究を目的とする。

\langname 話者の大半はモノリンガルであり、当該島嶼部の村落内部には他言語話者が存在しない。
\langname の系統関係は未だ立証されておらず、現在のところ
孤立した言語(language isolate)として扱われている。
歴史的比較研究を試みる上での資料不足から、近隣の主要な語族との明確な繋がりは
見出されていない。
しかし、地理的に近接するオーストロネシア語族に属する諸言語とのわずかな
接触による語彙的・音韻的な影響が断片的に示唆されている。
本論では周辺言語との関係性については示唆するにとどめ、具体的な調査はしない。

本研究は、\langname の構造を記録し、その内在的な法則性を確立することを目的とする。
本研究は以下の3つの主要な部分から構成される。

\begin{enumerate}
    \item \textbf{音韻論}: \langname の分節音および超分節音素の目録を提示し、
    音節構造と音韻規則を詳述する。
    \item \textbf{統語論}: 句構造規則、主要部(head)の配置、および基本的な文型を分析する。
    \item \textbf{形態論}: 語類体系を確立し、品詞と接辞の体系をまとめる。
\end{enumerate}

% TODO セクションの順序の理由 
\newpage

\section{音韻}\langname の音韻体系は、以下の表に示すように8つの子音と5つの母音からなる。
これらの音素は、語頭、語中、および母音の前後といった特定の音環境において、
さまざまな異音を伴って現れる。 
\subsection{子音}\langname の子音は表\ref{tab:consonants}に示された7種類である。

\begin{table}[H]
    \centering
    \begin{tabular}{lccc}
        \toprule
        & \textbf{両唇音} & \textbf{歯茎音} & \textbf{軟口蓋音} \\
        \midrule
        \textbf{鼻音} & \textipa{/m/} & \textipa{/n/} & \\
        \textbf{破裂音} & \textipa{/p/} & \textipa{/t/} & \textipa{/k/} \\
        \textbf{摩擦音} & \textipa{/B/} & & \\
        \textbf{流音} & & \textipa{/R/} & \\
      
  \bottomrule
    \end{tabular}
    \caption{\centering \langname の子音体系}
    \label{tab:consonants}
\end{table}

以下に、各子音の音素的な特徴と、単語中の出現例を挙げる。括弧内には音素の異音を示した。

\begin{itemize}
    \item \textbf{/m/}: 両唇鼻音。
    \begin{itemize}
        \item 例: mua [mua] (動物)
    \end{itemize}

    \item \textbf{/n/}: 歯茎鼻音。
    \begin{itemize}
        \item 例: ni [ni] (手で)
    \end{itemize}

    \item \textbf{/p/}: 無声両唇破裂音。
    \begin{itemize}
        \item 語頭では有気音[\textipa{p\super{h}}]として発音される。
        \item 例:  paniai \textipa{[p\super{h}aniai]} (家)
    \end{itemize}
% TODO : /p/, /t/, /k/ の有気音化は「語頭」という環境限定ですが、これは「強勢のある音節の語頭」や「母音間を除く語頭」など、より厳密な環境限定ではないか確認が必要です。例えば、語中で子音が連続した場合の振る舞いも確認してください。

    \item \textbf{/t/}: 無声歯茎破裂音。
    \begin{itemize}
        \item 語頭では有気音[\textipa{t\super{h}}]として発音される。
        \item 例: tue \textipa{[t\super{h}ue]} (船)
        \item 母音/i/, /e/の前後では後部歯茎破擦音[\textipa{tS}]として発音される。
        \item 例: tia \textipa{[tS\super{h}ia]} (鍋)
    \end{itemize}

    \item \textbf{/k/}: 無声軟口蓋破裂音。
    \begin{itemize}
        \item 語頭では有気音[\textipa{k\super{h}}]として発音される。
        \item 例: ka \textipa{[k\super{h}a]} (彼)
    \end{itemize}

    
\item \textbf{\textipa{/B/}}: 有声両唇摩擦音。
    \begin{itemize}
        \item 例: vavai \textipa{[BaBai]} (道)
    \end{itemize}
    
    \item \textbf{\textipa{/R/}}: 歯茎たたき音。
    \begin{itemize}
        \item 自由異音としてふるえ音[\textipa{r}]になることがある。
        \item 例: rari \textipa{[RaRi]} (唇)
    \end{itemize}
% TODO : 歯茎たたき音 /R/ のふるえ音 [\textipa{r}] への自由異音の記述は興味深いです。音素 /R/ の音素的な位置づけ(例:他の流音との関係)を明確にし、異音の発生傾向(例:話者の年齢や方言差など)があれば追記してください。
\end{itemize} 
\subsection{母音}\langname の母音は表\ref{tab:vowels}に示された5種類である。

\begin{table}[H]
    \centering
    \begin{tabular}{lccc}
        \toprule
        & \textbf{前舌} & \textbf{中舌} & \textbf{後舌} \\
        \midrule
        \textbf{狭母音} & \textipa{/i/} & & \textipa{/u/} \\
        \textbf{中母音} & \textipa{/e/} & & \textipa{/o/} \\
        \textbf{広母音} & & \textipa{/a/} & \\
        \bottomrule
    \end{tabular}
    \caption{\centering \langname の母音体系}
    \label{tab:vowels}
\end{table}

以下に、各母音の音素的な特徴と、単語中の出現例を挙げる。

% \textipa{i} \textipa{[i]} (彼) 
\begin{itemize}
    \item \textbf{\textipa{/i/}}: 前舌狭母音。
    \begin{itemize}
        \item 例: uairi \textipa{[uwaiRi]} (弓矢)
    \end{itemize}

    \item \textbf{\textipa{/u/}}: 後舌狭母音。
    \begin{itemize}
        \item 例: % TODO : 少なくとも一つの具体的な単語例(音素表記と音声表記)を挙げてください。
    \end{itemize}
% TODO : /u/ が唇音(/m/, /p/, /B/)の前後や、軟口蓋音(/k/)の前後で円唇性が強くなったり(例:[\textipa{u}]→[\textipa{u}])、音素的な変化がないか確認してください。

    \item \textbf{\textipa{/e/}}: 前舌半狭母音。
    \begin{itemize}
        \item 例: % TODO : 少なくとも一つの具体的な単語例(音素表記と音声表記)を挙げてください。
    \end{itemize}
% TODO : 母音/e/がストレス(強勢)のない音節で中央化(例:[\textipa{@}])しないか確認してください。

    \item \textbf{\textipa{/o/}}: 後舌半狭母音。
    \begin{itemize}
        \item 
例: % TODO : 少なくとも一つの具体的な単語例(音素表記と音声表記)を挙げてください。
    \end{itemize}
% TODO : 母音/o/が狭母音(/u/)や広母音(/a/)との対立において、明確な音韻的地位を持つことを確認し、最小対(Minimal Pair)があれば言及してください。

    \item \textbf{\textipa{/a/}}: 中舌広母音。
    \begin{itemize}
        \item 例: % TODO : 少なくとも一つの具体的な単語例(音素表記と音声表記)を挙げてください。
    \end{itemize}
% TODO : /a/ が前舌子音(/t/, /n/)と後舌子音(/k/)の前後で舌の位置が変化する異音(例:前舌化[\textipa{a}]や後舌化[\textipa{A}])を持たないか確認し、記述してください。
\end{itemize} 
\subsection{音節}% 基本的な制約の提示(主要命題)
\langname において、開音節のみが許可され、子音クラスターや音節末子音は観察されない。
また、二重母音も存在しない。
% 許可される音節構造の定義
上記の制約の結果、\langname において許可される音節構造は、以下の二つのタイプに限定される。

\begin{itemize}
    \item {V (母音のみ)}: 音節核 (Nucleus) のみからなる音節。
    \item {CV (子音+母音)}: 頭子音 (Onset) と音節核からなる音節。
\end{itemize}

% Coda/子音クラスター回避のメカニズム(介在母音挿入)
\langname において音節構造の厳格さが最も明確に示されるのは、形態素が接合する境界である。
% TODO : 指摘点: どの「介在母音」が挿入されるのか(例: /i/ や中立母音など)を明記することが、記述としては必須です。
ある語根が子音で終わる場合、それに続く形態素が子音で始まることは音韻的に許容されず、必ず介在母音
が挿入される。
これにより、潜在的に発生し得る音節末子音や子音クラスターが回避される。

\begin{exe}
    \ex \glll v-api \\
    v-pi \\
    eat-IMPF \\
        \glt 食べる
\end{exe}

% 母音連続の発生と制約(構造の帰結)
% TODO : 指摘点: 見出しに「構造の帰結」とありますが、この記述だけでは現象を指していません。「許容される母音連続と制約」など、現象を具体的に示す見出しが望ましいです。
% TODO : 指摘点: 母音連続の発生箇所(音節間または形態素間)を明記し、「母音連続が発生する」を「母音連続が観察される」などの客観的な表現に修正します。
{V} と {V}、および {CV} と {V} の組み合わせによって生じる音節間において、母音連続が発生する。
% TODO : 指摘点: ここで述べられている「o+u、u+oを除く」という制約は、なぜその二つだけが許容されないのかという**音韻論的動機付け**(例: 音高、円唇性など)を、可能であれば示唆するべきです。
o+u、u+oを除く任意の母音連続が許容される。

以下は、{V} と {V}、および {CV} と {V} の組み合わせによって生じる
許容される母音連続の例である。

\begin{itemize}
    \item u.a.i : 鼻 % TODO : 指摘点: 音節境界を明示する「.」の記法は適切ですが、国際音声記号(IPA)表記を併記するか、音韻表記であることを明記するのが一般的です。
    \item no.a.ke : 音
\end{itemize}

% 母音省略による制約の維持(形態音韻規則)
% TODO : 指摘点: 見出しを「母音省略(エリジオン)」や「形態素境界における母音省略規則」など、より具体的にし、単なる制約の「維持」ではなく**規則そのもの**を強調すべきです。
形態素間において上記の2母音が連続した場合、前の母音が省略される。
% TODO : 指摘点: 「上記の2母音」は、どの2母音を指すのかが不明確です。直前の「母音連続」を指すのであれば、「2つの母音」や「母音連続において」などと明確に述べるべきです。
\begin{exe}
    \ex \glll v-api \\
    v-pi \\
    eat-IMPF \\
   \glt 食べる
\end{exe}
% TODO : 指摘点: ここに具体的な例文を補足することで、この音韻規則がどのように適用されるかを示すことが記述言語学の論文では必須となります。
\newpage

\section{名詞}名詞は、格と数によって活用する。その活用形の違いから、名詞は以下の4つのサブクラスに分類される。
% TODO : 活用
% TODO : 活用形の違い→取りうる格と数?
% TODO : サブクラス
本節では各サブクラスの基本的な活用特性を概観し、具体的な活用形については次節以降で詳細に記述する。
% TODO : 活用特性
ここで、◯は活用形が存在し、その文脈で用いられることを示し、×は該当する活用形が存在せず、
その文脈では使用されないことを示す。

\begin{itemize}
    \item \textbf{一般名詞}: 人や物などの独立した実体を表す名詞。対格のみに活用形を持つ。
    \item \textbf{部位名詞}: 人や物の一部分を表す名詞。水や空気、砂などの物質を表す名詞も含まれる。
    対格と具格に活用形を持つ。
    \item \textbf{代名詞}: 人称代名詞など。対格と与格に活用形を持つ。
    \item \textbf{類別詞}: 名詞を分類する接辞または独立した語。すべての格と数に活用形を持つ。
\end{itemize}

名詞のサブクラスごとの活用形は、以下の表\ref{tab:noun_classes}に示される。

\begin{table}[H]
\centering
\begin{tabular}{|l||c|c|c|c|c|}
\hline
\textbf{} & \textbf{対格} & \textbf{具格} & \textbf{与格} & \textbf{所有} & \textbf{複数} \\
\hline
\hline
一般名詞 & ◯ & × & × & × & × \\
\hline
部位名詞 & ◯ & ◯ & × & × & × \\
\hline
代名詞 & ◯ & × & ◯ & × & × \\
\hline
類別詞 & ◯ & ◯ & ◯ & ◯ & ◯ \\
\hline
\end{tabular}
\caption{\centering 名詞クラスごとの活用特性}
\label{tab:noun_classes}
\end{table}

\vspace{1em}

次節では、これらの各サブクラスの活用形について、より具体的な語例を挙げて解説する。 
\subsection{一般名詞}人や物など、独立した実体を表す名詞である。

例
\begin{itemize}
\item \textipa{viri} (猫)
\item \textipa{tatuai} (木の一種)
\end{itemize}

一般名詞は数の区別をしない。
\begin{exe}
    \ex \gll ka-∅ pipoa runa-pi \\
        3.SG sago cut-IMPF \\
        \glt 彼はサゴを切る。
\end{exe}
\begin{exe}
    \ex \gll *ka-∅ pipoa-ia runa-pi \\
        3.SG sago-PL cut-IMPF \\
        \glt 彼はサゴを何本か切る。
\end{exe}
\begin{exe}
    \ex \gll ka-∅ pipoa upa-ia runa-pi \\
        3.SG sago CLF.plant-PL cut-IMPF \\
        \glt 彼はサゴを何本か切る。
\end{exe}
% TODO : 例文改良

一般名詞は格接辞を伴わない。
\begin{exe}
    \ex \gll ka-∅ pipoa voa-pi \\
        3.SG sago eat-IMPF \\
        \glt 彼はサゴを食べる。
\end{exe}
\begin{exe}
    \ex \gll *ka-∅ pipoa-na voa-pi \\
        3.SG sago-ACC eat-IMPF \\
        \glt 彼はサゴを食べる。
\end{exe}
% TODO : 格接辞の記述を強化する。
% 1. 例文 (1) のグロスラインで、一般名詞 *pipoa* の後に **sago** ではなく **sago-ABS** や **sago-NOM** のように、それが基底形またはゼロ格(絶対格や主格など)であることを示すグロスを付与することを検討してください。これにより、格が存在しないのではなく、ゼロ形態で現れることが明確になります。
% 2. 例文 (2) の非文法性を示す格接辞 *-na* のソース(例:他の名詞クラスから借用されたもの、またはフィールドワークで確認された誤用など)を簡潔に説明してください。
% 3. **目的語**としてだけでなく、**主語**、**間接目的語**、**道具格**、**場所格**などの文法機能においても格接辞を伴わないこと、および、それらの機能が語順や前置詞(または後置詞)によって担われていることを明確に記述してください。 
\subsection{部位名詞}水や空気、砂といった、個体として認識されにくい物質を表す名詞。
手や足などの、人体の部位や物体の部分を表す名詞も含まれる。。

例
\begin{itemize}
\item \textipa{tue} (手)
\item \textipa{uete} (水)
\end{itemize}

% TODO : 数への言及

部位名詞は対格と具格の両方に活用形を持つ。
% 活用形を持つ

\begin{exe}
    \ex \gll aba-k ra kiko-i pi-ak \\
        3-CLF.human TOP sago-ACC set-IMPF \\
        \glt 彼はサゴを食べる。
\end{exe}
% TODO : 例文
\subsection{代名詞}\langname の代名詞システムは、代名詞幹 (Pronoun Stem)と類別詞 (Classifier) 
からなる複合体である。
代名詞幹は単独では機能せず、類別詞が接辞化することで初めて完全な代名詞句を形成する。

この構造において、代名詞が持つ格(対格など)や数(単数・複数)の情報は、付属する類別詞に
よって排他的に示される。
代名詞句の内部構造における語順は、「代名詞-類別詞」である。

以下の例文は、代名詞幹\textipa{Fe}が、類別詞により複数形と対格標識を担うことを示す。

\begin{exe}
    \ex \gll pa kui, vuae-i iat pa-i nea-u-i \\
        MOUNT sun 1.PL-ACC eat CLF.powder-INS give-PRG-IMP \\
        \glt 私たちの日ごとの糧を、今日もお与えください。
\end{exe}
% TODO : ここ以上の内容を統語論に移動?

\paragraph{人称代名詞 (Personal Pronoun Stems)}
人称代名詞幹は、原則として{有生性の高い対象(人間}の指示に用いられる。
% TODO : 代名詞の有生性による対比は必要?
3人称代名詞を含め、代名詞は性差による区別を持たない。

\begin{table}[H]
    \centering
    \begin{tabular}{lcc}
        \toprule
        & 単数 & 複数  \\
        \midrule
        一人称 & ria & riai \\
        二人称 & vua & vuai \\
        三人称 & ka & kai \\
        \bottomrule
    \end{tabular}
    \caption{\centering \langname の人称代名詞}
    \label{tab:pronouns}
\end{table}

\paragraph{指示代名詞 (Demonstrative Pronoun Stems)}
本言語の指示代名詞は、{山と海という地理的ランドマークを基準とした絶対座標系
 (Absolute Frame of Reference) に基づいて分類される}という際立った特徴を持つ。
% TODO : 地理的ランドマーク?
% TODO : 近接性との関係
% TODO : 図を用いた説明もあり

\begin{itemize}
    \item \textbf{\textipa{pa}}: {山側}の方向にあるものを指す。(基準点からの絶対方向)
        \begin{itemize}
            \item \texttt{ka-tet}
            \quad 「山方向にある動物」(類別詞 \texttt{-tet} は「動物」クラス)
        \end{itemize}
    \item \textbf{\textipa{at}}: {海側}の方向にあるものを指す。(基準点からの絶対方向)
        \begin{itemize}
            \item \texttt{b1-tet}
            \quad 「海方向にある動物」
        \end{itemize}
    \item \textbf{\textipa{ri}}: 山と海に{垂直な方向}(例:川に沿った方向)にあるものを
    指す{絶対方向形式}を基本とする。
    % TODO : 川に沿った方向ではない
    ただし、話者と聞き手に近いもの(近接性/Proximate)を指す際にもこの形式が流用されることがある。
        \begin{itemize}
            \item \texttt{fiwi-di}
            \quad 「(垂直方向にある、あるいは)近くにある丸いもの」(類別詞 \texttt{-di} は「丸いもの」クラス)
        \end{itemize}
\end{itemize}
% TODO : 表と詳細な説明
% TODO : 例文を追加
\subsection{類別詞}\langname の名詞は、指示詞や数詞と共起する際に、必ず特定の類別詞(classifier)を伴う。
以下のリストは、本言語の主要な類別詞をまとめたものであるが、完全なものではない。
% TODO : 完全なリストを構築

% TODO : 類別詞を接辞から独立した語にする。
\begin{itemize}
    \item \textbf{動物}\\
    人間や動物を表す類別詞は最も数が多い。
    % 具体的な割合
        \begin{itemize}
            \item \textbf{api}: 人間を指す。
                \begin{itemize}
                    \item \texttt{uevo api} (子供 CLF.人間)
                    \quad 「一人の子供」
                \end{itemize}
            \item \textbf{mu}: 四足歩行の動物を指す。
                \begin{itemize}
                    \item \texttt{tui mu} (犬 CLF.動物)
                    \quad 「犬」
                \end{itemize}
            \item \textbf{-i'k}: 鳥、爬虫類などを指す。
                \begin{itemize}
                    \item \texttt{kawa-i'k} (カエル-CLF.鳥) % TODO
                    \quad 「一羽のカエル」
                \end{itemize}
            \item \textbf{-wobir}: 魚を指す。
                \begin{itemize}
                    \item \texttt{q'eb-wobir} (魚-CLF.魚) % TODO
                    \quad 「一匹の魚」
                \end{itemize}
        \end{itemize}
    \item \textbf{植物} % TODO : 生物としてまとめる?
        \begin{itemize}
            \item \textbf{-i'k}: 樹木を指す。鳥、爬虫類などを指す類別詞と同型。
                \begin{itemize}
                    \item \texttt{paup-i'k} (木-CLF.樹木) % TODO
                    \quad 「一本の木」
                \end{itemize}
            \item \textbf{-ab}: 草を指す。
                \begin{itemize}
                    \item \texttt{aba-ab} (草-CLF.草) % TODO
                    \quad 「一束の草」
                \end{itemize}
            \item \textbf{t'ed}: 食べられる草を指す。% TODO
                \begin{itemize}
                    \item \texttt{t'ed-t'ed} (草-CLF.食べられる草) % TODO
                    \quad 「一株の食べられる草」
                \end{itemize}
        \end{itemize}
    \item \textbf{名詞の形状・性質}
        \begin{itemize}
            \item \textbf{di}: 丸いものを指す。
                \begin{itemize}
                    \item \texttt{koko-di} (石-CLF.丸いもの) % TODO
                    \quad 「一つの石」
                \end{itemize}
            \item \textbf{re}: 細長いものや道具を指す。
                \begin{itemize}
                    \item \texttt{w'ir-re} (槍-CLF.細長いもの) % TODO
                    \quad 「一本の槍」
                \end{itemize}
            \item \textbf{ii}: 束や集団を指す。
                \begin{itemize}
                    \item \texttt{ii-ii} (群れ-CLF.集団) % TODO
                    \quad 「一つの群れ」
                \end{itemize}
            \item \textbf{kiq}: 薄いものや曲がっているものを指す。
                \begin{itemize}
                    \item \texttt{yab-kiq} (葉-CLF.薄いもの) % TODO
                    \quad 「一枚の葉」
                \end{itemize}
            \item \textbf{kaw}: 粉を指す。
                \begin{itemize}
                    \item \texttt{pob-kaw} (土-CLF.粉) % TODO
                    \quad 「一握りの土」
                \end{itemize}
            \item \textbf{q'eb}: 水を指す。% TODO : 消す?
                \begin{itemize}
                    \item \texttt{q'eb-q'eb} (水-CLF.水)
                    \quad 「一杯の水」
                \end{itemize}
            \item \textbf{k'ik}: 火を指す。% TODO : 消す?
                \begin{itemize}
                    \item \texttt{k'ik-k'ik} (火-CLF.火)
                    \quad 「一つの炎」
                \end{itemize}
        \end{itemize}
\end{itemize}
これらの類別詞は、名詞句において数詞や指示詞の直後に付加される接尾辞として機能する。
% TODO : 接尾辞→独立した語

以下の節でその機能を列挙する。
% TODO : 以下の内容は統語論に移動

\paragraph{一般名詞の代替}
類別詞は、特定の範疇全体を指す名詞として機能する。
\begin{exe}
\ex \gll tet-a pap ra si pi-fi-qr-ak\\
        CLF.animal-GEN fat TOP by hand eat-1-NEG-PRES\\
\glt 私は動物の脂は食べない。
\end{exe}

\paragraph{指示性の付与}

\langname の名詞は、単独で用いられる場合、特定の個体ではなく、種類全体や不特定の対象を指す。
一方、類別詞が名詞に後続した場合、その名詞は話し手と聞き手の間で共有される特定の対象を指し示す。
ここで、類別詞は名詞句に特定性を付与する役割を担っている。
類別詞が常に名詞の直後に置かれる。
\begin{exe}
\ex \gll d\'ii t\'eow-i pi-fe-er\\
pig talk say-1-PL.PAST \\
\glt 私たちは豚の話をしていた。
\end{exe}
\begin{exe}
\ex \gll d\'ii et t\'eow-i pi-fe-er\\
pig CLF.animal talk say-1-PL.PAST \\
\glt 私たちはとある豚の話をしていた。
\end{exe}

\paragraph{数量詞・指示詞との併用}

数詞や指示詞が名詞に前置される場合、類別詞は名詞句の必須要素としてその名詞に後続する。
このとき、語順は指示詞-数詞-名詞-類別詞の順となり、ここでも類別詞は常に名詞の直後に置かれる。
\begin{exe}
\ex \gll pawe bwe et\\
二 犬 CLF.動物\\
\glt 犬二匹
\end{exe}
\begin{exe}
\ex \gll aba bwe et\\
これ 犬 CLF.動物\\
\glt この犬
\end{exe}


\paragraph{名詞の分類}

\langname の類別詞は、名詞が属する\textbf{意味的範疇}を決定する役割を持つ。
同じ名詞でも、後続する類別詞の違いにより、その指示対象の分類が変化する。
\begin{exe}
\ex \gll bwe et\\
犬 CLF.動物\\
\glt 犬(動物としての犬)
\end{exe}
\begin{exe}
\ex \gll bwa ak\\
犬 CLF.人間\\
\glt 犬(人間を指す際の比喩的な呼び方)
\end{exe}
\subsection{数詞}\paragraph{基数詞}
\langname の基数詞は以下の通りである。
\begin{itemize}
    \item 1: teq
    \item 2: pawe
    \item 3: a'ir
    \item 4: kaw
    \item 5: pi'i
\end{itemize}

これらの数詞は、名詞、そしてそれに続く類別詞とともに使用される。
\begin{itemize}
    \item \textit{teq-ak} 「一人」 (teq-CLF.人間)
    \item \textit{pawe bwe-et} 「二匹の犬」 (pawe bwe-CLF.動物)
\end{itemize}

\paragraph{用法}
数詞が名詞句に含まれる場合、類別詞に複数接尾辞(-'ey)は付加されない。これは、数詞そのものが数量を明確に示しているためである。
\begin{verbatim}
pawe bwe-et
二 犬-CLF.動物
「二匹の犬」
\end{verbatim}
% 以下セルフレビュー
\subsection{名詞の活用}名詞は、格と数に応じた形態変化を示す。
活用は下位分類ごとに異なるパターンをとり、以下にその詳細を記述する。

\langname の名詞の格は、主に語尾に付加される\textbf{接尾辞}によって示される。
一つの接尾辞が複数の意味的・統語的機能を持つことが特徴である。
また、場所や方向を表す格は、特定の意味を含意する\textbf{助詞}によっても表現される。

\paragraph{接尾辞による格表示}

\langname の主要な格は、名詞の語尾に付く接尾辞によって区別される。

\begin{itemize}
    \item \textbf{-i / -∅}: この接尾辞は、以下に示す複数の格機能を持つ。
    \begin{enumerate}
    \item \textbf{対格}: 他動詞の直接目的語を示す。この機能は、名詞句が動詞句の先頭に置かれることによって示される。この語順の規則性により、対格の名詞句では接尾辞-iがしばしば\textbf{ゼロ形態素-0}として省略される。
    \begin{exe}
        \ex \gll Wobir-i peak. \\
        魚-ACC 食べる \\
        \glt 魚を食べる。
    \end{exe}
    \begin{exe}
        \ex \gll Wobir-0 peak. \\
        魚-ACC 食べる \\
        \glt 魚を食べる。
    \end{exe}

    \item \textbf{具格}: 動作の手段や道具を示す。\textbf{主格以外の複数の項を持つ他動詞句において、対格の名詞句に続く位置に置かれた場合}に具格を表す。
    \begin{exe}
        \ex \gll akak bwi-i q'es-ek. \\
        he stick-INS hit-PAST \\
        \glt 彼は棒で殴った。
    \end{exe}

    \begin{exe}
        \ex \gll akak bwi-i t'aq-0-i q'es-ek. \\
    彼 棒-INS 魚-ACC 殴る-PAST \\
        \glt 彼は魚を棒で殴った。
    \end{exe}

    \item \textbf{与格}: 間接目的語を示す。基本的に人間や動物といった有生性のある名詞が入る。\textbf{'si' (与える)や'kawisi' (餌をやる)といった贈与の意を表す動詞では、対格の名詞句に続く位置に置かれた場合}に与格を表す。
    \begin{exe}
        \ex \gll Akak-i riki-i-ak. \\
    彼-DAT 贈り物-渡す-PRES \\
        \glt 彼に贈り物を渡す。
    \end{exe}

    \begin{exe}
        \ex \gll Akak-i ab-i si-ek. \\
    彼-DAT 食物-ACC 与える-PAST \\
        \glt 彼は彼に食べ物を与えた。
    \end{exe}
\end{enumerate}

    \item \textbf{-a}: この接尾辞は、以下の格機能を持つ。
    \begin{itemize}
        \item \textbf{自動詞の主格}: 自動詞の主語を示す。
        \begin{exe}
        \ex \gll Akak-a biribiriak.\\
            彼-NOM 泳ぐ\\
        \glt 彼が泳ぐ。
        \end{exe}
        \item \textbf{属格}: 所有や帰属の関係を示す。
        \begin{exe}
        \ex \gll akak-a koi.\\
            彼-GEN 家\\
        \glt 彼の家。
        \end{exe}
    \end{itemize}

    \item \textbf{-oit}: この接尾辞は、場所を示す。
    \begin{itemize}
        \item \textbf{所格}: 動作が行われる場所を示す。
        \begin{exe}
        \ex \gll koi-oit biribiriak.\\
            家-LOC 泳ぐ\\
        \glt 家で泳ぐ。
        \end{exe}
    \end{itemize}
\end{itemize}

格接尾辞はしばしば省略されるが、具格、与格の\textbf{-i}、および所格の\textbf{-oit}において特に顕著である。
これらの省略された格は、文脈、または語順によってその役割が示される。

\paragraph{助詞による格表示}

場所や方向を表す格は、接尾辞に加えて、特定の意味的含意を持つ\textbf{助詞}によっても表現される。

\begin{itemize}
    \item \textbf{qew}: \textbf{向格}を示す。「~に向かって」の意。
    向かう方向が山と海の方向に対して垂直であることを含意する。
    \begin{exe}
    \ex \gll akak qew tobiak.\\
        彼 垂直.向格 行く\\
    \glt 彼は(川を渡って)行く。
    \end{exe}

    \item \textbf{qe}: \textbf{向格}を示す。「~に向かって」の意。
    向かう方向が山の方向であることを含意する。
    \begin{exe}
    \ex \gll akak qe tobiak.\\
        彼 山.向格 行く\\
    \glt 彼は(山の方向に)行く。
    \end{exe}
    
    \item \textbf{ri}: \textbf{向格}を示す。「~に向かって」の意。
    向かう方向が海の方向であることを含意する。
    \begin{exe}
    \ex \gll akak ri tobiak.\\
        彼 海.向格 行く\\
    \glt 彼は(海の方向に)行く。
    \end{exe}

    \item \textbf{dusi}: \textbf{通過格}を示す。「~を超えて」の意。
    対象を避けて向かうことを含意する。
    \begin{exe}
    \ex \gll akak dusi ki tobiak.\\
        彼 避けて.通過格 岩 行く\\
    \glt 彼は(岩を避けて)行く。
    \end{exe}
    
    \item \textbf{ibi}: \textbf{通過格}を示す。「~を通って」の意。
    対象を通過して向かうことを含意する。
    \begin{exe}
    \ex \gll akak ibi abi tobiak.\\
        彼 通過.通過格 森 行く\\
    \glt 彼は(森を通って)行く。
    \end{exe}
\end{itemize}

\paragraph{数活用}

\langname の名詞は、単独で用いられる場合、特定の個体ではなく、種類全体や不特定の対象を指すため、
数を表示しない。しかし、類別詞を伴うことで単数・複数を区別する。

数は、類別詞に付加される接尾辞によって表現される。

\begin{itemize}
    \item \textbf{単数}: \texttt{-∅}
        \begin{itemize}
            \item 単数形は、特に接尾辞を付加せず、ゼロ形態素(zero morpheme)によって示される。
        \end{itemize}
    \item \textbf{複数}: \texttt{-'ey}
        \begin{itemize}
            \item 複数接尾辞の使用は任意である。
            \item 数詞が名詞句に含まれる場合、複数接尾辞は付加されない。
        \end{itemize}
\end{itemize}

これらの接尾辞は、類別詞の直後に付加される。

\begin{exe}
\ex
\gll paup tet-0 \\
木 CLF.動物-SG \\
\glt 一本の木

\ex
\gll paup tet-'ey \\
木 CLF.動物-PL \\
\glt 複数の木

\ex
\gll pawe bwe et \\
二 CLF.動物 \\
\glt 犬二匹

\ex
\gll *pawe bwe et-'ey \\
二 CLF.動物-PL \\
\glt 犬二匹 (非文法的)
\end{exe} % :TODO セルフレビュー
\newpage

\section{形容詞}\begin{itemize}
    \item 形容詞は、格、数、人称などによって\textbf{形態論的な変化をしない}。
    \item 形容詞が修飾する名詞の格や数は、その名詞自体に付加される形態素によって示される。
\end{itemize}

\langname の\textbf{形容詞}は、修飾する名詞に\textbf{前置される}という厳格な統語的位置を占める。

\begin{itemize}
    \item \textbf{非屈折性 (Non-inflectional)}: 形容詞は、格、数、人称といった文法的範疇によって
    \textbf{形態論的な変化をしない}。
    これは、形容詞語幹にいかなる屈折形態素も付加されないことを意味する。
\end{itemize}

この形容詞の非屈折性は、修飾される名詞との\textbf{一致 (Agreement)} を示さない
という重要な形態統語的結果をもたらす。

\begin{itemize}
    \item \textbf{文法範疇の担い手}: 形容詞が修飾する名詞の格や数などの情報は、
    形容詞自体ではなく、\textbf{その名詞自体に付加される形態素によって示される}。
\end{itemize}

この構造は、\langname の文法システムにおける形態論的役割の明確な分担を示唆する。
すなわち、形容詞は専ら\textbf{意味的機能}(属性の付加)を担うのに対し、名詞は意味的機能に加え、
統語的機能(格、数)を担う屈折形態素を伴う。
これは、\langname が\textbf{名詞に焦点を当てた格と数の標示システム}を持つことを示唆している。 % :TODO セルフレビュー
\newpage

\section{名詞句}
% TODO : 相対的な人称
\newpage

\section{動詞}本セクションでは、\langname の動詞の形態論を記述する。
% TODO : 指摘点: 「複雑な形態変化」という表現はやや曖昧です。この複雑さが具体的にどのようなタイプか(例: 膠着性(agglutinative)が強い、融合的(fusional)である、異形態(allomorphy)が多いなど)を暗示すると、より記述的になります。
\langname の動詞は、時制、相、法、態といった文法範疇に応じた複雑な形態変化を示す。
% TODO : 指摘点: 動詞の定義(例: 屈折によって識別される品詞、語幹の種類など)を導入部に簡潔に含めると、分析の基盤が明確になります。
これらの変化は、主に接尾辞によって実現される。
% TODO : 指摘点: 形態論の分析の前提として、動詞の**語幹(Stem)**や**語根(Root)**と**接辞(Affix)**の境界の定義をどこかで明確に述べることが、この概観で示唆されていても良いでしょう。
本節では、まず動詞の形態論的特徴を概観し、次節以降で各範疇の具体的な活用規則と語例について詳細に
記述する。
\subsection{相}\langname の動詞は、語幹に接尾辞を付加することで、時制(出来事の時間的な位置)と相(動作の開始、継続、完了)を同時に標示する。以下の接尾辞は、動詞語幹に直接付加される。

\begin{itemize}
\item \textbf{-ak}: \textbf{不定相}(Imperfective Aspect)\
特定の時制や相を強調しない、最も一般的な動詞形である。現在の出来事や、習慣的な行為を記述する際に用いられる。
\begin{itemize}
\item 例: ``tobi\textbf{ak}'' /tobiak/ (行く)
\end{itemize}

\item \textbf{-qi}: \textbf{開始相}(Inchoative Aspect)\\
動作の開始を表す。
\begin{itemize}
    \item 例: ``tobi\textbf{qi}'' /tobiqi/ (行き始める)
\end{itemize}

\item \textbf{-ak'e}: \textbf{進行相}(Progressive Aspect)\\
動作が進行中であることを示す。
\begin{itemize}
    \item 例: ``tobi\textbf{ak'e}'' /tobiak'e/ (行っている最中である)
\end{itemize}

\item \textbf{-ru}: \textbf{完了相}(Perfective Aspect)\\
動作が完了したことを示す。完了した過去の出来事を記述する際に用いられる。
\begin{itemize}
    \item 例: ``tobi\textbf{ru}'' /tobiru/ (行き終わる)
\end{itemize}

\item \textbf{-ako}: \textbf{経験相}(Experiential Aspect)\\
過去に一度でもその動作を経験したことがあることを示す。
\begin{itemize}
    \item 例: ``tobi\textbf{ako}'' /tobiako/ (行ったことがある)
\end{itemize}

\end{itemize} % :TODO セルフレビュー
\subsection{否定}動詞の意味内容の否定は、相接辞の後ろに接辞 "-ue" を挿入することで示される。

以下の例は、動詞 "tivoa" 「行く」が否定接辞 "-ue" を伴うことで意味が否定される。
ことを示す。

\begin{exe}
\ex \gll tivoa-pi-ue \\
        go-NEG-ASP \\
        \glt `行かない'
\end{exe}
%
例文において、相接辞 "-pi" の後ろに否定接辞 "-ue" が挿入され、
「行かない」という否定の意味が構成されている。 % :TODO セルフレビュー
\subsection{命令}動詞に対して\textbf{命令法 (Imperative)} の意味を付与する場合、接辞 \textbf{-wi} を用いる。
この命令接辞は、\textbf{相接辞(Aspect Marker)}の直後に後置される。

命令形を構成する際、動詞語幹に付加される相接辞は、例外なく \textbf{進行相 (Progressive Aspect)} 
を示す形態 \textbf{-ak'e} を取ることが義務付けられている。他の相形態との共起は許可されない。

この構造は以下のスキーマで表される。
%
\begin{enumerate}
    \item $[ \text{V}_{\text{root}} ] - \textbf{ak'e}_{\text{PROG}} - \textbf{wi}_{\text{IMP}} - [ \text{Agreement/Other Suffixes} ]$
\end{enumerate}
%
接辞 \textit{-wi} は、動詞の表す動作や状態を話者が聞き手に実行するよう促す\textbf{命令接辞}として
機能する。

命令法の表現において、接辞 \textbf{-wi} は\textbf{文体 (Style)} や
\textbf{レジスター (Register)} に応じた\textbf{異形態的変異}を示す。
特に、\textbf{乱暴な表現}や\textbf{親密で非公式な状況}において、
命令接辞 \textit{-wi} は省略されることがある。
この場合、命令の意味は進行相接辞 \textit{-ak'e} のみで担われ、文脈によって命令法として解釈される。

以下の例は、動詞 \textit{qipu} 「行く」が進行相接辞 \textit{-ak'e} と命令接辞 \textit{-wi} を
伴うことで、命令の意味「行け」を構成することを示す。

\begin{exe}
\ex \gll qipu-ak'e-wi \\
        go-PROG-IMP \\
        \glt `行け' \hfill (\textit{標準的/中立的命令})
\end{exe}

\begin{exe}
\ex \gll qipu-ak'e \\
        go-PROG \\
        \glt `行け' \hfill (\textit{乱暴な/非公式な命令})
\end{exe}
%
例文 (\ref{exe}: ex2) は、\textit{-wi} が省略された形式であり、
話し手が聞き手に対してより直接的、あるいは乱暴な口調を用いていることを示す。
この省略された形態は、\textit{-wi} が必須ではなく、社会言語的な機能を持つことを示唆する。 % :TODO セルフレビュー
\subsection{他動詞化}\langname では、動詞の\textbf{他動詞化(使役化}を形態論的に標示する。
これは、使役者(causer)を伴う動詞句を生成する手段であり、自動詞的あるいは非使役的な動詞語幹に
適用され、動詞句に「〜させる」という使役の意味を加える。

使役化は、動詞の\textbf{語幹 (Stem)} と\textbf{相接辞 (Aspectual Suffix)} の間に
\textbf{使役接辞 \texttt{-se-}}を挿入することで実現される。

$$ \text{動詞語幹} + \texttt{-se-} + \text{相接辞} $$

以下の例文は、自動詞的な動詞語幹 (\texttt{run}) が使役接辞 \texttt{-se-} によって他動詞化し、
主語が動作主(Agent)から使役者(Causer)へと変化することを示す。

\begin{exe}
        \ex \gll kawa-k-u \textipa{PRi-ak}\\
        woman-CLF.human-NOM die-IMPR\\
        \glt 彼女は死んだ。
\end{exe}
\begin{exe}
        \ex \gll \textipa{Fiwu} kawa-k \textipa{k1k-i} se-ak, \textipa{PRi-se-ak} \\
        これ woman-CLF.human fire-INS do-IMPR die-CAUS-IMPR \\
        \glt 彼は彼女を焼き殺した。
\end{exe}

例文(\ref{ex:caus_auto}a)に示す\textbf{非使役形}は、動詞語幹 \textipa{PRi} に不定相接辞
 \textipa{-ak} が直接後続する構造をとる。
 これに対し、例文(\ref{ex:caus_auto}b)の\textbf{使役形}は、動詞語幹 \textipa{PRi} と不定相接辞
  \textipa{-ak} の間に\textbf{使役接辞 \textipa{-SE-} が挿入される}ことによって形成される。

この\texttt{-se-}の挿入により、動詞の\textbf{項構造 (Argument Structure)} が変化し、
動詞は主語(使役者)に加えて、使役の対象(ここでは\texttt{child-ACC})を
\textbf{対格 (Accusative)} の形で取るようになる。 % :TODO セルフレビュー
\newpage

\section{副詞}\langname における副詞の主要な特徴は、その統語的位置と形態的な不変性にある。

% 記述言語学では、統語的位置を明確に定義することが重要です。
\langname の副詞は、主に動詞の直前に配置され、被修飾語(動詞)に先行する。
この位置は、動詞句内での標準的な副詞の配置を示す。

一方で、時間、場所など、より広い文脈に関わる一部の副詞は、文頭(節の最左端)に配置されることが
観察される。
これは、それらの副詞が文全体を修飾し、談話的な機能を持つことを示唆する。

% 語形変化がないことを「形態的な不変性」として表現し、厳密性を高めます。
\langname の副詞は、屈折的な語形変化(活用)を起こさない。
この特徴は、\langname における不変化詞(Particle)または不変化語(Invariable Word)としての
副詞の品詞的な性質を裏付けている。
\newpage

\section{動詞句}
\newpage

\section{文構造}

% TODO : \textbf{}不要