\section{序論}\label{sec:introduction}

本稿は、ニューギニア島近郊の無名の島嶼部の沿岸に位置する複数の村落で、現在、推定
約600人の話者によって話されている\langname の包括的な記述的研究を目的とする。

\langname 話者の大半はモノリンガルであり、当該島嶼部の村落内部には他言語話者が存在しない。
\langname の系統関係は未だ立証されておらず、現在のところ
孤立した言語(language isolate)として扱われている。
歴史的比較研究を試みる上での資料不足から、近隣の主要な語族との明確な繋がりは
見出されていない。
しかし、地理的に近接するオーストロネシア語族に属する諸言語とのわずかな
接触による語彙的・音韻的な影響が断片的に示唆されている。
本論では周辺言語との関係性については示唆するにとどめ、具体的な調査はしない。

本研究は、\langname の構造を記録し、その内在的な法則性を確立することを目的とする。
本研究は以下の3つの主要な部分から構成される。

\begin{enumerate}
    \item \textbf{音韻論}: \langname の分節音および超分節音素の目録を提示し、
    音節構造と音韻規則を詳述する。
    \item \textbf{統語論}: 句構造規則、主要部(head)の配置、および基本的な文型を分析する。
    \item \textbf{形態論}: 語類体系を確立し、品詞と接辞の体系をまとめる。
\end{enumerate}

% TODO セクションの順序の理由
\newpage

\section{音韻}\langname の音韻体系は、以下の表に示すように8つの子音と5つの母音からなる。
これらの音素は、語頭、語中、および母音の前後といった特定の音環境において、
さまざまな異音を伴って現れる。
\subsection{音節}\paragraph{音節構造}\quad\\
\langname の許される音節構造は CV,V の2種類のみである。

% TODO : IPA 現状エラー
\begin{tabular}{llll}
    \multicolumn{2}{l}{例 : } \\ 
    & u\'e & 良い \\
    & kuki & 目 \\
\end{tabular}

子音連続や末子音は存在せず、借用語においてはそれらを回避するために子音後に母音/i/が挿入される。

\begin{tabular}{llll}
    \multicolumn{2}{l}{例 : } \\ 
    & k\'iritimati & 年齢(トクピシン krismas より) \\
\end{tabular}

音節内の母音にはそれぞれ2段階の声調のどちらかが付属する。

\paragraph{語の音節数}\quad\\
\langname の語根は多くが2~3つの音節からなる。

\begin{tabular}{llll}
    \multicolumn{2}{l}{例 : } \\ 
    & vet\'e & 水 \\
    & \'at\'ir\'i & 魚 \\
\end{tabular}

これらの語根に接辞が付加したり、語根どうしが複合したりすることで、語はより多くの音節を含むようになる。

\begin{tabular}{llll}
    \multicolumn{2}{l}{例 : } \\ 
    & vet\'e-vet\'e & 雨 \\
    & \'at\'ir\'i-r\'i & 魚(複数) \\
\end{tabular}
\subsection{音素}\langname は7つの子音音素、7つの母音音素に加え、高低2種類の声調を持つ。
以下の表は\langname の子音音素と母音音素をまとめた表である。

\begin{table}[H]
    \centering
    \begin{tabular}{lccc}
        \toprule
        & \textbf{両唇音} & \textbf{歯茎音} & \textbf{軟口蓋音} \\
        \midrule
        \textbf{鼻音} & \textipa{/m/ [m]} & \textipa{/n/ [n]} & \\
        \textbf{破裂音} & \textipa{/p/ [p]} & \textipa{/t/ [t]} & \textipa{/k/ [k]}  \\
        \textbf{摩擦音} & \textipa{/v/ [B]} & & \\
        \textbf{流音} & & \textipa{/r/ [R]} &  \\
      
  \bottomrule
    \end{tabular}
    \caption{\centering \langname の子音体系}
    \label{tab:consonants}
\end{table}

\begin{table}[H]
    \centering
    \begin{tabular}{lcc}
        \toprule
        & \textbf{前舌}  & \textbf{後舌} \\
        \midrule
        \textbf{狭母音} & \textipa{/i/} & \textipa{/u/} \\
        \textbf{半狭母音} & \textipa{/e/} & \textipa{o} \\
        \textbf{半広母音} & \textipa{/E/} & \textipa{/O/} \\
        \textbf{広母音} & & \textipa{/a/} \\
        \bottomrule
    \end{tabular}
    \caption{\centering \langname の母音体系}
    \label{tab:vowels}
\end{table}
% 提案: 
% これは音素体系の表であるため、/i/ などと同様にスラッシュで囲み、\textipa{/o/} のように**音素表記(スラッシュ)**を使用してください。また、後舌半狭母音は通常 [o] で表されますが、IPAチャートに基づいて適切な音素記号を選択することが重要です。
% 提案: 
% この4段階の分類で正しい場合は、各音素(/e/, /E/, /o/, /O/)が実際にどのIPA記号に対応するか(例: /E/ は \textipa{[openo]} か \textipa{[textepsilon]} か)を本文中で一度明確に示してください。
\subsubsection{子音}\paragraph{子音の発音}
\langname の子音は幅広い異音を持つ。
以下に、各子音の語内で実現される音声について説明する。

\subparagraph{M}
/m/は両唇鼻音\textipa{[m]}として発音される。

\begin{tabular}{llll}
    & \textipa{mipOa [miFOA]} & 芽 \\
\end{tabular}

\subparagraph{P}
/p/は両唇無声破裂音\textipa{[p]}として発音される。
母音間では、/p/は自由異音として摩擦音\textipa{[F]}で発音されることがある。

\begin{tabular}{llll}
    & \textipa{p\'Oia [pOiA]} & 家 \\
    & \textipa{up\'a [upA]~[uFA]} & 鳥 \\
\end{tabular}

\subparagraph{V}
/v/は両唇有声摩擦音\textipa{[B]}または両唇接近音\textipa{[\textlowering{B}]}で発音される。

\begin{tabular}{llll}
    & \textipa{v\'a\'E [BAE]~[\textlowering{B}AE]} & 2 \\
\end{tabular}
% TODO ([B] と [\textlowering{B}])が自由異音か条件異音か

\subparagraph{N}
/n/は歯茎鼻音\textipa{[n]}として発音される。

\begin{tabular}{llll}
    & \textipa{ninE [ninE]} & もらう \\
\end{tabular}

\subparagraph{T}
/t/は歯茎無声破裂音\textipa{[t]}として発音される。
母音/i//ye/の前では破擦音\textipa{[\t{ts}]}~\textipa{[\t{tS}]}で発音される。

\begin{tabular}{llll}
    & \textipa{titEa [\t{tS}itEa]} & 屋根 \\
\end{tabular}
% TODO ([\t{ts}]~[\t{tS}])の変異の条件

\subparagraph{R}
/r/は歯茎はじき音\textipa{[R]}として発音される。

\begin{tabular}{llll}
    & \textipa{riri [RiRi]} & 音 \\
\end{tabular}

\subparagraph{K}
/k/は軟口蓋無声破裂音\textipa{[k]}として発音される。

\begin{tabular}{llll}
    & \textipa{keni [keni]} & 言葉 \\
\end{tabular}

\paragraph{ミニマルペア}

以下に、子音のミニマルペアを例示する。

\subparagraph{P-V}
 
% TODO 例
%\begin{tabular}{lllll}
%    & keni & 言葉 & keni & 言葉 \\
% \end{tabular}

\subparagraph{V-M}
 
% TODO 例

\subparagraph{T-R}
 
% TODO 例

\subparagraph{R-N}
 
% TODO 例

\subparagraph{P-T}
 
% TODO 例

\subparagraph{P-K}
 
% TODO 例

\subparagraph{T-K}
 
% TODO 例
\subsubsection{母音}\paragraph{母音の発音}\quad\\
以下に、各母音の語内で実現される音声について説明する。

\subparagraph{/a/}
/a/は後舌広母音\textipa{[A]}として発音される。
% TODO 例

\subparagraph{/\textipa{E}/}
/\textipa{E}/は前舌半広母音\textipa{[E]}として発音される。
% TODO 例

\subparagraph{/e/}
/e/は前舌半狭母音\textipa{[e]}として発音される。
% TODO 例

\subparagraph{/i/}
/i/は前舌狭母音\textipa{[i]}として発音される。
% TODO 例

\subparagraph{/\textipa{O}/}
/\textipa{O}/は後舌半広母音\textipa{[O]}として発音される。
% TODO 例

\subparagraph{/o/}
/o/は後舌半狭母音\textipa{[o]}として発音される。
% TODO 例

\subparagraph{/u/}
/u/は前舌狭母音\textipa{[u]}として発音される。
% TODO 例

% TODO 異音の記述の充実
\subsubsection{声調}\paragraph{声調の発音}
\langname では同じ声調の音節でも、語内の位置によって異なる音高を持つ。
そのため、接頭辞が付いた場合、語内の位置がずれることによって各音節の音高は変化する。
また、声調とは別に語全体に音声的なダウンステップが生じるので、それとの組み合わせで音声に現れる音高は単純な高低ではなくなる。
% TODO IPA表記(コンパイル成功後)

\begin{tabular}{llll}
    & \'at\'ir\'i [\textipa{A}\super 5\textipa{ti}\super 5\textipa{Ri}\super 3] & 魚 \\
    & \'i-\'at\'ir\'i [i\super5\textipa{A}\super5\textipa{ti}\super4\textipa{Ri}\super2] & 大きい魚 \\
\end{tabular}

\subparagraph{高声調}
語頭の音節では、高声調は高い音程で発音される。
2番目以降の音節では、前の音節の声調によって音程が変わる。
前の声調が低声調のとき、前の音節と同じかわずかに高い音程で発音される。
前の声調が高声調のとき、音声的なダウンステップにより前の音節よりわずかに低い音程で発音される。

\begin{tabular}{llll}
    & \textipa{v\'E} [\textipa{BE}\super5] & 食べる \\
    & \textipa{v\'av\'E} [\textipa{BA}\super5\textipa{BE}\super4] & 3 \\
    & \textipa{\'OvEv\'a} [\textipa{O}\super4\textipa{BE}\super3\textipa{BA}\super2] & バナナ1房 \\
\end{tabular}

\subparagraph{低声調}
語頭の音節では、低声調は中ぐらいの音程で発音される。
2番目以降の音節では、音声的なダウンステップにより低声調は前の音節より低い音程で発音される。
高声調に続く低声調は音声的なダウンステップによるよりさらに低い音程で発音される。

\begin{tabular}{llll}
    & \textipa{nE} [\textipa{nE}\super2] & 持つ、使う \\
    & \textipa{k\'unE} [\textipa{ku}\super4\textipa{nE}\super2] & ネズミ \\
    & \textipa{meme} [\textipa{me}\super2\textipa{me}\super1] & 蛇 \\
\end{tabular}
\newpage

\section{統語論}本節では、\langname の文や句の構造について説明する。
\subsection{文}\input{section/文.tex}
\subsubsection{名詞文}\paragraph{名詞文の特徴}\quad\\
名詞句が主要部となる文を名詞文と呼ぶ。
名詞文は主要部が表すものの存在し、補部が表すものとの間に何らかの同一性があることを表す。

\paragraph{説明}
名詞文の補部に定性があり、主要部に定性がない場合、主要部は補部の性質を補足する。

\begin{exe}
    \ex \gll [u\'ak\'a ap\'a piku\'o] [XXX] \\
        父の名前 XXX \\
    \glt 父の名前はXXXだ。
\end{exe}
\begin{exe}
    \ex \gll [k\'ap\'a] [k\'um\'am\textipa{E}] \\
        彼 病気 \\
    \glt 彼は病気だ
\end{exe}

\paragraph{発見}
名詞文の補部に定性がない場合、名詞文は補部が表すものの発見を表す。
多くの場合、主要部には定性があり、発見した内容を表す。

\begin{exe}
    \ex \gll [r\'i\'i mipop\'a] [XXX k\'a\'a] \\
        あの子供 XXXのもの \\
    \glt あれがXXXの子供だ
\end{exe}
% 「発見」の補部: 
% 例2の補部 [XXX k'a'a] は名詞句の構造(XXX の k'a'a、おそらく名詞句の修飾構造)を含んでいるようですが、主要部と補部の境界線をもう少し明確に定義すると、読者が文構造を把握しやすくなります。
\subsubsection{形容詞文}
\subsubsection{動詞文}
\subsubsection{文中の副詞句}
\subsection{名詞句}
\subsubsection{代名詞}
\subsubsection{類別名詞}
\subsubsection{一般名詞}
\subsubsection{部位名詞}
\subsubsection{形容詞句による修飾}
\subsubsection{動詞句による修飾}
\subsection{形容詞句}
\subsubsection{形容詞}
\subsubsection{名詞を含む形容詞句}
\subsubsection{動詞を含む形容詞句}
\subsection{動詞句}
\subsubsection{自動詞}
\subsubsection{他動詞}
\subsubsection{動詞連続}
\newpage

\section{形態論}
\subsection{類別名詞}
\subsubsection{数}
\subsection{動詞}
\subsubsection{相}
\subsubsection{他動詞化}
\subsubsection{可能}
\subsection{複合語}
\subsection{畳語}