文は主要部となる句と、それを補足するいくつかの句からなる。
主要部となる句は文末に置かれる。

% TODO 人名について考察
% 句構造を表す方法
\begin{exe}
    \ex \gll [u\'ak\'a ap\'a piku\'o] [XXX] \\
    父の名前 XXX \\
    \glt 父の名前はXXXだ。
\end{exe}
\begin{exe}
    \ex \gll [reap\'a] [t\'iv\'o-p\'a] \\
    私 歩いた \\
    \glt 私は歩いた。
\end{exe}

主要部となる句と他の句がどのような関係を持つかは、主要部となる句の種類による。

% 語順の明確化: 
% 記述からSVやSOVが強く示唆されますが、文法全体を通じて支配的な**句の構成要素の順序(例:SOV)**を明記すると、読者にとってさらに分かりやすくなります。
% 他動詞文の例の追加: 
% 現在の例は自動詞文(SV)と名詞述語文(例1)のみです。**他動詞文(S-O-V)**の例を一つ加えることで、この言語がSOV語順であることを確定的に示すことができ、文法記述としての説得力が増します。