\langname の文は、以下の3つの主要な要素から構成される。
\begin{itemize}
    \item \textbf{述語 (Predicate)}: 文の中心となる要素で、行為や状態を表す。通常は動詞がこの役割を担う。
    \item \textbf{項 (Argument)}: 述語が要求する必須の要素であり、行為の主体(主語)や対象(目的語)を示す。
    \item \textbf{補語 (Adjunct)}: 述語や文全体を修飾する要素で、時間、場所、様態などの付加的な情報を提供する。項とは異なり、文の必須要素ではない。
\end{itemize}

\subsection{基本的な語順}
\langname の基本的な語順は、\textbf{補語・項・述語}の順である 。
この語順は、文の主要な情報を伝える上で重要な役割を果たす。

\begin{exe}
    \ex \gll ki ra bi d'es-ek.\\
    男 TOP 女 殴る-PAST\\
    \glt 男は女を殴った。
\end{exe}
この例文では、`ki` (男) が主題を示す助詞 `ra` とともに項となり、`bi` (女) も項として機能している。文末の `d'es-ek` (殴った) が述語である。

---
\subsection{複数の要素}
1つの文には、複数の項や述語、補語を置くことができる 。

\subsubsection{複数の述語}
複数の述語が連続して現れることで、一連の動作を表現する。

\begin{exe}
    \ex \gll ki ra bi q'es-i bi-ak.\\
    男 TOP 女 殴る 蹴る-PAST\\
    \glt 男は女を殴り、蹴った。
\end{exe}
この例文では、`q'es-i`(殴る)と `bi-ak`(蹴った)という2つの述語が、`bi`(女)という1つの項に対して適用されている。

\subsubsection{複数の項}
複数の項が連続して現れることで、並列した複数の対象を表現することができる。この場合、助詞を使わずに項を並べることで、それらが単一の動詞に作用するまとまりとして機能する。

\begin{exe}
    \ex \gll ki ra bi baaki q'es-ek.\\
    男 TOP 女 子供 殴る-PAST\\
    \glt 男は女と子供を殴った。
\end{exe}
この例文では、`bi`(女)と `baaki`(子供)が連続して現れることで、これら2つの名詞が動詞`q'es-ek`(殴った)の対象となっていることを示している。

\subsubsection{複数の補語}
補語は文頭に置かれ、時間や場所など様々な情報を付加する。

\begin{exe}
    \ex \gll dwakus ki ra bi q'es-ek.\\
    昨日 男 TOP 女 殴る-PAST\\
    \glt 昨日、男は女を殴った。
\end{exe}
この例では、`dwakus` (昨日) が補語として文頭に置かれ、述語の行為が行われた時間を表している。