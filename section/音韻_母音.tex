\langname の母音は表\ref{tab:vowels}に示された5種類である。

\begin{table}[H]
    \centering
    \begin{tabular}{lccc}
        \toprule
        & \textbf{前舌} & \textbf{中舌} & \textbf{後舌} \\
        \midrule
        \textbf{狭母音} & \textipa{/i/} & & \textipa{/u/} \\
        \textbf{中母音} & \textipa{/e/} & & \textipa{/o/} \\
        \textbf{広母音} & & \textipa{/a/} & \\
        \bottomrule
    \end{tabular}
    \caption{\centering \langname の母音体系}
    \label{tab:vowels}
\end{table}

以下に、各母音の音素的な特徴と、単語中の出現例を挙げる。

% \textipa{i} \textipa{[i]} (彼) 
\begin{itemize}
    \item \textbf{\textipa{/i/}}: 前舌狭母音。
    \begin{itemize}
        \item 例: % TODO : 少なくとも一つの具体的な単語例(音素表記と音声表記)を挙げてください。
    \end{itemize}
% TODO : /i/ の異音(例:子音/R/の後での中央化[i]など)が存在するか確認し、その音声環境を明確に記述してください。

    \item \textbf{\textipa{/u/}}: 後舌狭母音。
    \begin{itemize}
        \item 例: % TODO : 少なくとも一つの具体的な単語例(音素表記と音声表記)を挙げてください。
    \end{itemize}
% TODO : /u/ が唇音(/m/, /p/, /B/)の前後や、軟口蓋音(/k/)の前後で円唇性が強くなったり(例:[\textipa{u}]→[\textipa{u}])、音素的な変化がないか確認してください。

    \item \textbf{\textipa{/e/}}: 前舌半狭母音。
    \begin{itemize}
        \item 例: % TODO : 少なくとも一つの具体的な単語例(音素表記と音声表記)を挙げてください。
    \end{itemize}
% TODO : 母音/e/がストレス(強勢)のない音節で中央化(例:[\textipa{@}])しないか確認してください。

    \item \textbf{\textipa{/o/}}: 後舌半狭母音。
    \begin{itemize}
        \item 
例: % TODO : 少なくとも一つの具体的な単語例(音素表記と音声表記)を挙げてください。
    \end{itemize}
% TODO : 母音/o/が狭母音(/u/)や広母音(/a/)との対立において、明確な音韻的地位を持つことを確認し、最小対(Minimal Pair)があれば言及してください。

    \item \textbf{\textipa{/a/}}: 中舌広母音。
    \begin{itemize}
        \item 例: % TODO : 少なくとも一つの具体的な単語例(音素表記と音声表記)を挙げてください。
    \end{itemize}
% TODO : /a/ が前舌子音(/t/, /n/)と後舌子音(/k/)の前後で舌の位置が変化する異音(例:前舌化[\textipa{a}]や後舌化[\textipa{A}])を持たないか確認し、記述してください。
\end{itemize}