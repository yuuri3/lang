\langname の母音は表\ref{tab:vowels}に示された5種類である。

\begin{table}[H]
    \centering
    \begin{tabular}{lccc}
        \toprule
        & \textbf{前舌} & \textbf{中舌} & \textbf{後舌} \\
        \midrule
        \textbf{狭母音} & \textipa{/i/} & & \textipa{/u/} \\
        \textbf{中母音} & \textipa{/e/} & & \textipa{/o/} \\
        \textbf{広母音} & & \textipa{/a/} & \\
        \bottomrule
    \end{tabular}
    \caption{\centering \langname の母音体系}
    \label{tab:vowels}
\end{table}

以下に、各母音の音素的な特徴と、単語中の出現例を挙げる。

% \textipa{i} \textipa{[i]} (彼) 
\begin{itemize}
    \item \textbf{\textipa{/i/}}: 前舌狭母音。
    \begin{itemize}
        \item 例: uairi \textipa{[waiRi]} (弓矢)
    \end{itemize}

    \item \textbf{\textipa{/u/}}: 後舌狭母音。
    \begin{itemize}
        \item 例: kuna \textipa{[k\super{h}una]} (時間)
    \end{itemize}

    \item \textbf{\textipa{/e/}}: 前舌中母音。
    \begin{itemize}
        \item 例: memi \textipa{[memi]} (蛇)
    \end{itemize}

    \item \textbf{\textipa{/o/}}: 後舌中母音。
    \begin{itemize}
        \item 例: voa \textipa{[Boa]} (食べる)
    \end{itemize}

    \item \textbf{\textipa{/a/}}: 中舌広母音。
    \begin{itemize}
        \item 例: rara \textipa{[RaRa]} (蔓)
    \end{itemize}
\end{itemize}
% TODO : 異音
% TODO : 最小対