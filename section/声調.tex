\paragraph{声調の発音}
\langname では同じ声調の音節でも、語内の位置によって異なる音高を持つ。
そのため、接頭辞が付いた場合、語内の位置がずれることによって各音節の音高は変化する。
また、声調とは別に、語全体の傾向として音高が連続的に下がっていくので、それとの組み合わせで音声に現れる音高は単純な高低ではなくなる。
% TODO IPA表記(コンパイル成功後)

\subparagraph{高声調}
語頭から2番目までの音節では、高声調は高い音程で発音される。
% TODO 例
3番目以降の音節では、前の音節の声調によって音程が変わる。
前の声調が低声調のとき、前の音節と同じかわずかに高い音程で発音される。
前の声調が高声調のとき、前の音節よりわずかに低い音程で発音される。
% TODO 例

\subparagraph{低声調}
語頭の音節では、低声調は中ぐらいの音程で発音される。
2番目以降の音節では、低声調は前の音節より低い音程で発音される。
% TODO 例