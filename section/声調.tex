\paragraph{声調の発音}
\langname では同じ声調の音節でも、語内の位置によって異なる音高を持つ。
そのため、接頭辞が付いた場合、語内の位置がずれることによって各音節の音高は変化する。
% TODO IPA表記(コンパイル成功後)

\subparagraph{高声調}
語頭から2番目までの音節では、高声調は高い音程で発音される。
% TODO 例
3番目以降の音節では、前の音節の声調によって音程が変わる。
前の声調が低声調のとき、前の音節と同じかわずかに高い音程で発音される。
前の声調が高声調のとき、前の音節よりわずかに低い音程で発音される。
% 提案: 
% この現象が体系全体にわたって規則的に起こるのか(Downdrift/Declination)、あるいは特定の声調の並びでのみ起こるのか(Downstep/Terracing)を明確に説明してください。
% TODO 例

\subparagraph{低声調}
語頭の音節では、低声調は中ぐらいの音程で発音される。
2番目以降の音節では、低声調は前の音節より低い音程で発音される。
% TODO 例