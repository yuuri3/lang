\paragraph{声調の発音}
\langname では同じ声調の音節でも、前後の環境によって異なる音高を持つ。
% 提案: 
% IPAの声調数体系 (Tone Letters) または声調線 (Contour Lines) を用いて音高変化を正確に表記してください。

\subparagraph{高声調}
語頭から2番目までの音節では、高声調は高い音程で発音される。
% 提案: 
% この**「語頭2音節」が音韻的な単位(特定の接頭辞など)に制約されるのか、あるいは単純な音節位置**に依存するのかを明確にしてください。また、声調の優勢が強勢と関連している場合は、強勢についても言及することを推奨します。
% TODO 例
3番目以降の音節では、前の音節の声調によって音程が変わる。
前の声調が低声調のとき、前の音節と同じかわずかに高い音程で発音される。
前の声調が高声調のとき、前の音節よりわずかに低い音程で発音される。
% 提案: 
% この現象が体系全体にわたって規則的に起こるのか(Downdrift/Declination)、あるいは特定の声調の並びでのみ起こるのか(Downstep/Terracing)を明確に説明してください。
% TODO 例

\subparagraph{低声調}
語頭の音節では、低声調は中ぐらいの音程で発音される。
2番目以降の音節では、低声調は前の音節より低い音程で発音される。
% TODO 例