\langname では、動詞の\textbf{他動詞化(使役化}を形態論的に標示する。
これは、使役者(causer)を伴う動詞句を生成する手段であり、自動詞的あるいは非使役的な動詞語幹に
適用され、動詞句に「〜させる」という使役の意味を加える。

使役化は、動詞の\textbf{語幹 (Stem)} と\textbf{相接辞 (Aspectual Suffix)} の間に
\textbf{使役接辞 \texttt{-se-}}を挿入することで実現される。

$$ \text{動詞語幹} + \texttt{-se-} + \text{相接辞} $$

以下の例文は、自動詞的な動詞語幹 (\texttt{run}) が使役接辞 \texttt{-se-} によって他動詞化し、
主語が動作主(Agent)から使役者(Causer)へと変化することを示す。

\begin{exe}
        \ex \gll kawa-k-u \textipa{PRi-ak}\\
        woman-CLF.human-NOM die-IMPR\\
        \glt 彼女は死んだ。
\end{exe}
\begin{exe}
        \ex \gll \textipa{Fiwu} kawa-k \textipa{k1k-i} se-ak, \textipa{PRi-se-ak} \\
        これ woman-CLF.human fire-INS do-IMPR die-CAUS-IMPR \\
        \glt 彼は彼女を焼き殺した。
\end{exe}

例文(\ref{ex:caus_auto}a)に示す\textbf{非使役形}は、動詞語幹 \textipa{PRi} に不定相接辞
 \textipa{-ak} が直接後続する構造をとる。
 これに対し、例文(\ref{ex:caus_auto}b)の\textbf{使役形}は、動詞語幹 \textipa{PRi} と不定相接辞
  \textipa{-ak} の間に\textbf{使役接辞 \textipa{-SE-} が挿入される}ことによって形成される。

この\texttt{-se-}の挿入により、動詞の\textbf{項構造 (Argument Structure)} が変化し、
動詞は主語(使役者)に加えて、使役の対象(ここでは\texttt{child-ACC})を
\textbf{対格 (Accusative)} の形で取るようになる。