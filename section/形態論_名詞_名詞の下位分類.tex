本言語において名詞は、その形態論的および統語論的な振る舞いの違いに基づき、
以下に定義する4つのクラスに分類される。
これらのクラスは、次節で詳細に記述する格および数による活用のパターンを決定する。

\paragraph{一般名詞}
人や物など、独立した実体を表す名詞であり、可算名詞と不可算名詞の両方を含む。本クラスの名詞は、対格のみに形態論的な活用形を持つ。
\begin{itemize}
\item 例: \textit{kato} (猫), \textit{tore} (木)
\end{itemize}

\paragraph{物質名詞}
水や空気、砂といった、個体として認識されにくい物質を表す名詞。一般名詞とは異なり、対格と具格の両方に活用形を持つ。
\begin{itemize}
\item 例: \textit{goro} (水), \textit{lino} (空気)
\end{itemize}

\paragraph{代名詞}
人称や指示を表す名詞。対格と与格に活用形を持ち、文中で特定の機能に特化して用いられる。
\begin{itemize}
\item 例: \textit{sao} (私), \textit{tayo} (彼/彼女)
\end{itemize}

\paragraph{類別詞}
名詞を分類する機能を持つ語彙であり、一般名詞や物質名詞と共起する。すべての格および数において活用形を持つ、形態論的に最も多様なクラスである。
\begin{itemize}
\item 例: \textit{duri} (動物類別詞), \textit{fara} (無生物類別詞)
\end{itemize}