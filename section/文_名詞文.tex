% 名詞が述語となる文を名詞文と呼ぶ。
% 述語としての名詞は項を取らない。

% \begin{exe}
%       \ex{\gll di ra tet.\\
%            それ TOP 蛇\\
%       \glt それは蛇だ。}
% \end{exe}

\langname において、名詞は文の述語として機能することがあり、このような文を\textbf{名詞文}と呼ぶ。名詞文は、主語となる対象が何であるか(同一性)、あるいはどのような性質を持つか(属性)を表現するために用いられる。

\paragraph{基本的な構造}
名詞文の基本的な語順は、「\textbf{項+助詞(ra)+述語となる名詞}」である。
この構造では、述語となる名詞は項を取らず、文の主題(項)と述語が等しい関係にあることを示す。

\begin{exe}
    \ex \gll di ra tet.\\
    それ TOP 蛇\\
    \glt それは蛇だ。
\end{exe}
この例文では、`di` (それ) が文の主題として提示され、`tet` (蛇) が述語としてその主題の同一性を定義している。

\begin{exe}
    \ex \gll akak ra aka-k.\\
    彼 TOP 彼-POS\\
    \glt 彼は彼のものだ。
\end{exe}
この例文は、代名詞が名詞文の述語として用いられる例である。`aka-k` (彼) が主題となり、`aka-k`に所有の接尾辞`-k`を伴うことで所有関係を表している。この場合、所有対象を指す`aka-k`が文の主題と等しいことを示している。

\paragraph{名詞文の役割}
名詞文は主に以下の2つの役割を果たす。
\begin{itemize}
    \item \textbf{同一性の提示}: 主題が何であるかを定義する。
    \begin{exe}
        \ex \gll q'eb ra wibaq.\\
        水 TOP 良い\\
        \glt 水は良い。
    \end{exe}
    この例では、`q'eb` (水) が持つ性質を述語 `wibaq` (良い) が示している。

    \item \textbf{属性の記述}: 主題が持つ性質や状態を述べる。
    \begin{exe}
        \ex \gll aka koi ra aba koi.\\
        あの 家 TOP 私の 家\\
        \glt あの家は私の家だ。
    \end{exe}
\end{itemize}
このように、名詞文は単純な語彙を用いて、複雑な概念的な関係性を表現する上で重要な役割を担っている。