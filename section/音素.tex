\langname は7つの子音音素、7つの母音音素に加え、高低2種類の声調を持つ。
以下の表は\langname の子音音素と母音音素をまとめた表である。

\begin{table}[H]
    \centering
    \begin{tabular}{lccc}
        \toprule
        & \textbf{両唇音} & \textbf{歯茎音} & \textbf{軟口蓋音} \\
        \midrule
        \textbf{鼻音} & \textipa{/m/ [m]} & \textipa{/n/ [n]} & \\
        \textbf{破裂音} & \textipa{/p/ [p]} & \textipa{/t/ [t]} & \textipa{/k/ [k]}  \\
        \textbf{摩擦音} & \textipa{/v/ [B]} & & \\
        \textbf{流音} & & \textipa{/r/ [R]} &  \\
      
  \bottomrule
    \end{tabular}
    \caption{\centering \langname の子音体系}
    \label{tab:consonants}
\end{table}

\begin{table}[H]
    \centering
    \begin{tabular}{lcc}
        \toprule
        & \textbf{前舌}  & \textbf{後舌} \\
        \midrule
        \textbf{狭母音} & \textipa{/i/ [i]} & \textipa{/u/ [u]} \\
        \textbf{半狭母音} & \textipa{/e/ [e]} & \textipa{/o/ [o]} \\
        \textbf{半広母音} & \textipa{/E/ [E]} & \textipa{/O/ [O]} \\
        \textbf{広母音} & & \textipa{/a/ [A]} \\
        \bottomrule
    \end{tabular}
    \caption{\centering \langname の母音体系}
    \label{tab:vowels}
\end{table}