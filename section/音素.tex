\langname は7つの子音音素、7つの母音音素に加え、高低2種類の声調を持つ。
以下の表は\langname の子音音素と母音音素をまとめた表である。

\begin{table}[H]
    \centering
    \begin{tabular}{lccc}
        \toprule
        & \textbf{両唇音} & \textbf{歯茎音} & \textbf{軟口蓋音} \\
        \midrule
        \textbf{鼻音} & \textipa{/m/} & \textipa{/n/} & \\
        \textbf{破裂音} & \textipa{/p/} & \textipa{/t/} & \textipa{/k/}  \\
        \textbf{摩擦音} & \textipa{/v/} & & \\
        \textbf{流音} & & \textipa{/r/} &  \\
      
  \bottomrule
    \end{tabular}
    \caption{\centering \langname の子音体系}
    \label{tab:consonants}
\end{table}
% 提案: 
% /v/ が両唇摩擦音として独立した音素である場合、その調音点と調音法を本文中(例:両唇摩擦音 /v/)に明記し、他の言語の一般的な傾向とは異なる\langname の特徴として強調してください。
% 提案: 
% \langname の /r/ がはじき音(\textipa{[ɾ]}または\textipa{[R]})として実現される場合、流音のカテゴリーは**「流音」ではなく「はじき音」または「ふるえ音/はじき音」**として分類した方が、より具体的に体系を表現できます。

\begin{table}[H]
    \centering
    \begin{tabular}{lcc}
        \toprule
        & \textbf{前舌}  & \textbf{後舌} \\
        \midrule
        \textbf{狭母音} & \textipa{/i/} & \textipa{/u/} \\
        \textbf{半狭母音} & \textipa{/e/} & \textipa{o} \\
        \textbf{半広母音} & \textipa{/E/} & \textipa{/O/} \\
        \textbf{広母音} & & \textipa{/a/} \\
        \bottomrule
    \end{tabular}
    \caption{\centering \langname の母音体系}
    \label{tab:vowels}
\end{table}
% 提案: 
% これは音素体系の表であるため、/i/ などと同様にスラッシュで囲み、\textipa{/o/} のように**音素表記(スラッシュ)**を使用してください。また、後舌半狭母音は通常 [o] で表されますが、IPAチャートに基づいて適切な音素記号を選択することが重要です。
% 提案: 
% この4段階の分類で正しい場合は、各音素(/e/, /E/, /o/, /O/)が実際にどのIPA記号に対応するか(例: /E/ は \textipa{[openo]} か \textipa{[textepsilon]} か)を本文中で一度明確に示してください。