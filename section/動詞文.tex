\paragraph{動詞文の特徴}
動詞句が主要部となる文を動詞文を呼ぶ。
動詞文において、補部は動作主、場所、時間などを表す。
それそれの補部の順序は自由であるが、一般的に時間→場所→動作主の順に置かれる。

\paragraph{動作主}
補部に人間や動物を表す名詞句が置かれると、その名詞句は動詞句の動作主と解釈される。

\begin{exe}
    \ex \gll [r\'ea p\'aap\'a] [titop\'e\'a] \\
        私の娘 寝ている \\
    \glt 私の娘は寝ている。
\end{exe}

\paragraph{場所}
補部に場所を表す名詞句が置かれると、その名詞句は動作の行われる場所と解釈される。

\begin{exe}
    \ex \gll [p\'aia p\'aia] [k\'aeop\'a] [v\'amip\'a] \\
        村 彼ら 戻った \\
    \glt 彼らは村に戻った。
\end{exe}

\paragraph{時間}
補部に時間を表す名詞句が置かれると、その名詞句は動作の行われる時間と解釈される。

\begin{exe}
    \ex \gll [p\'a\'iip\'a] [k\'aeop\'a] [ve\'ip\'a] \\
        昨日 彼ら 来た \\
    \glt 彼は昨日来た。
\end{exe}