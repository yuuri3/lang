本セクションでは、\langname の動詞の形態論を記述する。
\langname の動詞は、時制、相、法、態といった文法範疇に応じた高度に膠着的な形態変化を示す。
% TODO : 指摘点: 動詞の定義(例: 屈折によって識別される品詞、語幹の種類など)を導入部に簡潔に含めると、分析の基盤が明確になります。
これらの変化は、主に接尾辞によって実現される。
% TODO : 指摘点: 形態論の分析の前提として、動詞の**語幹(Stem)**や**語根(Root)**と**接辞(Affix)**の境界の定義をどこかで明確に述べることが、この概観で示唆されていても良いでしょう。
本節では、まず動詞の形態論的特徴を概観し、次節以降で各範疇の具体的な活用規則と語例について詳細に
記述する。