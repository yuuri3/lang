本稿では、\langname の動詞の形態論的特徴について論じる。
\langname の動詞は、時制、相、法、態といった文法範疇に応じた複雑な形態変化を示す。
これらの変化は、主に接尾辞によって実現される。
本節では、まず動詞の形態論的特徴を概観し、次節以降で各範疇の具体的な活用規則と語例について詳細に
記述する。