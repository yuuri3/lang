\langname の\textbf{類別名詞}は、一般名詞の数を表すだけでなく、名詞が持つ性質や意味合いを区別する役割も担う。
同じ名詞でも、後続する類別名詞が変わることで、指し示す対象が異なる場合がある。

\begin{itemize}
    \item \textbf{例}:
    \begin{exe}
        \ex \gll aki ba\\
            CLF.人間 子供\\
        \glt 子供
    \end{exe}
    
    \begin{exe}
        \ex \gll rei ba\\
            CLF.植物 新芽\\
        \glt 新芽
    \end{exe}
\end{itemize}

% このように、類別名詞は単なる数量表示以上の、分類的な機能を持っていると言えます。