名詞は、文中で以下の3つの主要な役割を担う。
これらの用法は、名詞が置かれる位置や、後続する助詞・接尾辞によって区別される。

\paragraph{主題 (Topic)}
文の主題となる名詞には、助詞`ra`が後続する。これは日本語の「は」に似た役割を持ち、文の焦点を示す。
他動詞の主語は、必ずこの主題として表される。

\begin{exe}
    \ex \gll akak ra biribiri-ak.\\
        彼 TOP 泳ぐ-PRES\\
    \glt 彼は泳ぐ。
\end{exe}

\paragraph{項 (Argument)}
述語が要求する要素であり、助詞を伴わない。以下の要素は、項として機能する。

\begin{itemize}
    \item \textbf{他動詞の目的語}
    \begin{itemize}
        \item 例:\textit{Wobir peak.}(魚を食べる。)
    \end{itemize}

    \item \textbf{自動詞の主語}
    \begin{itemize}
        \item 例:\textit{Akak biribiriak.}(彼が泳ぐ。)
    \end{itemize}
\end{itemize}

項は、動詞が示す動作の対象や場所などを表すが、文脈上明らかである場合は省略することが可能である。

\begin{exe}
    \ex \gll pe-ak.\\
        食べる\\
    \glt 食べる。
\end{exe}
この例文では、`pi`(食べる)という動詞のみで文が完結している。
これは、食べる対象が文脈から理解できるため、項が省略されている例である。

\paragraph{名詞の修飾}
名詞は他の名詞を修飾し、所有や帰属の関係を示すことができる。
この用法は、被修飾語の後に\textbf{属格接尾辞}`-a`を伴って現れる。

\begin{exe}
\ex \gll wobir-a buk\\
    魚-POS 目 \\
\glt 魚の目
\end{exe}

この例では、`wobir`(魚)が接尾辞`-a`を伴って\textbf{属格}となり、`buk`(目)を修飾することで、
「魚の目」という所有関係を表している。
% この接尾辞は、名詞を形容詞的に機能させる役割も持つ。

% 属格接尾辞 -a: 「名詞を形容詞的に機能させる役割も持つ」という説明は少し曖昧です。
% 文脈から「魚の目」のように、所有関係や帰属関係を表すことを意味していると分かりますが、
% これを「形容詞的」と表現すると、他の修飾の役割(例:色や形を修飾する形容詞)と混同される可能性があります。
% 「被修飾語に対して所有や帰属の関係を示す」、または**「連体修飾の役割を果たす」**といった表現の方が、
% より正確かもしれません。

\paragraph{副詞的用法}
名詞は、動詞を修飾する副詞として機能することもある。
副詞的用法の名詞は手段、場所などを表し、多くは接尾辞や助詞を伴う。

\begin{itemize}
    \item \textbf{手段}
    \begin{itemize}
        \item 例:\textit{Bri q\'esek.}(船で殴る。)
    \end{itemize}

    \item \textbf{場所}
    \begin{itemize}
        \item 例:\textit{Koi ake.}(家に行く。)
    \end{itemize}
\end{itemize}