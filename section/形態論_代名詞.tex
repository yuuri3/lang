\langname の代名詞システムは、代名詞幹 (Pronoun Stem)と類別詞 (Classifier) 
からなる複合体である。
代名詞幹は単独では機能せず、類別詞が接辞化することで初めて完全な代名詞句を形成する。

この構造において、代名詞が持つ格(対格など)や数(単数・複数)の情報は、付属する類別詞に
よって排他的に示される。
代名詞句の内部構造における語順は、「代名詞-類別詞」である。

以下の例文は、代名詞幹\textipa{Fe}が、類別詞により複数形と対格標識を担うことを示す。

\begin{exe}
    \ex \gll pa kui, vuae-i iat pa-i nea-u-i \\
        MOUNT sun 1.PL-ACC eat CLF.powder-INS give-PRG-IMP \\
        \glt 私たちの日ごとの糧を、今日もお与えください。
\end{exe}
% TODO : 例文

\paragraph{人称代名詞 (Personal Pronoun Stems)}
人称代名詞は、人間の指示に用いられる。
\langname の人称代名詞は三つの人称(1/2/3)と二つの数(単数/複数)からなる6種類である。
3人称代名詞を含め、代名詞は性差による区別を持たない。

\begin{table}[H]
    \centering
    \begin{tabular}{lcc}
        \toprule
        & 単数 & 複数  \\
        \midrule
        一人称 & ria & riai \\
        二人称 & vua & vuai \\
        三人称 & ka & kai \\
        \bottomrule
    \end{tabular}
    \caption{\centering \langname の人称代名詞}
    \label{tab:pronouns}
\end{table}

\paragraph{指示代名詞 (Demonstrative Pronoun Stems)}
本言語の指示代名詞は、島の地理を基準とした絶対座標系と話者からの距離を組み合わせた体系である。

\begin{itemize}
    \item \textipa{pa} : {山側}の方向にあるものを指す。
        \begin{exe}
            \ex \gll pa mu \\
            TO.MOUNTAIN CLF.animal \\
            \glt 山の方の動物
        \end{exe}
    \item \textipa{ta} : {海側}の方向にあるものを指す。
        \begin{exe}
            \ex \gll ta mu \\
            TO.SEA CLF.animal \\
            \glt 海の方の動物
        \end{exe}
    \item \textipa{ri} : 山と海に{垂直な方向}にあるものを指す。
    また、話者と聞き手に近いものを指す際にもこの形式が流用されることがある。
        \begin{exe}
            \ex \gll ri ki \\
            TO.SEA CLF.animal \\
            \glt (海岸沿いにある、あるいは)近くにある丸いもの
        \end{exe}
\end{itemize}