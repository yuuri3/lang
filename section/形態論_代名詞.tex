\langname の代名詞代名詞は単独では機能せず、類別詞が接辞化することで初めて完全な代名詞句を形成する。

この構造において、代名詞が持つ格(対格など)や数(単数・複数)の情報は、付属する類別詞に
よって排他的に示される。
代名詞句の内部構造における語順は、「代名詞 + 類別詞」である。

以下の例文は、代名詞幹\textipa{Fe}が、類別詞により複数形と対格標識を担うことを示す。

\begin{exe}
    \ex \gll pa kui, vuae-i iat pa-i nea-u-i \\
        MOUNT sun 1.PL-ACC eat CLF.powder-INS give-PRG-IMP \\
        \glt 私たちの日ごとの糧を、今日もお与えください。
\end{exe}
% TODO : 例文 : 記述と例文の間に深刻な不整合があります。
% 1. **代名詞幹\textipa{Fe}の検証**: 例文中の *vuae-i* は「代名詞幹 Fe」ではなく、人称代名詞 *vua* (2人称単数、表11行目) や *vuai* (2人称複数、表11行目) の変形か、または異なる代名詞に見えます。代名詞幹 \textipa{Fe} がどこにあるのか、また、この文でどのように機能しているのかを明記してください。
% 2. **例文の適切性**: 例文はキリスト教の「主の祈り」からの引用(日ごとの糧を)のように見えます。記述文法は、特定の宗教的・文語的な文脈ではなく、**日常的な会話**で最も頻繁に使用される構造を提示すべきです。日常的な代名詞の使用例に変更してください。
% 3. **グロスの修正**: *vuae-i* のグロスは *1.PL-ACC* となっていますが、表の人称代名詞は *ria* (1人称単数) と *riai* (1人称複数) です。このグロスは *1.PL-ACC* の意味を持つなら *riai-i* などであるべきであり、不整合を修正してください。

\paragraph{人称代名詞 (Personal Pronoun Stems)}
人称代名詞は、人間の指示に用いられる。
\langname の人称代名詞は三つの人称(1/2/3)と二つの数(単数/複数)からなる6種類である。
3人称代名詞を含め、代名詞は性差による区別を持たない。

\begin{table}[H]
    \centering
    \begin{tabular}{lcc}
        \toprule
        & 単数 & 複数  \\
        \midrule
        一人称 & ria & riai \\
        二人称 & vua & vuai \\
        三人称 & ka & kai \\
        \bottomrule
    \end{tabular}
    \caption{\centering \langname の人称代名詞}
    \label{tab:pronouns}
\end{table}
% TODO : 人称代名詞の記述を厳密化する。
% 1. **複合体との整合性**: 冒頭の記述(代名詞幹+類別詞)に基づき、表の項目(*ria*, *riai* など)が「完全な代名詞句」なのか「代名詞幹」のみなのかを明確にしてください。もし複合体であれば、どのようなデフォルトの類別詞(例:ゼロ類別詞 -∅ や human 類別詞 -k など)が含まれているのかを説明してください。
% 2. **包括性/排他性**: 複数の一人称代名詞が、聞き手を含む**包括 (Inclusive)** と、聞き手を含まない**排他 (Exclusive)** の区別を持つかどうかを確認し、持たない場合はその旨を明記してください。

\paragraph{指示代名詞 (Demonstrative Pronoun Stems)}
本言語の指示代名詞は、島の地理を基準とした絶対座標系と話者からの距離を組み合わせた体系である。

\begin{itemize}
    \item \textipa{pa} : {山側}の方向にあるものを指す。
        \begin{exe}
            \ex \gll pa mu \\
            TO.MOUNTAIN CLF.animal \\
            \glt 山の方の動物
        \end{exe}
    \item \textipa{ta} : {海側}の方向にあるものを指す。
        \begin{exe}
            \ex \gll ta mu \\
            TO.SEA CLF.animal \\
            \glt 海の方の動物
        \end{exe}
    \item \textipa{ri} : 山と海に{垂直な方向}にあるものを指す。
    また、話者と聞き手に近いものを指す際にもこの形式が流用されることがある。
        \begin{exe}
            \ex \gll ri ki \\
            TO.SEA CLF.animal \\
            \glt (海岸沿いにある、あるいは)近くにある丸いもの
        \end{exe}
\end{itemize}
% TODO : 指示代名詞の記述を修正・補強する。
% 1. **語彙的グロス**: 指示代名詞 *pa* のグロスは **TO.MOUNTAIN** ではなく、**DEM.MOUNTAIN** (Demonstrative: Mountainward) や **DEM.UPHILL** のように、それが代名詞であることを示すグロスに変更すべきです。
% 2. **例文の不整合**: 例文 (13) の *ri* のグロスは **TO.SEA** となっていますが、*ri* の説明は「山と海に**垂直な方向**」です。グロスを **DEM.TRANSVERSE** (垂直/横断方向) など、記述に合ったものに修正してください。
% 3. **類別詞の機能**: 例文 (12) のグロスは *CLF.animal* (動物類別詞) を示していますが、*CLF.animal* が指示代名詞にどのような格・数情報を付与しているかを説明してください(例:単数対格の動物など)。
% 4. **話者からの距離**: *ri* が「話者と聞き手に近いもの」を指すという記述は、**絶対座標系**(山・海)と**相対座標系**(話者からの距離)の混在を示唆しています。この2つの体系が**音韻論的**にどのように統合されているのか、あるいは、どちらかが優位なのかを明確に分析し記述してください。