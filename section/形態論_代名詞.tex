\langname の代名詞システムは、代名詞幹 (Pronoun Stem)と類別詞 (Classifier) 
からなる複合体である。
代名詞幹は単独では機能せず、類別詞が接辞化することで初めて完全な代名詞句を形成する。

この構造において、代名詞が持つ格(対格など)や数(単数・複数)の情報は、付属する類別詞に
よって排他的に示される。
代名詞句の内部構造における語順は、「代名詞-類別詞」である。

以下の例文は、代名詞幹\textipa{Fe}が、類別詞により複数形と対格標識を担うことを示す。

\begin{exe}
    \ex \gll pa kui, vuae-i iat pa-i nea-u-i \\
        MOUNT sun 1.PL-ACC eat CLF.powder-INS give-PRG-IMP \\
        \glt 私たちの日ごとの糧を、今日もお与えください。
\end{exe}
% TODO : ここ以上の内容を統語論に移動?

\paragraph{人称代名詞 (Personal Pronoun Stems)}
人称代名詞幹は、原則として{有生性の高い対象(人間}の指示に用いられる。
% TODO : 代名詞の有生性による対比は必要?
3人称代名詞を含め、代名詞は性差による区別を持たない。

\begin{itemize}
    \item \textbf{\textipa{ria}} : 1人称(話し手)
        \begin{itemize}
            \item \textipa{Fe-ak} % TODO
            \quad 「私(人間)」(単数)
        \end{itemize}
    \item \textbf{\textipa{vua}} : 2人称(聞き手)
        \begin{itemize}
            \item \textipa{pu-ak} % TODO
            \quad 「あなた(人間)」(単数)
    \end{itemize}
    \item \textbf{\textipa{ka}} : 3人称(第三者)
        \begin{itemize}
            \item \textipa{waba-k} % TODO
            \quad 「彼/彼女(人間)」(単数)
        \end{itemize}
\end{itemize}
% TODO : 表にする

\paragraph{指示代名詞 (Demonstrative Pronoun Stems)}
本言語の指示代名詞は、{山と海という地理的ランドマークを基準とした絶対座標系
 (Absolute Frame of Reference) に基づいて分類される}という際立った特徴を持つ。
% TODO : 地理的ランドマーク?
% TODO : 近接性との関係
% TODO : 図を用いた説明もあり

\begin{itemize}
    \item \textbf{\textipa{pa}}: {山側}の方向にあるものを指す。(基準点からの絶対方向)
        \begin{itemize}
            \item \texttt{ka-tet}
            \quad 「山方向にある動物」(類別詞 \texttt{-tet} は「動物」クラス)
        \end{itemize}
    \item \textbf{\textipa{at}}: {海側}の方向にあるものを指す。(基準点からの絶対方向)
        \begin{itemize}
            \item \texttt{b1-tet}
            \quad 「海方向にある動物」
        \end{itemize}
    \item \textbf{\textipa{ri}}: 山と海に{垂直な方向}(例:川に沿った方向)にあるものを
    指す{絶対方向形式}を基本とする。
    % TODO : 川に沿った方向ではない
    ただし、話者と聞き手に近いもの(近接性/Proximate)を指す際にもこの形式が流用されることがある。
        \begin{itemize}
            \item \texttt{fiwi-di}
            \quad 「(垂直方向にある、あるいは)近くにある丸いもの」(類別詞 \texttt{-di} は「丸いもの」クラス)
        \end{itemize}
\end{itemize}
% TODO : 表と詳細な説明
% TODO : 例文を追加