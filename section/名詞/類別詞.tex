\langname の類別詞を以下の表\ref{tab:clasifier}に示す。

\begin{table}[H]
    \centering
    \begin{tabular}{lcccccc}
        \toprule
        \multicolumn{2}{c}{人間/動物} & \multicolumn{2}{c}{植物} & \multicolumn{2}{c}{形状} \\
        \midrule
        ara & 男性 & ni & 植物 & vi & 容器 \\
        ava & 女性 & ama & 植物 & i & 容器 \\
        ana & 鳥 & hau & 食べられる植物 & & 部屋/建物 \\
        & 植物 & & & pa & 粉/不定形 \\
        & 道具 & & & vu & 丸いもの \\
        da & 魚 & & & vevo & 細いもの \\
        aa & 動物 & & & a & 立っている棒 \\
        & 虫 \\
        \bottomrule
    \end{tabular}
    \caption{\centering \langname の類別詞}
    \label{tab:clasifier}
\end{table}

\paragraph{人間/動物}

類別詞"ara"/"ava"は人間を表す。
"ara"は人間一般も表す。

\begin{tabular}{lll}
    \multicolumn{3}{l}{例}\\
    & hupi-ra & 父 \\
    & ve-va & 母 \\
\end{tabular}

類別詞"ana"は鳥を表す。"ana"は手に持つ道具や、全体を利用する植物も表す。

\begin{tabular}{lll}
    \multicolumn{3}{l}{例}\\
    & de-na & 小鳥の一種 \\
    & baia-na & マチェット \\
    & pari-na & サゴヤシ \\
\end{tabular}

類別詞"da"は魚を表す。

\begin{tabular}{lll}
    \multicolumn{3}{l}{例}\\
    & de-da & 魚の一種 \\
    & ba-da & 魚の一種 \\
\end{tabular}

類別詞"aa"はその他の動物を表す。

\begin{tabular}{lll}
    \multicolumn{3}{l}{例}\\
    & vura-a & 豚 \\
    & vivu-a & アリ \\
\end{tabular}

\paragraph{植物}

類別詞"ni"は小さな草本を表す。

\begin{tabular}{lll}
    \multicolumn{3}{l}{例}\\
    & mi-ni & 花 \\
\end{tabular}

類別詞"ama"は一部を利用する植物を表す。

\begin{tabular}{lll}
    \multicolumn{3}{l}{例}\\
    & vua-ma & ヤシ \\
\end{tabular}

類別詞"hau"は可食部のある植物を表す。
例外として、この類別詞は蛇も表す。

\begin{tabular}{lll}
    \multicolumn{3}{l}{例}\\
    & mari-hau & ヤム芋 \\
    & vepa-hau & 蛇 \\
\end{tabular}

\paragraph{形状}

類別詞"vi"は丸い容器を、"i"は四角い容器を表す。
また、"i"は部屋や建物も表す。

\begin{tabular}{lll}
    \multicolumn{3}{l}{例}\\
    & hui-vi & かご \\
    & hui-i & 四角いかご \\
    & hura-i & 宗教施設 \\
\end{tabular}

類別詞"pa"は特定の形状を持たない物体を表す。

\begin{tabular}{lll}
    \multicolumn{3}{l}{例}\\
    & vura-pa & 豚肉 \\
    & pari-pa & サゴヤシの粉 \\
\end{tabular}

類別詞"vu"/"vevo"/"a"はそれぞれ、丸いもの/細いもの/棒を表す。

\begin{tabular}{lll}
    \multicolumn{3}{l}{例}\\
    & vu-vu & 石 \\
    & ta-vevo & 蛇の一種 \\
    & tawi-a & 柱 \\
\end{tabular}

\paragraph{類別詞の用法}

類別詞は名詞に比べてより一般的な意味を表すために、単独で使われることもある。

\begin{exe}
    \ex \gll ni-o ara tai e-hai \\
    これ-所 男性 ~に 三人称-完了 \\
    \glt 誰かがそこに行った \\
\end{exe}

名詞にはそのカテゴリーに応じた類別詞が付くことがある。
類別詞が付くことによる意味の変化はないが、動作主を表す名詞には類別詞が付くことが多い。

\begin{exe}
    \ex \gll via-ra ni vuhau-e \\
    男-男性 女 殴る-三人称 \\
    \glt 男が女を殴る \\
\end{exe}

一部の名詞は類別詞の使い分けによって意味を変える。

\begin{tabular}{lll}
    \multicolumn{3}{l}{例} \\
    & vura-a & 生きている豚 \\
    & vura-pa & 死んでいる豚 \\
    & ba-na & 鶏 \\
    & ba-da & 魚の一種 \\
\end{tabular}