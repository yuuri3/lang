\langname は、7種類の人称代名詞と4種類の指示代名詞を持つ。

\paragraph{人称代名詞}
\label{pronouns}
\langname の人称代名詞を以下の表\ref{tab:pronouns}に示す。
代名詞は3つの人称と単数/複数の区別を持つ。
また、一人称複数の代名詞は除外(聞き手を含まない)、包括(聞き手を含む)の区別を持つ。

\begin{table}[H]
    \centering
    \begin{tabular}{lcc}
        \toprule
        & \textbf{単数} & \textbf{複数} \\
        \midrule
        \textbf{一人称除外} & vu & vuvo \\
        \textbf{一人称包括} & & vurou \\
        \textbf{二人称} & ti & tirou \\
        \textbf{三人称} & u & ua \\
        \bottomrule
    \end{tabular}
    \caption{\centering \langname の人称代名詞}
    \label{tab:pronouns}
\end{table}

\paragraph{指示代名詞}
\label{demonstrative}
\langname の指示代名詞を以下の表\ref{tab:pronouns}に示す。

\begin{table}[H]
    \centering
    \begin{tabular}{lcc}
        \toprule
        & \textbf{単数} & \textbf{複数} \\
        \midrule
        \textbf{近称} & ni & nirou \\
        \textbf{遠称} & to & torou \\
        \bottomrule
    \end{tabular}
    \caption{\centering \langname の指示代名詞}
    \label{tab:demonstrative}
\end{table}

\paragraph{代名詞の用法}

単独の代名詞は動作の対象や道具を表すときに用いられる。

\begin{exe}
    \ex \gll u ea ne-he-nai ! \\
    三人称 斧 二人称-渡す-開始 \\
    \glt 彼に斧を渡せ \\
\end{exe}

動作主を表すときは代名詞は類別詞を伴う。

\begin{exe}
    \ex \gll u-va dauti-dai-o e-hai \\
    三人称-女性 カヌー-行く-所 三人称-完了 \\
    \glt 彼女はカヌー乗り場に行った \\
\end{exe}