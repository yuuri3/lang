\paragraph{代名詞の複数形}

代名詞の複数形は\ref{pronouns}節、\ref{demonstrative}節に示した通りである。
代名詞の複数形は単独でしか使われず、
複数の語からなる名詞句に複数を表示するためには下記の類別詞の複数形が用いられる。

\paragraph{類別詞の複数形}

類別詞にの複数形は畳語で表される。
多くの類別詞であとの形態素は子音が弱化し、母音は二重母音となる。
一部の類別詞は不規則な複数形を持つ。
また、"pa"(粉、不定形なもの)は複数形を持たない。
類別詞の複数形を以下の表\ref{tab:clasifier_PL}に示す。

\begin{table}[H]
    \centering
    \begin{tabular}{lcclcclcc}
        \toprule
         & 単数形 & 複数形 && 単数形 & 複数形 && 単数形 & 複数形 \\
        \midrule
        男性 & ara & ararau & 植物 & ni & nirou & 粉 & pa & - \\
        女性 & ava & avavau & 植物 & ama & amavau & 丸いもの & vu & vuvu \\
        鳥 & ana & anarau & 食べられる植物 & hau & hahau & 細いもの & vevo & viva \\
        魚 & da & darau & 容器 & vi & vivou & 立っている棒 & a & aa \\
        動物 & aa & aei & 部屋 & i & ia \\
        \bottomrule
    \end{tabular}
    \caption{\centering 類別詞の複数形}
    \label{tab:clasifier_PL}
\end{table}

\paragraph{独立名詞の複数形}

ほとんどの独立名詞は語根を畳語にすることで複数を表示する。

\begin{exe}
    \ex \gll ni rawi-rawi ararau \\
    これ 家-家 男性.複数 \\
    \glt この村の人々
\end{exe}