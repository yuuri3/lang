\paragraph{代名詞の複数形}

代名詞の複数形は\ref{pronouns}節、\ref{demonstrative}節に示した通りである。

代名詞の複数形は単独でしか使われない。
複数形の代名詞に類別詞が付く場合、複数表示は類別詞に移動する。
所有者を表す代名詞が複数になる場合、代名詞は元の名詞から分離する。
\begin{exe}
    \ex \gll vu-rarau bia-rarau \\
    一人称複数-男性複数 子供-男性複数 \\
    \glt 私たちの子供たち
\end{exe}
\begin{exe}
    \ex \gll *vuvo-bia-rarau \\
    一人称複数-子供-男性複数 \\
    \glt 私たちの子供たち
\end{exe}

\paragraph{類別詞の複数形}

類別詞にの複数形は畳語で表される。
多くの類別詞であとの形態素は子音が弱化し、母音は二重母音となる。
一部の類別詞は不規則な複数形を持つ。
また、"pa"(粉、不定形なもの)は複数形を持たない。
類別詞の複数形を以下の表\ref{tab:clasifier_PL}に示す。

\begin{table}[H]
    \centering
    \begin{tabular}{lcclcclcc}
        \toprule
         & 単数形 & 複数形 && 単数形 & 複数形 && 単数形 & 複数形 \\
        \midrule
        男性 & ara & ararau & 植物 & ni & nirou & 粉 & pa & - \\
        女性 & ava & avavau & 植物 & ama & amavau & 丸いもの & vu & vuvu \\
        鳥 & ana & anarau & 食べられる植物 & hau & hahau & 細いもの & vevo & viva \\
        魚 & da & darau & 容器 & vi & vivou & 立っている棒 & a & aa \\
        動物 & aa & aei & 部屋 & i & ia \\
        \bottomrule
    \end{tabular}
    \caption{\centering 類別詞の複数形}
    \label{tab:clasifier_PL}
\end{table}

\paragraph{独立名詞の複数形}

ほとんどの独立名詞は語根を畳語にすることで複数を表示する。

\begin{exe}
    \ex \gll ni ravi-ravi ararau \\
    これ 家-家 男性.複数 \\
    \glt この村の人々
\end{exe}