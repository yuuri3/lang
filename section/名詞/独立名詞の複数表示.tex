代名詞と類別詞以外の独立名詞では、複数は以下の2通りの方法で示される。

\begin{itemize}
    \item "rerei" の後置
    \item 畳語
\end{itemize}

\paragraph{畳語}

ほとんどの独立名詞は語根を畳語にすることで複数を表示する。

\begin{exe}
    \ex \gll ni rawi-rawi ararau \\
    これ 家-家 男性.複数 \\
    \glt この村の人々
\end{exe}

\paragraph{"rerei"}

数が非常に多いとき、複数のものの中から一つに着目するとき、目的語として使われるときなど、
複数の表示に"rerei"が使われる。

\begin{exe}
    \ex \gll ararau mu-aei naimau rerei nou-e-ra \\
    男性.複数 豚-動物.複数 犬 複数 狩る-三人称-現在 \\
    \glt 人々は豚と犬を狩っている
\end{exe}

"rerei" は代名詞、類別詞の単数形とも上記の用法で用いられる。

\begin{exe}
    \ex \gll ni rawi rerei me-avai \\
    これ 家 複数 一人称-いる \\
    \glt 私はこの村に住んでいる
\end{exe}