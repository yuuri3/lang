\langname の名詞は、単独で用いられる場合、特定の個体ではなく、種類全体や不特定の対象を指すため、
数を表示しない。しかし、類別詞を伴うことで単数・複数を区別する。

数は、類別詞に付加される接尾辞によって表現される。

\begin{itemize}
    \item \textbf{単数}: \texttt{-∅}
        \begin{itemize}
            \item 単数形は、特に接尾辞を付加せず、ゼロ形態素(zero morpheme)によって示される。
        \end{itemize}
    \item \textbf{複数}: \texttt{-'ey}
        \begin{itemize}
            \item 複数接尾辞の使用は任意である。
            \item 数詞が名詞句に含まれる場合、複数接尾辞は付加されない。
        \end{itemize}
\end{itemize}

これらの接尾辞は、類別詞の直後に付加される。

\begin{exe}
\ex
\gll paup tet-0 \\
木 CLF.動物-SG \\
\glt 一本の木

\ex
\gll paup tet-'ey \\
木 CLF.動物-PL \\
\glt 複数の木

\ex
\gll pawe bwe et \\
二 CLF.動物 \\
\glt 犬二匹

\ex
\gll *pawe bwe et-'ey \\
二 CLF.動物-PL \\
\glt 犬二匹 (非文法的)
\end{exe}