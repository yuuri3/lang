副詞句は文中の各所に置かれ、文の意味を補足する。

\paragraph{文頭に置かれる副詞句}
場所、時間などの、文全体のシチュエーションを説明する副詞句は文頭に配置される。

\begin{exe}
    \ex \gll [mip\textipa{O}p\'a] [p\textipa{O}p\'a] [ap\'a] \\
        子供のとき 島 いた \\
    \glt 子供のとき、私は島にいた。
\end{exe}
% TODO 副詞化
% TODO 場所を表す副詞句

\paragraph{主要部の前に置かれる副詞句}
動作や状態の度合いや動きの方向などの、主要部を補足する副詞句は主要部の直前に配置される。

\begin{exe}
    \ex \gll [k\'ap\'a] [nani] [piti r\'ar\textipa{\'E}p\'a] \\
        彼 上に 口笛を吹いた \\
    \glt 彼は上を向いて口笛を吹いた。
\end{exe}
\begin{exe}
    \ex \gll [k\'ap\'a] [piti] [t\'iv\textipa{\'O}p\'a] \\
        彼 ゆっくり 歩いた \\
    \glt 彼はゆっくり歩いた。
\end{exe}

% 冒頭に「副詞句は文中の各所に置かれ」とあります。
% 現状では「文頭」と「主要部の直前」という2つの典型的な位置が示されています。もし、これら以外(例:文末など)に置かれる副詞句が(稀であっても)存在するならば、その旨を追記するか、あるいは冒頭の記述を「主な配置場所は文頭と主要部の直前である」のように限定すると、記述の正確性が増すかもしれません。