副詞句は文中の各所に置かれ、文の意味を補足する。

\paragraph{文頭に置かれる副詞句}
場所、時間などの、文全体のシチュエーションを説明する副詞句は文頭に配置される。

\begin{exe}
    \ex \gll [mip\textipa{O}p\'a] [p\textipa{O}p\'a] [ap\'a] \\
        子供のとき 島 いた \\
    \glt 子供のとき、私は島にいた。
\end{exe}
% TODO 副詞化

\paragraph{主要部の前に置かれる副詞句}
動作や状態の度合いや動きの方向などの、主要部を補足する副詞句は主要部の直前に配置される。

\begin{exe}
    \ex \gll [k\'ap\'a] [nani] [piti r\'ar\textipa{\'E}p\'a] \\
        彼 上に 口笛を吹く \\
    \glt 彼は上を向いて口笛を吹いた。
\end{exe}