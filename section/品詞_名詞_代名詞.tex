% 内容に関する指摘
% この文法書は、Markdown形式で非常にうまくまとめられています。
% 不自然な点や修正が必要な箇所は特にありませんが、より洗練された文法書にするための提案をいくつかさ
% せていただきます。

% 図の挿入に関する記述: 「``」というコメントは、LaTeXコード内でもそのままコメントとして残しました。
% 図を挿入する際には、graphicxパッケージを使って、`\includegraphics`コマンドで画像を挿入する必要
% があります。

% 例文の構成: 例文の日本語訳とDauwak語の間に、発音記号や語彙の解説(例:raは助詞、bake'は動詞など)
% を入れると、読者にとってより親切な文法書になります。

% これらの提案は、文法書をより本格的にするためのものです。現在の内容でも、文法書として十分に機能する
% 素晴らしい構成になっています。



\langname における代名詞は、人称だけでなく、話し手との相対的な位置関係や、概念の性質(現実か想像か)を同時に示す。\langname は3つの代名詞語根を持つ。

以下の表は\langname の代名詞語根とその用法をまとめた表である。

\begin{tabular}{ll}
\toprule
代名詞 & 用法 \\
\midrule
\textbf{aba} & 一人称\\
& 一人称の所有物\\
& 会話で着目しているもの \\
\addlinespace[1ex]
\textbf{ufu} & 二人称\\
& 話し手と聞き手に近いもの \\
\addlinespace[1ex]
\textbf{aka} & 三人称\\
& 海や山の方向にあるもの \\
\bottomrule
\end{tabular}

これらの代名詞は、名詞や動詞と結びつくことで、文の主語や目的語、所有物などを表現する。

% \section{代名詞の用法}

% \subsection{人称代名詞}
人称代名詞は、接尾辞\textbf{-k}を代名詞の語幹に付加することで形成される。この形式は、特定の個人を指す場合に用いられる。

\begin{itemize}
    \item \textbf{aba-k} : 私
    \begin{itemize}
        \item 例:\textit{Abak ra bak\'e.}(私は行く。)
    \end{itemize}
    \item \textbf{ufu-k} : あなた
    \begin{itemize}
        \item 例:\textit{Ufuk ra feak\'e po?.}(あなたは行きますか?)
    \end{itemize}
    \item \textbf{aka-k} : 彼、彼女
    \begin{itemize}
        \item 例:\textit{Akak ra \'er.}(彼は行った。)
    \end{itemize}
\end{itemize}

% \subsection{指示代名詞1}
指示代名詞は、単独で、または名詞の前に置かれることで、話し手や聞き手から見た対象物の位置関係を示す。

\begin{itemize}
    \item \textbf{aba} + 名詞 : 話し手の所有物
    \begin{itemize}
        \item 例:\textit{aba feak} (私の妻)
        \item 例:\textit{aba koi} (私の家)
    \end{itemize}
    \item \textbf{ufu} + 名詞 : 話し手と聞き手に近いもの
    \begin{itemize}
        \item 例:\textit{ufu feak} (そこの女性)
    \end{itemize}
    \item \textbf{aka} + 名詞 : 山や海の方向にあるもの
    \begin{itemize}
        \item 例:\textit{aka feak} (あそこの女性)
    \end{itemize}
\end{itemize}

これら3つの指示代名詞の体系は以下の図のような関係を持つ。
% \subsection{指示代名詞2}
\langname の代名詞は、物理的な位置だけでなく、概念的な位置も表現する。\textbf{aba}は現実の事象を指すことが多いが、\textbf{aka}は想像上のものや、話し手と聞き手の認識の外にある事柄を示すためにも用いられる。

\begin{itemize}
    \item \textbf{aba} + 名詞 : 会話で着目しているもの
    \begin{itemize}
        \item 例:\textit{Aba feak}(その女性)が話題になっており、\textbf{aba}を使ってその女性の行動を指す。
        \item \textit{Aba feak, aba bi ra koi q\'e bake.} ... (これは私の妻だ。彼女は)
    \end{itemize}
    \item \textbf{aka} + 名詞 : 想像上のもの
    \begin{itemize}
        \item 例:\textit{aka feak}(あの女性)
        \item 話し手と聞き手が、実在しない、あるいはその場にいない特定の人物について話す際に用いられる。これは、物語の登場人物や神話の存在を指す場合もある。
    \end{itemize}
\end{itemize}