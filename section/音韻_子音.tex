\langname の子音は表\ref{tab:consonants}に示された7種類である。

\begin{table}[H]
    \centering
    \begin{tabular}{lccc}
        \toprule
        & \textbf{両唇音} & \textbf{歯茎音} & \textbf{軟口蓋音} \\
        \midrule
        \textbf{鼻音} & \textipa{/m/} & \textipa{/n/} & \\
        \textbf{破裂音} & \textipa{/p/} & \textipa{/t/} & \textipa{/k/} \\
        \textbf{摩擦音} & \textipa{/B/} & & \\
        \textbf{流音} & & \textipa{/R/} & \\
      
  \bottomrule
    \end{tabular}
    \caption{\centering \langname の子音体系}
    \label{tab:consonants}
\end{table}

以下に、各子音の音素的な特徴と、単語中の出現例を挙げる。括弧内には音素の異音を示した。

\begin{itemize}
    \item \textbf{/m/}: 両唇鼻音。
    \begin{itemize}
        \item 例: mua [mua] (動物)
    \end{itemize}

    \item \textbf{/n/}: 歯茎鼻音。
    \begin{itemize}
        \item 例: ni [ni] (手で)
    \end{itemize}

    \item \textbf{/p/}: 無声両唇破裂音。
    \begin{itemize}
        \item 語頭では有気音[\textipa{p\super{h}}]として発音される。
        \item 例:  paniai \textipa{[p\super{h}aniai]} (家)
    \end{itemize}

    \item \textbf{/t/}: 無声歯茎破裂音。
    \begin{itemize}
        \item 語頭では有気音[\textipa{t\super{h}}]として発音される。
        \item 例: tue \textipa{[t\super{h}ue]} (船)
        \item 母音/i/, /e/の前後では後部歯茎破擦音[\textipa{tS}]として発音される。
        \item 例: tia \textipa{[tS\super{h}ia]} (鍋)
    \end{itemize}

    \item \textbf{/k/}: 無声軟口蓋破裂音。
    \begin{itemize}
        \item 語頭では有気音[\textipa{k\super{h}}]として発音される。
        \item 例: ka \textipa{[k\super{h}a]} (彼)
    \end{itemize}

    
\item \textbf{\textipa{/B/}}: 有声両唇摩擦音。
    \begin{itemize}
        \item 例: vavai \textipa{[BaBai]} (道)
    \end{itemize}
    
    \item \textbf{\textipa{/R/}}: 歯茎たたき音。
    \begin{itemize}
        \item 例: rari \textipa{[RaRi]} (唇)
    \end{itemize}
\end{itemize}