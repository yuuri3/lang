\langname の基数詞は以下の通りである。
\begin{itemize}
    \item 1: uea
    \item 2: vai
    \item 3: vavi
\end{itemize}

これらの数詞は、名詞、そしてそれに続く類別詞とともに使用される。
\begin{exe}
    \ex \gll uea api \\
    one CLF.human \\
    \glt 一人
\end{exe}
\begin{exe}
    \ex \gll vai tui mu \\
    two dog CLF.animal \\
    \glt 二匹の犬
\end{exe}

4以上の数詞は\langname の本来語にはない。
近隣言語から借用された数詞が使われることがある。

% TODO : 4以上の数詞の使用例

数詞が名詞句に含まれる場合、類別詞に複数接尾辞(-ia)は付加されない。
\begin{exe}
    \ex \gll kuaia mu-ia \\
    pig CLF.animal-PL \\
    \glt 豚たち
\end{exe}
\begin{exe}
    \ex \gll kuaia vavi-mu \\
    pig three-CLF.animal \\
    \glt 三頭の豚
\end{exe}
% TODO : グロスの表記の修正:例文(6)のグロス *vavi-mu* は「three-CLF.animal」となっていますが、数詞と類別詞が複合体であるならば、一語として *vavi.mu* のようにドットで区切るか、あるいは独立した語として *vavi mu* とグロスし、その統語的地位を明確にしてください。
% TODO : 複数形の抑制の検証:数詞がある場合でも、類別詞に複数接尾辞 *-ia* が付加されないことを示す**非文法的な例文**(例:*\gll kuaia vavi-mu-ia*)を提示し、この制約の厳密性を裏付けてください。