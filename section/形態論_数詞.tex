\paragraph{基数詞}
\langname の基数詞は以下の通りである。
\begin{itemize}
    \item 1: teq
    \item 2: pawe
    \item 3: a'ir
    \item 4: kaw
    \item 5: pi'i
\end{itemize}
% TODO : 他の数詞体系の記述
% 基数詞(1, 2, 3, 4, 5)のみが示されていますが、言語学の論文では以下の情報が不可欠です。
% 6以降の数字(特に10や20など)の形成規則(十進法か、五進法か、二十進法かなど)。
% **「0」**の語彙項目があるか、ある場合はどのような用法を持つか。

序数詞(1番目、2番目)、分配数詞(それぞれ2つずつ)など、他の種類の数詞がある場合は、それらについても言及が必要です。

これらの数詞は、名詞、そしてそれに続く類別詞とともに使用される。
\begin{itemize}
    \item \textit{teq-ak} 「一人」 (teq-CLF.人間)
    \item \textit{pawe bwe-et} 「二匹の犬」 (pawe bwe-CLF.動物)
\end{itemize}

\paragraph{用法}
数詞が名詞句に含まれる場合、類別詞に複数接尾辞(-'ey)は付加されない。
これは、数詞そのものが数量を明確に示しているためである。
% TODO : 複数接尾辞に関する分析の強化:
% 「数詞そのものが数量を明確に示しているため」複数接尾辞が付加されないという説明は妥当ですが、これが唯一の理由であることの証明が必要です。
% 例として、数詞を伴わない名詞句や、数詞以外の数量詞(例: 「多くの」「いくつかの」)が名詞句に含まれる場合に、この複数接尾辞がどのように振る舞うのか(付加されるか否か)を示す必要があります。
\begin{verbatim}
pawe bwe-et
二 犬-CLF.動物
「二匹の犬」
\end{verbatim}
% TODO : グロスの表記の修正