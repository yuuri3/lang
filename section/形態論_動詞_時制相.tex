\langname の動詞は、語幹に接尾辞を付加することで、相を標示する。
以下の接尾辞は、動詞語幹に直接付加される。

\begin{itemize}

    \item \textbf{-pi}: \textbf{完了相}(Perfective Aspect)\\
    動作が完了したことを示す。完了した過去の出来事を記述する際に用いられる。
    \begin{itemize}
        \item 例: tivoa-pi (行き終わる)
    \end{itemize}

    \item \textbf{-pei}: \textbf{未完了相}(Imperfective Aspect)\\
    動作が完了していないことを示す。
    現在の出来事や、習慣的な行為を記述する際にも用いられる。
    \begin{itemize}
    \item 例: tivoa-pei (行く)
    \end{itemize}

    \item \textbf{-u}: \textbf{開始相}(Inchoative Aspect)\\
    動作の開始を表す。
    \begin{itemize}
        \item 例: tivoa-u (行き始める)
    \end{itemize}

    \item \textbf{-e}: \textbf{未開始相}(Inchoative Aspect)\\
    動作がまだ始まっていないことを表す。
    動作の否定として使われることもある。
    \begin{itemize}
        \item 例: tivoa-e ((まだ)行かない)
    \end{itemize}

\end{itemize}