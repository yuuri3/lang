\langname の動詞は、語幹に接尾辞を付加することで、時制(出来事の時間的な位置)と相(動作の開始、継続、完了)を同時に標示する。以下の接尾辞は、動詞語幹に直接付加される。

\begin{itemize}
\item \textbf{-ak}: \textbf{不定相}(Imperfective Aspect)\
特定の時制や相を強調しない、最も一般的な動詞形である。現在の出来事や、習慣的な行為を記述する際に用いられる。
\begin{itemize}
\item 例: ``tobi\textbf{ak}'' /tobiak/ (行く)
\end{itemize}

\item \textbf{-qi}: \textbf{開始相}(Inchoative Aspect)\\
動作の開始を表す。
\begin{itemize}
    \item 例: ``tobi\textbf{qi}'' /tobiqi/ (行き始める)
\end{itemize}

\item \textbf{-ak'e}: \textbf{進行相}(Progressive Aspect)\\
動作が進行中であることを示す。
\begin{itemize}
    \item 例: ``tobi\textbf{ak'e}'' /tobiak'e/ (行っている最中である)
\end{itemize}

\item \textbf{-ru}: \textbf{完了相}(Perfective Aspect)\\
動作が完了したことを示す。完了した過去の出来事を記述する際に用いられる。
\begin{itemize}
    \item 例: ``tobi\textbf{ru}'' /tobiru/ (行き終わる)
\end{itemize}

\item \textbf{-ako}: \textbf{経験相}(Experiential Aspect)\\
過去に一度でもその動作を経験したことがあることを示す。
\begin{itemize}
    \item 例: ``tobi\textbf{ako}'' /tobiako/ (行ったことがある)
\end{itemize}

\end{itemize}