動詞は、動作、状態、出来事を表す基本的な品詞である。
\langname の動詞は、名詞とは異なり、人称や時制を示すための接尾辞を伴って語形変化する。

\begin{tabular}{cc}
    \toprule
    \textbf{\langname} & \textbf{意味} \\
    \midrule
    pi & 食べる \\
    kikodi & うれしい \\ 
    biribiri & 泳ぐ \\
    \bottomrule
\end{tabular}

\subsubsection{動詞の用法}
動詞は文中で、主に以下の3つの役割を果たす。

\paragraph{動詞句の主部}
動詞は文の述語となり、行為や状態を直接表現する役割を担う。
\begin{exe}
    \ex \gll wobir ra biribiri-ak.\\
    魚 TOP 泳ぐ-PRES \\
    \glt 魚が泳ぐ。
\end{exe}
この例文では、`biribiri-ak` (泳ぐ) が動詞句の中心となり、主題である`wobir` (魚) の動作を表している。

\paragraph{名詞句の主部}
動詞は文の述語として、名詞文のように振る舞うことができる。この場合、動詞が主題の性質を定義する役割を担う。
\begin{exe}
    \ex \gll biribiri ra wobir.\\
    泳ぐ TOP 魚\\
    \glt 泳いでいるのは魚だ。
\end{exe}
ここでは、`biribiri` (泳ぐ) が述語となり、主題である`wobir` (魚) が「泳ぐという行為を行うもの」であることを示している。

\paragraph{名詞句の修飾}
動詞は、名詞の前に置かれることで、その名詞を修飾し、その名詞が行う動作や状態を説明することができる。
\begin{exe}
    \ex \gll biribiri wobir.\\
    泳ぐ 魚 \\
    \glt 泳いでいる魚。
\end{exe}
この例では、`biribiri` (泳ぐ) が名詞`wobir` (魚) を修飾し、「泳ぐ」という動作を行う「魚」であることを示している。これは英語における現在分詞の形容詞的用法に似ている。

\subsubsection{動詞の活用}
\langname の動詞は、人称と時制に応じて語形が変化する。これらの情報は、動詞の語幹に接尾辞を付加することで示される。

\begin{tabular}{llll}
    \toprule
    \textbf{語形} & \textbf{人称} & \textbf{時制} & \textbf{意味} \\
    \midrule
    kikodi-b-ak & 一人称 & 現在 & 私は好きだ \\
    kikodi-b-aiy & 一人称 & 過去 & 私は好きだった \\
    kikodi-p-uak & 二人称 & 現在 & あなたは好きだ \\
    \bottomrule
\end{tabular}

\subsubsection{名詞と動詞の転用}
\langname のいくつかの語は、語形を変えずに名詞と動詞の両方として使われる。この現象は、文脈によって語の品詞が決定されることを示している。

\begin{exe}
    \ex \gll koi-koi kusi bri-b-aiy.\\
    家-家 ~から 船-1-PAST \\
    \glt 私は船で村を出た。
\end{exe}
この例文では、`bri`が「船で進む」という意味の動詞として使われている。一方で、この語は「船」という名詞としても機能する。