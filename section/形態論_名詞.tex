名詞は、格と数によって活用する。その活用形の違いから、名詞は以下の4つのサブクラスに分類される。
% TODO : 活用
% TODO : 活用形の違い→取りうる格と数?
% TODO : サブクラス
本節では各サブクラスの基本的な活用特性を概観し、具体的な活用形については次節以降で詳細に記述する。
% TODO : 活用特性
ここで、◯は活用形が存在し、その文脈で用いられることを示し、×は該当する活用形が存在せず、
その文脈では使用されないことを示す。

\begin{itemize}
    \item \textbf{一般名詞}: 人や物などの独立した実体を表す名詞。対格のみに活用形を持つ。
    \item \textbf{部位名詞}: 人や物の一部分を表す名詞。水や空気、砂などの物質を表す名詞も含まれる。
    対格と具格に活用形を持つ。
    \item \textbf{代名詞}: 人称代名詞など。対格と与格に活用形を持つ。
    \item \textbf{類別詞}: 名詞を分類する接辞または独立した語。すべての格と数に活用形を持つ。
\end{itemize}

名詞のサブクラスごとの活用形は、以下の表\ref{tab:noun_classes}に示される。

\begin{table}[H]
\centering
\begin{tabular}{|l||c|c|c|c|c|}
\hline
\textbf{} & \textbf{対格} & \textbf{具格} & \textbf{与格} & \textbf{所有} & \textbf{複数} \\
\hline
\hline
一般名詞 & ◯ & × & × & × & × \\
\hline
部位名詞 & ◯ & ◯ & × & × & × \\
\hline
代名詞 & ◯ & × & ◯ & × & × \\
\hline
類別詞 & ◯ & ◯ & ◯ & ◯ & ◯ \\
\hline
\end{tabular}
\caption{\centering 名詞クラスごとの活用特性}
\label{tab:noun_classes}
\end{table}

\vspace{1em}

次節では、これらの各サブクラスの活用形について、より具体的な語例を挙げて解説する。