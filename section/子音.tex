\paragraph{子音の発音}\quad\\
\langname の子音は幅広い異音を持つ。
以下に、各子音の語内で実現される音声について説明する。

\subparagraph{/m/}
/m/は両唇鼻音\textipa{[m]}として発音される。
% TODO 例

\subparagraph{/p/}
/p/は両唇無声破裂音\textipa{[p]}として発音される。
% TODO 例

母音間では、/p/は摩擦音\textipa{[F]}として発音されることがある。
% TODO 例

\subparagraph{/v/}
/v/は両唇有声摩擦音\textipa{[B]}または両唇接近音\textipa{[\textlowering{B}]}で発音される。
% TODO 例
% TODO ([B] と [\textlowering{B}])が自由異音か条件異音か

\subparagraph{/n/}
/n/は歯茎鼻音\textipa{[n]}として発音される。
% TODO 例

\subparagraph{/t/}
/t/は歯茎無声破裂音\textipa{[t]}として発音される。
% TODO 例

母音/i//ye/の前では破擦音\textipa{[\t{ts}]}~\textipa{[\t{tS}]}で発音される。
% TODO 例
% TODO ([\t{ts}]~[\t{tS}])の変異の条件

\subparagraph{/r/}
/r/は歯茎はじき音\textipa{[R]}として発音される。
% TODO 例

\subparagraph{/k/}
/k/は軟口蓋無声破裂音\textipa{[k]}として発音される。
% TODO 例