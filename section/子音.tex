\paragraph{子音の発音}
\langname の子音は幅広い異音を持つ。
以下に、各子音の語内で実現される音声について説明する。

\subparagraph{/m/}
/m/は両唇鼻音\textipa{[m]}として発音される。

\begin{tabular}{llll}
    & \textipa{mipOa [miFOA]} & 芽 \\
\end{tabular}

\subparagraph{/p/}
/p/は両唇無声破裂音\textipa{[p]}として発音される。
母音間では、/p/は自由異音として摩擦音\textipa{[F]}で発音されることがある。

\begin{tabular}{llll}
    & \textipa{p\'Oia [pOiA]} & 家 \\
    & \textipa{up\'a [upA]~[uFA]} & 鳥 \\
\end{tabular}

\subparagraph{/v/}
/v/は両唇有声摩擦音\textipa{[B]}または両唇接近音\textipa{[\textlowering{B}]}で発音される。

\begin{tabular}{llll}
    & \textipa{v\'a\'E [BAE]~[\textlowering{B}AE]} & 2 \\
\end{tabular}
% TODO ([B] と [\textlowering{B}])が自由異音か条件異音か

\subparagraph{/n/}
/n/は歯茎鼻音\textipa{[n]}として発音される。

\begin{tabular}{llll}
    & \textipa{ninE [ninE]} & もらう \\
\end{tabular}

\subparagraph{/t/}
/t/は歯茎無声破裂音\textipa{[t]}として発音される。
母音/i//ye/の前では破擦音\textipa{[\t{ts}]}~\textipa{[\t{tS}]}で発音される。

\begin{tabular}{llll}
    & \textipa{titEa [\t{tS}itEa]} & 屋根 \\
\end{tabular}
% TODO ([\t{ts}]~[\t{tS}])の変異の条件

\subparagraph{/r/}
/r/は歯茎はじき音\textipa{[R]}として発音される。

\begin{tabular}{llll}
    & \textipa{riri [RiRi]} & 音 \\
\end{tabular}

\subparagraph{/k/}
/k/は軟口蓋無声破裂音\textipa{[k]}として発音される。

\begin{tabular}{llll}
    & \textipa{keni [keni]} & 言葉 \\
\end{tabular}

\paragraph{ミニマルペア}