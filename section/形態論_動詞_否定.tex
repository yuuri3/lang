動詞の意味内容を否定する場合、動詞語幹と相接辞の間に接辞 \textbf{-qr-} を挿入することで実現する。
この接辞は、動詞複合体において語幹の直後に現れる。
この構造は以下のスキーマで表される。
%
\begin{enumerate}
    \item $[ \text{V}_{\text{root}} ] - \textbf{qr} - [ \text{Aspect/Tense Marker} ] - [ \text{Agreement/Other Suffixes} ]$
\end{enumerate}
%
接辞 \textit{-qr-} は\textbf{否定相接辞}として機能し、動詞の表す動作や状態の非実現性、または否定をマークする。

以下の例は、動詞 \textit{qipu} 「行く」が否定接辞 \textit{-qr-} を伴うことで意味が否定されることを示す。

\begin{exe}
\ex \gll qipu-qr-ak \\
        go-NEG-ASP \\
        \glt `行かない'
\end{exe}
%
例文において、\textit{qipu} 「行く」 と相接辞 \textit{-ak} (相を示す接辞) の間に否定接辞 \textit{-qr-} が挿入され、「行かない」という否定の意味が構成されている。

% 補足:環境によっては、\section* ではなく \subsection* や \subsubsection* が適切かもしれません。
% 記述言語学では、通常、例文の表記に \ex... \gll... \glt... のような環境(ling-package など)を用います。