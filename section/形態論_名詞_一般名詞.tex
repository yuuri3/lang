人や物など、独立した実体を表す名詞である。

例
\begin{itemize}
\item \textipa{viri} (猫)
\item \textipa{tatuai} (木の一種)
\end{itemize}

一般名詞は数の区別をしない。
\begin{exe}
    \ex \gll ka-∅ pipoa runa-pi \\
        3.SG sago cut-IMPF \\
        \glt 彼はサゴを切る。
\end{exe}
\begin{exe}
    \ex \gll *ka-∅ pipoa-ia runa-pi \\
        3.SG sago-PL cut-IMPF \\
        \glt 彼はサゴを何本か切る。
\end{exe}
\begin{exe}
    \ex \gll ka-∅ pipoa upa-ia runa-pi \\
        3.SG sago CLF.plant-PL cut-IMPF \\
        \glt 彼はサゴを何本か切る。
\end{exe}
% TODO : 例文改良

一般名詞は格接辞を伴わない。
\begin{exe}
    \ex \gll ka-∅ pipoa voa-pi \\
        3.SG sago eat-IMPF \\
        \glt 彼はサゴを食べる。
\end{exe}
\begin{exe}
    \ex \gll *ka-∅ pipoa-na voa-pi \\
        3.SG sago-ACC eat-IMPF \\
        \glt 彼はサゴを食べる。
\end{exe}
% TODO : 具格の例文