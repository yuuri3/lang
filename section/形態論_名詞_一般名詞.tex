人や物など、独立した実体を表す名詞であり、数(単数・複数)の区別を持たない。
% 数への言及、文章リズム

例
\begin{itemize}
\item \textipa{viri} (猫)
\item \textipa{tatuai} (木の一種)
\end{itemize}

一般名詞は、対格を示す場合にのみ形態論的な活用形を持つ。
% 形態論的な活用形

\begin{exe}
    \ex \gll kai pipoa-i voa-pi \\
        3-CLF sago-ACC eat-IMPF \\
        \glt 彼はサゴを食べる。
\end{exe}