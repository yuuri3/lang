人や物など、独立した実体を表す名詞である。

例
\begin{itemize}
\item \textipa{viri} (猫)
\item \textipa{tatuai} (木の一種)
\end{itemize}

一般名詞は数の区別をしない。
% TODO : 例文 : 「数の区別をしない」ことを示す具体的な証拠が必要です。
% 1. 「多数」または「複数」の意味で名詞が使用されている文の例 (例: 多くの猫、木々) を提示してください。
% 2. 代わりに数詞や量詞(例: many, all)が数の情報を担うことを示す例を挙げてください。
% 3. 数に関する文法特徴(例: 集合名詞、可算/不可算の区別)が存在しないことを明記してください。

一般名詞は格接辞を伴わない。
\begin{exe}
    \ex \gll ka-∅ pipoa voa-pi \\
        3.SG sago eat-IMPF \\
        \glt 彼はサゴを食べる。
\end{exe}
\begin{exe}
    \ex \gll *ka-∅ pipoa-na voa-pi \\
        3.SG sago-ACC eat-IMPF \\
        \glt 彼はサゴを食べる。
\end{exe}
% TODO : 格接辞の記述を強化する。
% 1. 例文 (1) のグロスラインで、一般名詞 *pipoa* の後に **sago** ではなく **sago-ABS** や **sago-NOM** のように、それが基底形またはゼロ格(絶対格や主格など)であることを示すグロスを付与することを検討してください。これにより、格が存在しないのではなく、ゼロ形態で現れることが明確になります。
% 2. 例文 (2) の非文法性を示す格接辞 *-na* のソース(例:他の名詞クラスから借用されたもの、またはフィールドワークで確認された誤用など)を簡潔に説明してください。
% 3. **目的語**としてだけでなく、**主語**、**間接目的語**、**道具格**、**場所格**などの文法機能においても格接辞を伴わないこと、および、それらの機能が語順や前置詞(または後置詞)によって担われていることを明確に記述してください。