\langname の動詞はスロットⅠ~スロットⅣまでの4つの部分からなる。
それぞれの要素には以下の要素が含まれる。
また、動詞語根は語の内部の位置によって動詞Ⅰ~動詞Ⅲの3種類に分類される。

\begin{table}[H]
    \centering
    \begin{tabular}{ll}
        \toprule
        位置 & 要素 \\
        \midrule
        スロットⅠ & 動詞Ⅰ \\
        スロットⅡ & 人称接辞 \\
        スロットⅢ & 動詞Ⅱ \\
        & 使役 \\
        & 可能 \\
        スロットⅣ & 動詞Ⅲ \\
        & 時制/相 \\
        \bottomrule
    \end{tabular}
    \caption{\centering 動詞句の構造}
    \label{tab:verb_section}
\end{table}

それぞれの位置すべてに要素が入ることも可能であるが、必ずしもすべての位置に要素が入っている必要はない。

\begin{exe}
    \ex \gll vo-e-he-ra \\
    食べる-三人称-使役-現在 \\
    \glt 彼は食べさせている \\
\end{exe}
\begin{exe}
    \ex \gll me-vei \\
    一人称-持つ \\
    \glt 私は持つ \\
\end{exe}