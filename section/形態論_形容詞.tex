\begin{itemize}
    \item 形容詞は、格、数、人称などによって\textbf{形態論的な変化をしない}。
    \item 形容詞が修飾する名詞の格や数は、その名詞自体に付加される形態素によって示される。
\end{itemize}

\langname の\textbf{形容詞}は、修飾する名詞に\textbf{前置される}という厳格な統語的位置を占める。

\begin{itemize}
    \item \textbf{非屈折性 (Non-inflectional)}: 形容詞は、格、数、人称といった文法的範疇によって
    \textbf{形態論的な変化をしない}。
    これは、形容詞語幹にいかなる屈折形態素も付加されないことを意味する。
\end{itemize}

この形容詞の非屈折性は、修飾される名詞との\textbf{一致 (Agreement)} を示さない
という重要な形態統語的結果をもたらす。

\begin{itemize}
    \item \textbf{文法範疇の担い手}: 形容詞が修飾する名詞の格や数などの情報は、
    形容詞自体ではなく、\textbf{その名詞自体に付加される形態素によって示される}。
\end{itemize}

この構造は、\langname の文法システムにおける形態論的役割の明確な分担を示唆する。
すなわち、形容詞は専ら\textbf{意味的機能}(属性の付加)を担うのに対し、名詞は意味的機能に加え、
統語的機能(格、数)を担う屈折形態素を伴う。
これは、\langname が\textbf{名詞に焦点を当てた格と数の標示システム}を持つことを示唆している。