\langname の母音音素を表\ref{tab:vowels}に示す。

\begin{table}[H]
    \centering
    \begin{tabular}{lccc}
        \toprule
        & \textbf{前舌} & \textbf{中舌} & \textbf{後舌} \\
        \midrule
        \textbf{広母音} & & a & \\
        \textbf{半狭母音} & e & \multirow{2}{*}{u} & \multirow{2}{*}{o} \\
        \textbf{狭母音} & i & & \\
        \bottomrule
    \end{tabular}
    \caption{\centering \langname の母音体系}
    \label{tab:vowels}
\end{table}

\paragraph{/a/}
音素/a/は中舌広母音\textipa{[a]}で発音される。

\begin{tabular}{lll}
    viava & \textipa{[BiaBa]} & 大きい \\
    vea & \textipa{[Bea]} & 与える \\
\end{tabular}

両唇音/m/,/p/,/b/,/v/の後では、/a/は後舌母音\textipa{[A]}として発音される。

\begin{tabular}{lll}
    mana & \textipa{[mAna]} & 大きい \\
\end{tabular}

\paragraph{/e/}
音素/e/は前舌半狭母音\textipa{[e]}で発音される。

\begin{tabular}{lll}
    meda & \textipa{[me\super{n}da]} & 魚 \\
    bea & \textipa{[bea]} & ナイフ \\
\end{tabular}

\paragraph{/i/}
音素/i/は前舌狭母音\textipa{[i]}で発音される。

\begin{tabular}{lll}
    mia & \textipa{[mia]} & 手 \\
    vivi & \textipa{[BiBi]} & 乳房 \\
\end{tabular}

\paragraph{/o/}
音素/o/は後舌半狭母音\textipa{[o]}で発音される。

\begin{tabular}{lll}
    do & \textipa{[do]} & 鳥の一種 \\
    vevo & \textipa{[BeBo]} & 毛 \\
\end{tabular}

\paragraph{/u/}
音素/u/は中舌半狭母音\textipa{[@]}で発音される。

\begin{tabular}{lll}
    nua & \textipa{[n@a]} & 樹木の一種 \\
    du & \textipa{[d@]} & 星 \\
\end{tabular}

両唇音/m/,/p/,/b/,/v/の後では、/u/は後舌狭母音\textipa{[u]}として発音される。

\begin{tabular}{lll}
    mu & \textipa{[mu]} & 目 \\
    vivu & \textipa{[BiBu]} & アリ \\
\end{tabular}