\langname の子音音素を表\ref{tab:consonants}に示す。

\begin{table}[H]
    \centering
    \begin{tabular}{lcccc}
        \toprule
        & \textbf{両唇音} & \textbf{歯茎音} & \textbf{軟口蓋音} & \textbf{声門音} \\
        \midrule
        \textbf{鼻音} & m & n & & \\
        \textbf{無声破裂音} & p & \multicolumn{2}{c}{t} & \\
        \textbf{有声破裂音} & b & \multicolumn{2}{c}{d} & \\
        \textbf{摩擦音} & v & & \multicolumn{2}{c}{h} \\
        \textbf{流音} & & r & \\
        \bottomrule
    \end{tabular}
    \caption{\centering \langname の子音体系}
    \label{tab:consonants}
\end{table}

\paragraph{/m/}
音素/m/は両唇鼻音\textipa{[m]}と発音される。

\begin{tabular}{lll}
    mu & \textipa{[mu]} & 目 \\
    ama & \textipa{[AmA]} & 植物 \\
\end{tabular}

\paragraph{/p/}
音素/p/は無声両唇破裂音\textipa{[p]}と発音される。

\begin{tabular}{lll}
    pi & \textipa{[pi]} & 口 \\
    paei & \textipa{[pAei]} & キノコ \\
\end{tabular}

母音間では/p/は摩擦音\textipa{[F]}で発音される。

\begin{tabular}{lll}
    vepaei & \textipa{[BeFAei]} & 蛇 \\
\end{tabular}

\paragraph{/b/}
音素/b/は有声両唇破裂音\textipa{[p]}と発音される。

\begin{tabular}{lll}
    bi & \textipa{[bi]} & 舌 \\
    bea & \textipa{[bea]} & ナイフ \\
\end{tabular}

母音間では/b/は前鼻音化音\textipa{[\super{m}b]}で発音される。

\begin{tabular}{lll}
    nauba & \textipa{[nA\textsubarch{U}\super{m}bA]} & 弓 \\
\end{tabular}

\paragraph{/v/}
音素/b/は有声両唇摩擦音\textipa{[B]}または有声両唇接近音\textipa{[\textlowering{B}]}と発音される。

\begin{tabular}{lll}
    vevo & \textipa{[BeBo]} & 毛 \\
    hiva & \textipa{[hiBA]} & 木の一種 \\
\end{tabular}

\paragraph{/n/}
音素/n/は歯茎鼻音\textipa{[n]}と発音される。

\begin{tabular}{lll}
    niva & \textipa{[niBA]} & 女 \\
    mana & \textipa{[mAna]} & 花 \\
\end{tabular}

\paragraph{/t/}
音素/t/は無声歯茎破裂音\textipa{[t]}と発音される。

\begin{tabular}{lll}
    tu & \textipa{[t@]} & 唇 \\
\end{tabular}

/ota/,/ata/の環境では、/t/は軟口蓋音\textipa{[k]}で発音される。

\begin{tabular}{lll}
    atai & \textipa{[aka\textsubarch{I}]} & 地面に座る \\
\end{tabular}

母音/i/の後では、/t/は摩擦音\textipa{[s]}~\textipa{[S]}で発音される。

\begin{tabular}{lll}
    tita & \textipa{[tisa]} & 体 \\
\end{tabular}

\paragraph{/d/}
音素/d/は有声歯茎破裂音\textipa{[d]}と発音される。

\begin{tabular}{lll}
    de & \textipa{[de]} & 魚の一種 \\
\end{tabular}

母音/a/の前では、/d/は軟口蓋音\textipa{[g]}で発音される。

\begin{tabular}{lll}
    daia & \textipa{[ga\textsubarch{I}a]} & 動物 \\
\end{tabular}

母音間では/d/は前鼻音化音\textipa{[\super{n}d]}~\textipa{[\super{N}g]}で発音される。

\begin{tabular}{lll}
    meda & \textipa{[me\super{n}da]} & 魚 \\
    adai & \textipa{[a\super{N}ga\textsubarch{I}]} & 歩く \\
\end{tabular}

\paragraph{/r/}
音素/r/は歯茎はじき音\textipa{[R]}と発音される。

\begin{tabular}{lll}
    rari & \textipa{[RaRi]} & 水 \\
    rerai & \textipa{[ReRa\textsubarch{I}]} 電気 \\
\end{tabular}

\paragraph{/h/}
音素/h/は無声声門摩擦音\textipa{[h]}~軟口蓋音\textipa{[x]}と発音される。

\begin{tabular}{lll}
    hupira & \textipa{[h@piRa]} & 父 \\
    hiha & \textipa{[xiha]} 首 \\
\end{tabular}