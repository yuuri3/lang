\paragraph{名詞文の特徴}\quad\\
名詞句が主要部となる文を名詞文と呼ぶ。
名詞文は主要部が表すものの存在し、補部が表すものとの間に何らかの同一性があることを表す。

\paragraph{説明}
名詞文の補部に定性があり、主要部に定性がない場合、主要部は補部の性質を補足する。

\begin{exe}
    \ex \gll [u\'ak\'a ap\'a piku\'o] [XXX] \\
        父の名前 XXX \\
    \glt 父の名前はXXXだ。
\end{exe}
\begin{exe}
    \ex \gll [k\'ap\'a] [k\'um\'am\textipa{E}] \\
        彼 病気 \\
    \glt 彼は病気だ
\end{exe}

\paragraph{発見}
名詞文の補部に定性がない場合、名詞文は補部が表すものの発見を表す。
多くの場合、主要部には定性があり、発見した内容を表す。

\begin{exe}
    \ex \gll [r\'i\'i ap\'a] [XXX k\'a\'a mipop\'a] \\
        あの人 XXXの子供 \\
    \glt あれがXXXの子供だ
\end{exe}