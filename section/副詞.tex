\langname における副詞の主要な特徴は、その統語的位置と形態的な不変性にある。

% 記述言語学では、統語的位置を明確に定義することが重要です。
\langname の副詞は、主に動詞の直前に配置され、被修飾語(動詞)に先行する。
この位置は、動詞句内での標準的な副詞の配置を示す。

一方で、時間、場所など、より広い文脈に関わる一部の副詞は、文頭(節の最左端)に配置されることが
観察される。
これは、それらの副詞が文全体を修飾し、談話的な機能を持つことを示唆する。

% 語形変化がないことを「形態的な不変性」として表現し、厳密性を高めます。
\langname の副詞は、屈折的な語形変化(活用)を起こさない。
この特徴は、\langname における不変化詞(Particle)または不変化語(Invariable Word)としての
副詞の品詞的な性質を裏付けている。