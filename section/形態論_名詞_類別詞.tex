\langname の名詞は、指示詞や数詞と共起する際に、必ず特定の類別詞(classifier)を伴う。
以下のリストは、本言語の主要な類別詞をまとめたものであるが、完全なものではない。
% TODO : 完全なリストを構築

% TODO : 類別詞を接辞から独立した語にする。
\begin{itemize}
    \item \textbf{有生性}\\
    % TODO : 有生性→動物
    類別詞は、有生性階層において、種レベルに近い詳細な分類(鳥類・魚類など)を行う。
    % TODO : 有生性階層?
        \begin{itemize}
            \item \textbf{-ak}: 人間を指す。
                \begin{itemize}
                    \item \texttt{kak-ak} (子供-CLF.人間) % TODO
                    \quad 「一人の子供」
                \end{itemize}
            \item \textbf{-tet / -et}: 動物を指す。
            -tet の形態論的異形である -et は、先行する名詞の末尾が母音で終わる場合に付加される。
            類別詞-etに先行する名詞の語末母音/a/は、/e/へと変化する。
            % TODO : 異形態については別にまとめる
                \begin{itemize}
                    \item \texttt{paup-tet} (木-CLF.動物) % TODO
                    \quad 「木(生物としての木)」
                    \item \texttt{bwe-et} (犬-CLF.動物) % TODO
                    \quad 「犬(動物としての犬)」
                \end{itemize}
            \item \textbf{-i'k}: 鳥、爬虫類などを指す。
                \begin{itemize}
                    \item \texttt{kawa-i'k} (カエル-CLF.鳥) % TODO
                    \quad 「一羽のカエル」
                \end{itemize}
            \item \textbf{-wobir}: 魚を指す。
                \begin{itemize}
                    \item \texttt{q'eb-wobir} (魚-CLF.魚) % TODO
                    \quad 「一匹の魚」
                \end{itemize}
        \end{itemize}
    \item \textbf{植物} % TODO : 生物としてまとめる?
        \begin{itemize}
            \item \textbf{-i'k}: 樹木を指す。鳥、爬虫類などを指す類別詞と同型。
                \begin{itemize}
                    \item \texttt{paup-i'k} (木-CLF.樹木) % TODO
                    \quad 「一本の木」
                \end{itemize}
            \item \textbf{-ab}: 草を指す。
                \begin{itemize}
                    \item \texttt{aba-ab} (草-CLF.草) % TODO
                    \quad 「一束の草」
                \end{itemize}
            \item \textbf{t'ed}: 食べられる草を指す。% TODO
                \begin{itemize}
                    \item \texttt{t'ed-t'ed} (草-CLF.食べられる草) % TODO
                    \quad 「一株の食べられる草」
                \end{itemize}
        \end{itemize}
    \item \textbf{名詞の形状・性質}
        \begin{itemize}
            \item \textbf{di}: 丸いものを指す。
                \begin{itemize}
                    \item \texttt{koko-di} (石-CLF.丸いもの) % TODO
                    \quad 「一つの石」
                \end{itemize}
            \item \textbf{re}: 細長いものや道具を指す。
                \begin{itemize}
                    \item \texttt{w'ir-re} (槍-CLF.細長いもの) % TODO
                    \quad 「一本の槍」
                \end{itemize}
            \item \textbf{ii}: 束や集団を指す。
                \begin{itemize}
                    \item \texttt{ii-ii} (群れ-CLF.集団) % TODO
                    \quad 「一つの群れ」
                \end{itemize}
            \item \textbf{kiq}: 薄いものや曲がっているものを指す。
                \begin{itemize}
                    \item \texttt{yab-kiq} (葉-CLF.薄いもの) % TODO
                    \quad 「一枚の葉」
                \end{itemize}
            \item \textbf{kaw}: 粉を指す。
                \begin{itemize}
                    \item \texttt{pob-kaw} (土-CLF.粉) % TODO
                    \quad 「一握りの土」
                \end{itemize}
            \item \textbf{q'eb}: 水を指す。% TODO : 消す?
                \begin{itemize}
                    \item \texttt{q'eb-q'eb} (水-CLF.水)
                    \quad 「一杯の水」
                \end{itemize}
            \item \textbf{k'ik}: 火を指す。% TODO : 消す?
                \begin{itemize}
                    \item \texttt{k'ik-k'ik} (火-CLF.火)
                    \quad 「一つの炎」
                \end{itemize}
        \end{itemize}
\end{itemize}
これらの類別詞は、名詞句において数詞や指示詞の直後に付加される接尾辞として機能する。
% TODO : 接尾辞→独立した語

以下の節でその機能を列挙する。
% TODO : 以下の内容は統語論に移動

\paragraph{一般名詞の代替}
類別詞は、特定の範疇全体を指す名詞として機能する。
\begin{exe}
\ex \gll tet-a pap ra si pi-fi-qr-ak\\
        CLF.animal-GEN fat TOP by hand eat-1-NEG-PRES\\
\glt 私は動物の脂は食べない。
\end{exe}

\paragraph{指示性の付与}

\langname の名詞は、単独で用いられる場合、特定の個体ではなく、種類全体や不特定の対象を指す。
一方、類別詞が名詞に後続した場合、その名詞は話し手と聞き手の間で共有される特定の対象を指し示す。
ここで、類別詞は名詞句に特定性を付与する役割を担っている。
類別詞が常に名詞の直後に置かれる。
\begin{exe}
\ex \gll d\'ii t\'eow-i pi-fe-er\\
pig talk say-1-PL.PAST \\
\glt 私たちは豚の話をしていた。
\end{exe}
\begin{exe}
\ex \gll d\'ii et t\'eow-i pi-fe-er\\
pig CLF.animal talk say-1-PL.PAST \\
\glt 私たちはとある豚の話をしていた。
\end{exe}

\paragraph{数量詞・指示詞との併用}

数詞や指示詞が名詞に前置される場合、類別詞は名詞句の必須要素としてその名詞に後続する。
このとき、語順は指示詞-数詞-名詞-類別詞の順となり、ここでも類別詞は常に名詞の直後に置かれる。
\begin{exe}
\ex \gll pawe bwe et\\
二 犬 CLF.動物\\
\glt 犬二匹
\end{exe}
\begin{exe}
\ex \gll aba bwe et\\
これ 犬 CLF.動物\\
\glt この犬
\end{exe}


\paragraph{名詞の分類}

\langname の類別詞は、名詞が属する\textbf{意味的範疇}を決定する役割を持つ。
同じ名詞でも、後続する類別詞の違いにより、その指示対象の分類が変化する。
\begin{exe}
\ex \gll bwe et\\
犬 CLF.動物\\
\glt 犬(動物としての犬)
\end{exe}
\begin{exe}
\ex \gll bwa ak\\
犬 CLF.人間\\
\glt 犬(人間を指す際の比喩的な呼び方)
\end{exe}