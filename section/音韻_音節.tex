\langname の音節構造は、{極めてシンプルで制約的}である。
% TODO :「極めてシンプルで制約的」や「厳格であるにもかかわらず」といった主観的な表現
% ({} で囲まれた箇所)は、論文では避けるべきです。客観的な事実(例:許可される音節構造のタイプがVとCVの2種類のみである)を述べるに留めるべきです。
基本的には、開音節のみが許可され、子音クラスター(連続)や音節末子音(Coda)は認められない。
% TODO : 定義の明確化(子音連続・Coda)
% **「子音クラスター(連続)や音節末子音(Coda)は認められない」**という記述は、データに基づいた明確な証拠とともに提示されるべきです。この制約が音韻的な観察(例:形態素境界でのふるまい)からどのように裏付けられるのかを詳述する必要があります。

\langname において許可される音節構造は、以下の二つの基本的なタイプに限定される。

\begin{itemize}
    \item {V (母音のみ)}: 音節核 (Nucleus) のみからなる音節。
    \item {CV (子音+母音)}: 頭子音 (Onset) と音節核からなる音節。
\end{itemize}

\langname の音節構造は厳格であるにもかかわらず、形態論的あるいは統語論的な境界において、
{母音連続 (Vowel Sequence)} が高頻度で発生する点に音韻的な特徴が見られる。
% TODO : 母音連続の定義と種類
% 「母音連続 (Vowel Sequence) が高頻度で発生する」という記述は興味深いですが、それが何種類あるのか、どのような母音の組み合わせが許可されるのか(例: V1V2 は許可、V3V4 は禁止など)といった制約を記述することで、分析の価値が高まります。

母音連続は、単一の二重母音としてではなく、二つの独立した音節が隣接することで形成される。
% TODO : 音韻的証拠の不足
% 母音連続が「二つの独立した音節が隣接」していることを裏付ける、具体的な音韻的証拠(例:ストレスパターン、長音節・短音節の区別、音調のふるまい、特定の音韻規則がその境界でブロックされるか否かなど)の記述がありません。この証拠を示すことで、「単一の二重母音ではない」という主張の説得力が向上します。
この現象は、音節構造のシンプルさ(V/CVのみ)を維持するための音韻規則の回避として解釈できる。
% TODO : 「音韻規則の回避として解釈できる」という解釈
% 記述の後に別途考察セクションを設けるなどして分離し、より具体的な音韻規則(例:音節再構成規則)の提案や、なぜそれが「回避」であると解釈できるかの詳細な論拠を示すべきです。

以下は、{V} と {V}、および {CV} と {V} の組み合わせによって生じる
母音連続の例である。

\begin{exe}
    \ex \gll ta.o.i \\ % ta.o.i = 3音節
    \\
        \glt '水色の'
\end{exe}
\begin{exe}
    \ex \gll fe.i.a \\ % fe.i.a = 3音節
    \\
        \glt '彼らの家'
\end{exe}
% TODO : 音節区切りの明示
% 例文のグロス行(\gll)では、音節の区切りを明確に示すのが標準的な慣習です。現状の ta.o.i や fe.i.a のようなドット(.)による区切りは適切ですが、本文中でも音節の定義と区切り方が明記されているかを確認すべきです。
% 例文の下のコメント(% ta.o.i = 3音節)は、本文中で正式に説明されるべき情報です。
% TODO : 形態素境界の明示:
% TODO : グロス(\glt)の不備: