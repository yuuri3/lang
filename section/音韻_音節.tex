\langname の音節構造は、{極めてシンプルで制約的}である。
基本的には、開音節のみが許可され、子音クラスター(連続)や音節末子音(Coda)は認められない。

\langname において許可される音節構造は、以下の二つの基本的なタイプに限定される。

\begin{itemize}
    \item {V (母音のみ)}: 音節核 (Nucleus) のみからなる音節。
    \item {CV (子音+母音)}: 頭子音 (Onset) と音節核からなる音節。
\end{itemize}

\langname の音節構造は厳格であるにもかかわらず、形態論的あるいは統語論的な境界において、
{母音連続 (Vowel Sequence)} が高頻度で発生する点に音韻的な特徴が見られる。

母音連続は、単一の二重母音としてではなく、二つの独立した音節が隣接することで形成される。
この現象は、音節構造のシンプルさ(V/CVのみ)を維持するための音韻規則の回避として解釈できる。

以下は、{V} と {V}、および {CV} と {V} の組み合わせによって生じる
母音連続の例である。

\begin{exe}
    \ex \gll ta.o.i \\ % ta.o.i = 3音節
    \\
        \glt '水色の'
\end{exe}
\begin{exe}
    \ex \gll fe.i.a \\ % fe.i.a = 3音節
    \\
        \glt '彼らの家'
\end{exe}
% TODO : 例