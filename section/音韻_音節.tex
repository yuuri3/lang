\langname において許可される音節構造は、以下の二つの基本的なタイプに限定される。
基本的には、開音節のみが許可され、子音クラスター(連続)や音節末子音(Coda)は認められない。

\begin{itemize}
    \item {V (母音のみ)}: 音節核 (Nucleus) のみからなる音節。
    \item {CV (子音+母音)}: 頭子音 (Onset) と音節核からなる音節。
\end{itemize}

\langname において音節構造の厳格さが最も明確に示されるのは、形態素が接合する境界である。
ある語根が子音で終わる場合、それに続く形態素が子音で始まることは音韻的に許容されず、必ず介在母音
が挿入される。
これにより、潜在的に発生し得る音節末子音や子音クラスターが回避される。

\begin{exe}
    \ex \glll v-api \\
    v-pi \\
    eat-IMPF \\
        \glt 食べる
\end{exe}

音節間において、o+u,u+oを除く任意の母音が連続できる。
形態素間において上記の2母音が連続した場合、前の母音が省略される。

以下は、{V} と {V}、および {CV} と {V} の組み合わせによって生じる
母音連続の例である。

\begin{itemize}
    \item uai : 鼻
    \item noake : 音
\end{itemize}