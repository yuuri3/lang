% 基本的な制約の提示(主要命題)
% TODO : 指摘点: 記述言語学では「許可される/認められない」という表現よりも、「存在する/存在しない」「観察される/観察されない」といった、データに基づいたより客観的な表現が好まれます。
\langname において、開音節のみが許可され、子音クラスターや音節末子音は認められない。
また、二重母音も存在しない。
% 許可される音節構造の定義
% TODO : 指摘点: ここに「その結果」とありますが、これは上記の制約から論理的に導かれる「結果」であることをより明確に述べるべきです。
その結果、\langname において許可される音節構造は、以下の二つのタイプに限定される。

\begin{itemize}
    \item {V (母音のみ)}: 音節核 (Nucleus) のみからなる音節。
    \item {CV (子音+母音)}: 頭子音 (Onset) と音節核からなる音節。
\end{itemize}

% Coda/子音クラスター回避のメカニズム(介在母音挿入)
% TODO : 指摘点: このセクションの見出しは、記述対象の現象(例:「形態素境界での音節構造維持」や「介在母音挿入による音節制約の維持」)を具体的に示すとより明確です。
\langname において音節構造の厳格さが最も明確に示されるのは、形態素が接合する境界である。
% TODO : 指摘点: どの「介在母音」が挿入されるのか(例: /i/ や中立母音など)を明記することが、記述としては必須です。
ある語根が子音で終わる場合、それに続く形態素が子音で始まることは音韻的に許容されず、必ず介在母音
が挿入される。
これにより、潜在的に発生し得る音節末子音や子音クラスターが回避される。

\begin{exe}
    \ex \glll v-api \\
    v-pi \\
    eat-IMPF \\
        \glt 食べる
\end{exe}

% 母音連続の発生と制約(構造の帰結)
% TODO : 指摘点: 見出しに「構造の帰結」とありますが、この記述だけでは現象を指していません。「許容される母音連続と制約」など、現象を具体的に示す見出しが望ましいです。
% TODO : 指摘点: 母音連続の発生箇所(音節間または形態素間)を明記し、「母音連続が発生する」を「母音連続が観察される」などの客観的な表現に修正します。
{V} と {V}、および {CV} と {V} の組み合わせによって生じる音節間において、母音連続が発生する。
% TODO : 指摘点: ここで述べられている「o+u、u+oを除く」という制約は、なぜその二つだけが許容されないのかという**音韻論的動機付け**(例: 音高、円唇性など)を、可能であれば示唆するべきです。
o+u、u+oを除く任意の母音連続が許容される。

以下は、{V} と {V}、および {CV} と {V} の組み合わせによって生じる
許容される母音連続の例である。

\begin{itemize}
    \item u.a.i : 鼻 % TODO : 指摘点: 音節境界を明示する「.」の記法は適切ですが、国際音声記号(IPA)表記を併記するか、音韻表記であることを明記するのが一般的です。
    \item no.a.ke : 音
\end{itemize}

% 母音省略による制約の維持(形態音韻規則)
% TODO : 指摘点: 見出しを「母音省略(エリジオン)」や「形態素境界における母音省略規則」など、より具体的にし、単なる制約の「維持」ではなく**規則そのもの**を強調すべきです。
形態素間において上記の2母音が連続した場合、前の母音が省略される。
% TODO : 指摘点: 「上記の2母音」は、どの2母音を指すのかが不明確です。直前の「母音連続」を指すのであれば、「2つの母音」や「母音連続において」などと明確に述べるべきです。
\begin{exe}
    \ex \glll v-api \\
    v-pi \\
    eat-IMPF \\
   \glt 食べる
\end{exe}
% TODO : 指摘点: ここに具体的な例文を補足することで、この音韻規則がどのように適用されるかを示すことが記述言語学の論文では必須となります。