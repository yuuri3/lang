\langname において許可される音節構造は、以下の二つの基本的なタイプに限定される。
基本的には、開音節のみが許可され、子音クラスター(連続)や音節末子音(Coda)は認められない。

\begin{itemize}
    \item {V (母音のみ)}: 音節核 (Nucleus) のみからなる音節。
    \item {CV (子音+母音)}: 頭子音 (Onset) と音節核からなる音節。
\end{itemize}

\langname において音節構造の厳格さが最も明確に示されるのは、形態素が接合する境界である。
ある語根が子音で終わる場合、それに続く形態素が子音で始まることは音韻的に許容されず、必ず介在母音
が挿入される。
これにより、潜在的に発生し得る音節末子音や子音クラスターが回避される。

\begin{exe}
    \ex \glll v-api \\
    v-pi \\
    eat-IMPF \\
        \glt 食べる
\end{exe}

音節間において、o+u,u+oを除く任意の母音が連続できる。
形態素間において上記の2母音が連続した場合、前の母音が省略される。

以下は、{V} と {V}、および {CV} と {V} の組み合わせによって生じる
母音連続の例である。

\begin{itemize}
    \item uai : 鼻
    \item noake : 音
\end{itemize}

% TODO : 提案するセクション構成
% 以下の5つのステップでセクションを構成することで、論理的な流れが完成します。
% 基本的な制約の提示(主要命題)
% 許可される音節構造の定義
% Coda/子音クラスター回避のメカニズム(介在母音挿入)
% 母音連続の発生と制約(構造の帰結)
% 母音省略による制約の維持(形態音韻規則)

% TODO : 音韻規則の明確化
% 「音節間において、o+u, u+o を除く任意の母音が連続できる」
% これは、単一形態素内でも形態素境界でも、VVがV.Vとして許容されることを意味します。
% この「任意の母音」の連続が音節構造 (V/CV) の直接的な結果である点を強調します。
% 「形態素間において上記の2母音が連続した場合、前の母音が省略される」
% これは o または u が u または o に先行する場合、前の母音が脱落(エリジオン)し、結果として
% V2の V 音節のみが残るという形態音韻規則であることを明記すべきです。

% TODO : 例文の役割の明確化
% uai や noake の例文は、「o+u,u+o を除く母音連続が許可される」ことを示す対照的なデータとして提示し、
% それぞれの語がどのように音節分割されるかを明記する必要があります。