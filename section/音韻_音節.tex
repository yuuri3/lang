\langname において、開音節のみが許可され、子音クラスターや音節末子音は観察されない。
また、二重母音も存在しない。
上記の制約の結果、\langname において許可される音節構造は、以下の二つのタイプに限定される。

\begin{itemize}
    \item {V (母音のみ)}: 音節核 (Nucleus) のみからなる音節。
    \item {CV (子音+母音)}: 頭子音 (Onset) と音節核からなる音節。
\end{itemize}

{V} と {V}、および {CV} と {V} の組み合わせによって生じる音節間および形態素間において、
以下の表に記述された母音連続が観察される。
% TODO : 表を追加
以下は、{V} と {V}、および {CV} と {V} の組み合わせによって生じる
許容される母音連続の例である。
狭母音の後ろに母音が連続するとき、母音間に非音素的な半母音([w][j])が挟まる。
\begin{itemize}
    \item uai [u.wa.i] : 鼻
    \item noake [no.a.ke] : 音
\end{itemize}
% TODO : 母音連続のルールを整理 : 
% 4. 母音連続が「音節間および形態素間」で観察されるとあるため、それぞれの境界で母音連続がどのように生じ、音節構造がどのように変化しないか(例:形態素境界が音節境界と一致するか)について具体的な例を挙げて説明してください。