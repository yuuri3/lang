% 基本的な制約の提示(主要命題)
\langname において、開音節のみが許可され、子音クラスターや音節末子音は認められない。
また、二重母音も存在しない。
% 許可される音節構造の定義
その結果、\langname において許可される音節構造は、以下の二つのタイプに限定される。

\begin{itemize}
    \item {V (母音のみ)}: 音節核 (Nucleus) のみからなる音節。
    \item {CV (子音+母音)}: 頭子音 (Onset) と音節核からなる音節。
\end{itemize}

% Coda/子音クラスター回避のメカニズム(介在母音挿入)

\langname において音節構造の厳格さが最も明確に示されるのは、形態素が接合する境界である。
ある語根が子音で終わる場合、それに続く形態素が子音で始まることは音韻的に許容されず、必ず介在母音
が挿入される。
これにより、潜在的に発生し得る音節末子音や子音クラスターが回避される。

\begin{exe}
    \ex \glll v-api \\
    v-pi \\
    eat-IMPF \\
        \glt 食べる
\end{exe}

% 母音連続の発生と制約(構造の帰結)

{V} と {V}、および {CV} と {V} の組み合わせによって生じる音節間において、母音連続が発生する。
o+u、u+oを除く任意の母音連続が許容される。

以下は、{V} と {V}、および {CV} と {V} の組み合わせによって生じる
許容される母音連続の例である。

\begin{itemize}
    \item u.a.i : 鼻
    \item no.a.ke : 音
\end{itemize}

% 母音省略による制約の維持(形態音韻規則)

形態素間において上記の2母音が連続した場合、前の母音が省略される。
\begin{exe}
    \ex \glll v-api \\
    v-pi \\
    eat-IMPF \\
        \glt 食べる
\end{exe}
% TODO : 例示