\langname の名詞の格は、主に語尾に付加される\textbf{接尾辞}によって示される。
一つの接尾辞が複数の意味的・統語的機能を持つことが特徴である。
また、場所や方向を表す格は、特定の意味を含意する\textbf{助詞}によっても表現される。

\paragraph{接尾辞による格表示}

\langname の主要な格は、名詞の語尾に付く接尾辞によって区別される。

\begin{itemize}
    \item \textbf{-i / -∅}: この接尾辞は、以下に示す複数の格機能を持つ。
    \begin{enumerate}
    \item \textbf{対格}: 他動詞の直接目的語を示す。この機能は、名詞句が動詞句の先頭に置かれることによって示される。この語順の規則性により、対格の名詞句では接尾辞-iがしばしば\textbf{ゼロ形態素-0}として省略される。
    \begin{exe}
        \ex \gll Wobir-i peak. \\
        魚-ACC 食べる \\
        \glt 魚を食べる。
    \end{exe}
    \begin{exe}
        \ex \gll Wobir-0 peak. \\
        魚-ACC 食べる \\
        \glt 魚を食べる。
    \end{exe}

    \item \textbf{具格}: 動作の手段や道具を示す。\textbf{主格以外の複数の項を持つ他動詞句において、対格の名詞句に続く位置に置かれた場合}に具格を表す。
    \begin{exe}
        \ex \gll akak bwi-i q'es-ek. \\
        he stick-INS hit-PAST \\
        \glt 彼は棒で殴った。
    \end{exe}

    \begin{exe}
        \ex \gll akak bwi-i t'aq-0-i q'es-ek. \\
    彼 棒-INS 魚-ACC 殴る-PAST \\
        \glt 彼は魚を棒で殴った。
    \end{exe}

    \item \textbf{与格}: 間接目的語を示す。基本的に人間や動物といった有生性のある名詞が入る。\textbf{'si' (与える)や'kawisi' (餌をやる)といった贈与の意を表す動詞では、対格の名詞句に続く位置に置かれた場合}に与格を表す。
    \begin{exe}
        \ex \gll Akak-i riki-i-ak. \\
    彼-DAT 贈り物-渡す-PRES \\
        \glt 彼に贈り物を渡す。
    \end{exe}

    \begin{exe}
        \ex \gll Akak-i ab-i si-ek. \\
    彼-DAT 食物-ACC 与える-PAST \\
        \glt 彼は彼に食べ物を与えた。
    \end{exe}
\end{enumerate}

    \item \textbf{-a}: この接尾辞は、以下の格機能を持つ。
    \begin{itemize}
        \item \textbf{自動詞の主格}: 自動詞の主語を示す。
        \begin{exe}
        \ex \gll Akak-a biribiriak.\\
            彼-NOM 泳ぐ\\
        \glt 彼が泳ぐ。
        \end{exe}
        \item \textbf{属格}: 所有や帰属の関係を示す。
        \begin{exe}
        \ex \gll akak-a koi.\\
            彼-GEN 家\\
        \glt 彼の家。
        \end{exe}
    \end{itemize}

    \item \textbf{-oit}: この接尾辞は、場所を示す。
    \begin{itemize}
        \item \textbf{所格}: 動作が行われる場所を示す。
        \begin{exe}
        \ex \gll koi-oit biribiriak.\\
            家-LOC 泳ぐ\\
        \glt 家で泳ぐ。
        \end{exe}
    \end{itemize}
\end{itemize}

格接尾辞はしばしば省略されるが、具格、与格の\textbf{-i}、および所格の\textbf{-oit}において特に顕著である。
これらの省略された格は、文脈、または語順によってその役割が示される。

\paragraph{助詞による格表示}

場所や方向を表す格は、接尾辞に加えて、特定の意味的含意を持つ\textbf{助詞}によっても表現される。

\begin{itemize}
    \item \textbf{qew}: \textbf{向格}を示す。「~に向かって」の意。
    向かう方向が山と海の方向に対して垂直であることを含意する。
    \begin{exe}
    \ex \gll akak qew tobiak.\\
        彼 垂直.向格 行く\\
    \glt 彼は(川を渡って)行く。
    \end{exe}

    \item \textbf{qe}: \textbf{向格}を示す。「~に向かって」の意。
    向かう方向が山の方向であることを含意する。
    \begin{exe}
    \ex \gll akak qe tobiak.\\
        彼 山.向格 行く\\
    \glt 彼は(山の方向に)行く。
    \end{exe}
    
    \item \textbf{ri}: \textbf{向格}を示す。「~に向かって」の意。
    向かう方向が海の方向であることを含意する。
    \begin{exe}
    \ex \gll akak ri tobiak.\\
        彼 海.向格 行く\\
    \glt 彼は(海の方向に)行く。
    \end{exe}

    \item \textbf{dusi}: \textbf{通過格}を示す。「~を超えて」の意。
    対象を避けて向かうことを含意する。
    \begin{exe}
    \ex \gll akak dusi ki tobiak.\\
        彼 避けて.通過格 岩 行く\\
    \glt 彼は(岩を避けて)行く。
    \end{exe}
    
    \item \textbf{ibi}: \textbf{通過格}を示す。「~を通って」の意。
    対象を通過して向かうことを含意する。
    \begin{exe}
    \ex \gll akak ibi abi tobiak.\\
        彼 通過.通過格 森 行く\\
    \glt 彼は(森を通って)行く。
    \end{exe}
\end{itemize}