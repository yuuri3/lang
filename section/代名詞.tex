\paragraph{代名詞の一覧}
\langname の代名詞の一覧を表\ref{tab:pronouns}に示す。

\begin{table}[H]
    \centering
    \begin{tabular}{lcc}
        \toprule
        & 単数 & 複数 \\
        \midrule
        一人称 & r\'ea & reia \\
        二人称 & v\'oa & voia \\
        三人称 & k\'a & kaia \\
  \bottomrule
    \end{tabular}
    \caption{\centering \langname の代名詞}
    \label{tab:pronouns}
\end{table}

\paragraph{代名詞の異形}
% 【提案】
%  r\'eaが基本形であるならば、なぜ例文では rea や reap\'a に変化するのかを説明してください。
% 【提案】
%  この -p\'a が、修飾される代名詞に付くマーカーなのか、所有格などの格変化なのか、あるいは代名詞句の終わりを示す標識なのかを代名詞の異形セクションで明確に記述してください。

\paragraph{代名詞句}

代名詞はそれ単体で名詞句を構成することができる。

\begin{exe}
    \ex \gll [rea] [p\'a\'a t\'ii nep\'e\'a] \\
        私 料理する \\
    \glt 私は料理する。
\end{exe}

代名詞は形容詞句や動詞句によって修飾することができる。
名詞や名詞句によって代名詞を修飾することはできない。

\begin{exe}
    \ex \gll nup\'i reap\'a \\
        強い 私 \\
    \glt 強い人(=私)
\end{exe}
\begin{exe}
    \ex \gll \'at\'i reap\'a \\
        食べる 私 \\
    \glt 食べる人(=私)
\end{exe}
\begin{exe}
    \ex \gll *t\'ii reap\'a \\
        料理 私 \\
    \glt 料理の人(=私)
\end{exe}
% 例文の強化