\paragraph{代名詞の一覧}
\langname の代名詞の一覧を表\ref{tab:pronouns}に示す。

\begin{table}[H]
    \centering
    \begin{tabular}{lcc}
        \toprule
        & 単数 & 複数 \\
        \midrule
        一人称 & r\'ea & r\'eia \\
        二人称 & v\'oa & v\'oia \\
        三人称 & k\'a & k\'aia \\
  \bottomrule
    \end{tabular}
    \caption{\centering \langname の代名詞}
    \label{tab:pronouns}
\end{table}

\paragraph{代名詞の異形}

代名詞には接辞"-p\'a"が任意でつくことがある。
% TODO 例

\paragraph{代名詞句}

代名詞はそれ単体で名詞句を構成することができる。

\begin{exe}
    \ex \gll [r\'ea] [p\'a\'a t\'ii nep\'e\'a] \\
        私 料理する \\
    \glt 私は料理する。
\end{exe}

代名詞は形容詞句や動詞句によって修飾することができる。
名詞や名詞句によって代名詞を修飾することはできない。

\begin{exe}
    \ex \gll nup\'i r\'eap\'a \\
        強い 私 \\
    \glt 強い人(=私)
\end{exe}
\begin{exe}
    \ex \gll \'at\'i r\'eap\'a \\
        食べる 私 \\
    \glt 食べる人(=私)
\end{exe}
\begin{exe}
    \ex \gll *t\'ii r\'eap\'a \\
        料理 私 \\
    \glt 料理の人(=私)
\end{exe}
% 例文の強化