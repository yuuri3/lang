\paragraph{代名詞の一覧}
\langname の代名詞の一覧を表\ref{tab:pronouns}に示す。

\begin{table}[H]
    \centering
    \begin{tabular}{lcc}
        \toprule
        & 単数 & 複数 \\
        \midrule
        一人称 & r\'ea & r\'eia \\
        二人称 & v\'oa & v\'oia \\
        三人称 & k\'a & k\'aia \\
  \bottomrule
    \end{tabular}
    \caption{\centering \langname の代名詞}
    \label{tab:pronouns}
\end{table}

\paragraph{代名詞の異形}

代名詞には接辞"-p\'a"がつくことがある。
接辞"-p\'a"には以下の機能がある。

強調したい代名詞には接辞が付けられる。
% TODO 例文

修飾語のついた代名詞には接辞が付けられる。
% TODO 例文

\paragraph{代名詞句}

代名詞はそれ単体で名詞句を構成することができる。

\begin{exe}
    \ex \gll [r\'ea] [p\'a\'a t\'ii nep\'e\'a] \\
        私 料理する \\
    \glt 私は料理する。
\end{exe}

代名詞は形容詞句や動詞句によって修飾することができる。
名詞や名詞句によって代名詞を修飾することはできない。

\begin{exe}
    \ex \gll nup\'i r\'eap\'a \\
        強い 私 \\
    \glt 強い人(=私)
\end{exe}
\begin{exe}
    \ex \gll \'at\'i r\'eap\'a \\
        食べる 私 \\
    \glt 食べる人(=私)
\end{exe}
\begin{exe}
    \ex \gll *t\'ii r\'eap\'a \\
        料理 私 \\
    \glt 料理の人(=私)
\end{exe}
% 例文の強化

% 提案(必須):
% 修飾語と被修飾語の順序: 例文2と3から、修飾語(形容詞句/動詞句)が代名詞(被修飾語)の前に来る ([修飾語] [代名詞]) ことが分かります。この規則を文章で明記すると、統語構造の理解が深まります。
% 追加案の例: 「代名詞を修飾する形容詞句または動詞句は、代名詞に先行して配置される(修飾語-被修飾語の順)。」
% 例文の配置の修正: 例文2と3では、修飾語(nup'i 強い / at'i 食べる)が代名詞句の核である r'eap'a の前に来ているにもかかわらず、GLOSS(\gll)の行で順序が逆転していません。これは混乱を招くため、GLOSSの順序を実際の語順と一致させる必要があります。

% 現状(例文2):
% \gll nup'i r'eap'a \\ 強い 私

% 提案(例文2):
% \gll nup'i r'eap'a \\ 強い 私 $\rightarrow$ GLOSSは語順通りで問題ありませんでした。
% *ただし、nup'i r'eap'a の構造が [強い] [私] であり、「私」が Head(主要部)で「強い」が修飾語であることを明記した方が良いでしょう。

% 現状(例文3 誤り例):
% \gll *t'ii r'eap'a \\ 料理 私
% この例は「名詞による修飾は不可」を示すものですが、その禁止されている構造 ([名詞] [代名詞]) が、他の許可されている構造 ([形容詞/動詞句] [代名詞]) と同じHead-Finalであるため、一貫しています。このままの記載で問題ありません。